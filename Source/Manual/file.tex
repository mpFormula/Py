%% 
%% This is file, `file.tex',
%% generated with the extract package.
%% 
%% Generated on :  2015/07/31,19:36
%% From source  :  mpFormulaPy.tex
%% Using options:  active,generate=file,extract-cmd={chapter,section},extract-env={mpFunctionsExtract}
%% 
\documentclass[12pt,a4paper,openany]{book}

\begin{document}

\chapter{Preface}

\chapter{Introduction}

\section{Overview: Features and Setup}

\section{License}

\section{No Warranty}

\section{Related Software}

\chapter{Tutorials}

\section{Why multi-precision arithmetic?}

\section{Graphics using Latex}

\section{Graphics using .NET Framework}

\section{Eval, Options, Tables and Charts}

\begin{mpFunctionsExtract}
\mpFunctionOne
{Eval? String?  the result of the evaluation of an arithmetic expression, containing number and functions, but no variables.}
{Expression? String? an arithmetic expression.}
\end{mpFunctionsExtract}

\begin{mpFunctionsExtract}
\mpFunctionOne
{Options? String?  an identifier for a set of calculation options.}
{BaseOptions? String? an identifier for a set of base calculation options.}
\end{mpFunctionsExtract}

\begin{mpFunctionsExtract}
\mpFunctionOne
{Table? Range?  an identifier for a set of calculation options.}
{TableRef? String? a reference for a table.}
\end{mpFunctionsExtract}

\begin{mpFunctionsExtract}
\mpFunctionOne
{Chart? String?  an identifier for an XML Chart.}
{Data? Range? a reference for a data table.}
\end{mpFunctionsExtract}

\chapter{Python: Built-in numerical types}

\section{Truth Value Testing}

\section{Boolean Operations: and, or, not}

\section{Comparisons}

\section{Numeric Types - int, float, complex}

\section{Long integers}

\section{Fractions}

\section{Decimals}

\chapter{Basic Usage}

\section{Number types}

\section{Precision and representation issues}

\section{Conversion of formatted numbers}

\begin{mpFunctionsExtract}
\mpWorksheetFunctionTwoNotImplemented
{ROMAN? String? Converts an arabic numeral to roman, as text.}
{Number? mpNum? The Arabic numeral you want converted.}
{Form? Integer? A number from 0 to 4 specifying the type of roman numeral you want. The roman numeral style ranges from Classic to Simplified, becoming more concise as the value of form increases..}
\end{mpFunctionsExtract}

\begin{mpFunctionsExtract}
\mpWorksheetFunctionOneNotImplemented
{ARABIC? String? Converts an roman numeral to arabic, as text.}
{Number? mpNum? The roman numeral you want converted.}
\end{mpFunctionsExtract}

\begin{mpFunctionsExtract}
\mpWorksheetFunctionOneNotImplemented
{BIN2DEC? mpNum? Converts a binary number to decimal.}
{Number? mpNum? The binary number you want to convert. Number cannot contain more than 10 characters (10 bits). The most significant bit of number is the sign bit. The remaining 9 bits are magnitude bits. Negative numbers are represented using two's-complement notation.}
\end{mpFunctionsExtract}

\begin{mpFunctionsExtract}
\mpWorksheetFunctionTwoNotImplemented
{BIN2HEX? mpNum? Converts a binary number to decimal.}
{Number? mpNum? The binary number you want to convert. Number cannot contain more than 10 characters (10 bits). The most significant bit of number is the sign bit. The remaining 9 bits are magnitude bits. Negative numbers are represented using two's-complement notation.}
{Places? mpNum? The number of characters to use. If places is omitted, BIN2HEX uses the minimum number of characters necessary. Places is useful for padding the return value with leading 0s (zeros).}
\end{mpFunctionsExtract}

\begin{mpFunctionsExtract}
\mpWorksheetFunctionTwoNotImplemented
{BIN2OCT? mpNum? Converts a binary number to octal.}
{Number? mpNum? The binary number you want to convert. Number cannot contain more than 10 characters (10 bits). The most significant bit of number is the sign bit. The remaining 9 bits are magnitude bits. Negative numbers are represented using two's-complement notation.}
{Places? mpNum? The number of characters to use. If places is omitted, BIN2OCT uses the minimum number of characters necessary. Places is useful for padding the return value with leading 0s (zeros).}
\end{mpFunctionsExtract}

\begin{mpFunctionsExtract}
\mpWorksheetFunctionTwoNotImplemented
{DEC2BIN? mpNum? Converts a decimal number to binary.}
{Number? mpNum? The decimal integer you want to convert. If number is negative, valid place values are ignored and DEC2BIN returns a 10-character (10-bit) binary number in which the most significant bit is the sign bit. The remaining 9 bits are magnitude bits. Negative numbers are represented using two's-complement notation.}
{Places? mpNum? The number of characters to use. If places is omitted, DEC2BIN uses the minimum number of characters necessary. Places is useful for padding the return value with leading 0s (zeros).}
\end{mpFunctionsExtract}

\begin{mpFunctionsExtract}
\mpWorksheetFunctionTwoNotImplemented
{DEC2HEX? mpNum? Converts a decimal number to hexadecimal.}
{Number? mpNum? The decimal integer you want to convert. If number is negative, places is ignored and DEC2HEX returns a 10-character (40-bit) hexadecimal number in which the most significant bit is the sign bit. The remaining 39 bits are magnitude bits. Negative numbers are represented using two's-complement notation.}
{Places? mpNum? The number of characters to use. If places is omitted, DEC2HEX uses the minimum number of characters necessary. Places is useful for padding the return value with leading 0s (zeros).}
\end{mpFunctionsExtract}

\begin{mpFunctionsExtract}
\mpWorksheetFunctionTwoNotImplemented
{DEC2OCT? mpNum? Converts a decimal number to octal.}
{Number? mpNum? The decimal integer you want to convert. If number is negative, places is ignored and DEC2OCT returns a 10-character (30-bit) octal number in which the most significant bit is the sign bit. The remaining 29 bits are magnitude bits. Negative numbers are represented using two's-complement notation.}
{Places? mpNum? The number of characters to use. If places is omitted, DEC2OCT uses the minimum number of characters necessary. Places is useful for padding the return value with leading 0s (zeros).}
\end{mpFunctionsExtract}

\begin{mpFunctionsExtract}
\mpWorksheetFunctionTwoNotImplemented
{HEX2BIN? mpNum? Converts a  hexadecimal number to binary.}
{Number? mpNum? The hexadecimal number you want to convert. Number cannot contain more than 10 characters. The most significant bit of number is the sign bit (40th bit from the right). The remaining 9 bits are magnitude bits. Negative numbers are represented using two's-complement notation.}
{Places? mpNum? The number of characters to use. If places is omitted, HEX2BIN uses the minimum number of characters necessary. Places is useful for padding the return value with leading 0s (zeros).}
\end{mpFunctionsExtract}

\begin{mpFunctionsExtract}
\mpWorksheetFunctionOneNotImplemented
{HEX2DEC? mpNum? Converts a hexadecimal number to decimal.}
{Number? mpNum? The hexadecimal number you want to convert. Number cannot contain more than 10 characters (40 bits). The most significant bit of number is the sign bit. The remaining 39 bits are magnitude bits. Negative numbers are represented using two's-complement notation}
\end{mpFunctionsExtract}

\begin{mpFunctionsExtract}
\mpWorksheetFunctionTwoNotImplemented
{HEX2OCT? mpNum? Converts a hexadecimal number to octal.}
{Number? mpNum? The hexadecimal number you want to convert. Number cannot contain more than 10 characters. The most significant bit of number is the sign bit. The remaining 39 bits are magnitude bits. Negative numbers are represented using two's-complement notation.}
{Places? mpNum? The number of characters to use. If places is omitted, HEX2OCT uses the minimum number of characters necessary. Places is useful for padding the return value with leading 0s (zeros).}
\end{mpFunctionsExtract}

\begin{mpFunctionsExtract}
\mpWorksheetFunctionTwoNotImplemented
{OCT2BIN? mpNum? Converts a octal number to binary.}
{Number? mpNum? The octal number you want to convert. Number may not contain more than 10 characters. The most significant bit of number is the sign bit. The remaining 29 bits are magnitude bits. Negative numbers are represented using two's-complement notation.}
{Places? mpNum? The number of characters to use. If places is omitted, OCT2BIN uses the minimum number of characters necessary. Places is useful for padding the return value with leading 0s (zeros).}
\end{mpFunctionsExtract}

\begin{mpFunctionsExtract}
\mpWorksheetFunctionOneNotImplemented
{OCT2DEC? mpNum? Converts an octal number to decimal.}
{Number? mpNum? The octal number you want to convert. Number may not contain more than 10 octal characters (30 bits). The most significant bit of number is the sign bit. The remaining 29 bits are magnitude bits. Negative numbers are represented using two's-complement notation.}
\end{mpFunctionsExtract}

\begin{mpFunctionsExtract}
\mpWorksheetFunctionTwoNotImplemented
{OCT2HEX? mpNum? Converts an octal number to hexadecimal.}
{Number? mpNum? The octal number you want to convert. Number may not contain more than 10 octal characters (30 bits). The most significant bit of number is the sign bit. The remaining 29 bits are magnitude bits. Negative numbers are represented using two's-complement notation.}
{Places? mpNum? The number of characters to use. If places is omitted, OCT2HEX uses the minimum number of characters necessary. Places is useful for padding the return value with leading 0s (zeros).}
\end{mpFunctionsExtract}

\begin{mpFunctionsExtract}
\mpWorksheetFunctionThree
{BASE? mpNum? converts a number into a text representation with the given radix (base).}
{Number? mpNum? The number that you want to convert. Must be an integer greater than or equal to 0 and less than $2^53$.}
{Radix? mpNum? The base radix that you want to convert the number into. Must be an integer greater than or equal to 2 and less than or equal to 36.}
{MinLength? mpNum? The minimum length of the returned string. Must be an integer greater than or equal to 0.}
\end{mpFunctionsExtract}

\begin{mpFunctionsExtract}
\mpWorksheetFunctionTwoNotImplemented
{DECIMAL? mpNum? Converts a text representation of a number in a given base into a decimal number.}
{Text? String? The string length of Text must be less than or equal to 255 characters.}
{Radix? mpNum? Radix must be an integer greater than or equal to 2 (binary, or base 2) and less than or equal to 36 (base 36).}
\end{mpFunctionsExtract}

\section{Conversion and printing}

\section{Rounding}

\begin{mpFunctionsExtract}
\mpWorksheetFunctionTwoNotImplemented
{ROUND? mpNum? a number rounded to a specified number of digits}
{Number? mpNum? A real number you want to round.}
{Digits? mpNum? The number of digits to which you want to round. Negative rounds to the left of the decimal point; zero to the nearest integer.}
\end{mpFunctionsExtract}

\begin{mpFunctionsExtract}
\mpWorksheetFunctionTwoNotImplemented
{CEILING? mpNum? a number rounded up to the nearest multiple of significance}
{Number? mpNum? A real number you want to round.}
{Significance? mpNum? The multiple to which you want to round.}
\end{mpFunctionsExtract}

\begin{mpFunctionsExtract}
\mpWorksheetFunctionTwoNotImplemented
{CEILING.PRECISE? mpNum? a number rounded up to the nearest multiple of significance}
{Number? mpNum? A real number you want to round.}
{Significance? mpNum? The multiple to which you want to round.}
\end{mpFunctionsExtract}

\begin{mpFunctionsExtract}
\mpWorksheetFunctionTwoNotImplemented
{CEILING.MATH? mpNum? a number rounded up to the nearest multiple of significance}
{Number? mpNum? A real number you want to round.}
{Significance? mpNum? The multiple to which you want to round.}
\end{mpFunctionsExtract}

\begin{mpFunctionsExtract}
\mpFunctionOne
{ceil? mpNum?  a number down to the nearest integer.}
{x? mpNum? A real number.}
\end{mpFunctionsExtract}

\begin{mpFunctionsExtract}
\mpWorksheetFunctionTwoNotImplemented
{FLOOR? mpNum? a number rounded down to the nearest multiple of significance}
{Number? mpNum? A real number you want to round.}
{Significance? mpNum? The multiple to which you want to round. Number and Significance must either both be positive or both negative}
\end{mpFunctionsExtract}

\begin{mpFunctionsExtract}
\mpWorksheetFunctionTwoNotImplemented
{FLOOR.PRECISE? mpNum? a number rounded down to the nearest integer or to the nearest multiple of significance}
{Number? mpNum? A real number you want to round.}
{Significance? mpNum? The multiple to which you want to round.}
\end{mpFunctionsExtract}

\begin{mpFunctionsExtract}
\mpWorksheetFunctionTwoNotImplemented
{FLOOR.MATH? mpNum? a number rounded downto the nearest multiple of significance}
{Number? mpNum? A real number you want to round.}
{Significance? mpNum? The multiple to which you want to round.}
\end{mpFunctionsExtract}

\begin{mpFunctionsExtract}
\mpFunctionOne
{floor? mpNum?  a number down to the nearest integer.}
{x? mpNum? A real number.}
\end{mpFunctionsExtract}

\begin{mpFunctionsExtract}
\mpWorksheetFunctionTwoNotImplemented
{TRUNC? mpNum? a number truncated to an integer by removing the decimal, or fractional, part of a number}
{Number? mpNum? A real number you want to round.}
{Digits? mpNum? A number specifying the precision of te truncation, 0 if omitted.}
\end{mpFunctionsExtract}

\begin{mpFunctionsExtract}
\mpWorksheetFunctionOneNotImplemented
{EVEN? mpNum? the rounded value of $x$. Rounds a positive number up and a negative number down to the nearest even integer.}
{x? mpNum? A real number.}
\end{mpFunctionsExtract}

\begin{mpFunctionsExtract}
\mpWorksheetFunctionOneNotImplemented
{ODD? mpNum? the rounded value of $x$. Rounds a positive number up and a negative number down to the nearest odd integer.}
{x? mpNum? A real number.}
\end{mpFunctionsExtract}

\begin{mpFunctionsExtract}
\mpWorksheetFunctionOneNotImplemented
{INT? mpNum?  a number down to the nearest integer.}
{x? mpNum? A real number.}
\end{mpFunctionsExtract}

\begin{mpFunctionsExtract}
\mpFunctionOne
{nint? mpNum?  a number down to the nearest integer.}
{x? mpNum? A real number.}
\end{mpFunctionsExtract}

\begin{mpFunctionsExtract}
\mpFunctionOne
{frac? mpNum? the fractional part of $x$.}
{x? mpNum? A colpmex or real number.}
\end{mpFunctionsExtract}

\begin{mpFunctionsExtract}
\mpWorksheetFunctionTwoNotImplemented
{ROUNDDOWN? mpNum? a number rounded down, toward zero.}
{Number? mpNum? A real number you want to round.}
{Digits? mpNum? A number specifying the precision of te truncation, 0 if omitted.}
\end{mpFunctionsExtract}

\begin{mpFunctionsExtract}
\mpWorksheetFunctionTwoNotImplemented
{ROUNDUP? mpNum? a number rounded down, away from zero.}
{Number? mpNum? A real number you want to round.}
{Digits? mpNum? A number specifying the precision of te truncation, 0 if omitted.}
\end{mpFunctionsExtract}

\begin{mpFunctionsExtract}
\mpWorksheetFunctionTwoNotImplemented
{MROUND? mpNum? a number rounded to the desired multiple.}
{Number? mpNum? A real number you want to round.}
{Multiple? mpNum? The multiple to which you want to round.}
\end{mpFunctionsExtract}

\begin{mpFunctionsExtract}
\mpWorksheetFunctionTwoNotImplemented
{QUOTIENT? mpNum? the integer portion of a division.}
{x? mpNum? A real number}
{y? mpNum? A real number}
\end{mpFunctionsExtract}

\section{Components of Real and Complex Numbers}

\begin{mpFunctionsExtract}
\mpFunctionTwo
{ldexp? mpNum? $x \cdot 2^{y}$}
{x? mpNum? A real number.}
{y? mpNum? A real number.}
\end{mpFunctionsExtract}

\begin{mpFunctionsExtract}
\mpFunctionOne
{frexp? mpNumList? returns simultaneously significand and exponent of $x$}
{x? mpNum? A real number.}
\end{mpFunctionsExtract}

\begin{mpFunctionsExtract}
\mpWorksheetFunctionTwoNotImplemented
{COMPLEX? String? a complex number $z$ build from the real components $x$ and $y$, as string.}
{x? mpReal? A real number.}
{y? mpReal? A real number.}
\end{mpFunctionsExtract}

\begin{mpFunctionsExtract}
\mpFunctionTwo
{mpc? mpNum? a complex number $z$ build from the real components $x$ and $y$ as $z=x+iy$.}
{x? mpNum? A real number.}
{y? mpNum? A real number.}
\end{mpFunctionsExtract}

\begin{mpFunctionsExtract}
\mpFunctionOne
{polar? mpNum? Returns the polar representation of the complex number $z$.}
{z? mpNum? A complex or real number.}
\end{mpFunctionsExtract}

\begin{mpFunctionsExtract}
\mpFunctionTwo
{rect? mpNum? the complex number represented by polar coordinates $(r,\phi)$.}
{x? mpNum? A real number.}
{y? mpNum? A real number.}
\end{mpFunctionsExtract}

\begin{mpFunctionsExtract}
\mpWorksheetFunctionOneNotImplemented
{IMREAL? mpReal? the real component $x$ of $z=x+iy$.}
{z? String? A String representing a complex number.}
\end{mpFunctionsExtract}

\begin{mpFunctionsExtract}
\mpFunctionOne
{re? mpNum? the real part of $x$, $\Re(x)$.}
{z? mpNum? A complex number.}
\end{mpFunctionsExtract}

\begin{mpFunctionsExtract}
\mpWorksheetFunctionOneNotImplemented
{IMAGINARY? mpReal? the imaginary component $y$ of $z=x+iy$.}
{z? String? A String representing a complex number.}
\end{mpFunctionsExtract}

\begin{mpFunctionsExtract}
\mpFunctionOne
{im? mpNum? the imaginary part of $x$, $\Im(x)$.}
{z? mpNum? A complex number.}
\end{mpFunctionsExtract}

\begin{mpFunctionsExtract}
\mpWorksheetFunctionOneNotImplemented
{ABS? mpNum? the absolute value of $x$, $|x| = \sqrt{x^2}$.}
{x? mpNum? A real number.}
\end{mpFunctionsExtract}

\begin{mpFunctionsExtract}
\mpWorksheetFunctionOneNotImplemented
{IMABS? mpReal? the absolute value of $z=x+iy$}
{z? String? A String representing a complex number.}
\end{mpFunctionsExtract}

\begin{mpFunctionsExtract}
\mpFunctionOne
{abs? mpNum? the absolute value of $z=x+iy$}
{z? mpNum? A real or complex number.}
\end{mpFunctionsExtract}

\begin{mpFunctionsExtract}
\mpFunctionOne
{fabs? mpNum? the absolute value of $z=x+iy$}
{z? mpNum? A real or complex number.}
\end{mpFunctionsExtract}

\begin{mpFunctionsExtract}
\mpWorksheetFunctionOneNotImplemented
{IMARGUMENT? mpReal? the argument of $z=x+iy$}
{z? String? A String representing a complex number.}
\end{mpFunctionsExtract}

\begin{mpFunctionsExtract}
\mpFunctionOne
{arg? mpNum? the argument of $z=x+iy$}
{z? mpNum? A complex number.}
\end{mpFunctionsExtract}

\begin{mpFunctionsExtract}
\mpFunctionOne
{phase? mpNum? the argument of $z=x+iy$}
{z? mpNum? A complex number.}
\end{mpFunctionsExtract}

\begin{mpFunctionsExtract}
\mpWorksheetFunctionOneNotImplemented
{SIGN? mpNum? the value of the sign of $x, \text{sign}(x)$.}
{x? mpNum? A real number.}
\end{mpFunctionsExtract}

\begin{mpFunctionsExtract}
\mpFunctionOne
{sign? mpNum? the value of the sign of $x, \text{sign}(x)$.}
{x? mpNum? A real or complex number.}
\end{mpFunctionsExtract}

\begin{mpFunctionsExtract}
\mpWorksheetFunctionOneNotImplemented
{IMCONJUGATE? String? the conjugate of $z$, $\overline{z}=x-iy$}
{z? String? A String representing a complex number.}
\end{mpFunctionsExtract}

\begin{mpFunctionsExtract}
\mpFunctionOne
{conj? mpNum? the complex conjugate of $z$, $\overline{z}$}
{z? mpNum? A complex number.}
\end{mpFunctionsExtract}

\section{Arithmetic operations}

\begin{mpFunctionsExtract}
\mpFunctionThree
{fadd? mpNum? the sum of the numbers x and y, giving a floating-point result, optionally using a custom precision and rounding mode..}
{x? mpNum? A complex number.}
{y? mpNum? A complex number.}
{Keywords? String? prec, dps, exact, rounding.}
\end{mpFunctionsExtract}

\begin{mpFunctionsExtract}
\mpWorksheetFunctionOneNotImplemented
{IMSUM? String? the sum of up to 255 complex numbers.}
{z? String[]? An array of Strings representing an array of complex numbers.}
\end{mpFunctionsExtract}

\begin{mpFunctionsExtract}
\mpFunctionTwo
{fsum? mpNum? the sum of the numbers x and y, giving a floating-point result, optionally using a custom precision and rounding mode..}
{terms? mpNum? a finite number of terms.}
{Keywords? String? absolute=False, squared=False.}
\end{mpFunctionsExtract}

\begin{mpFunctionsExtract}
\mpWorksheetFunctionTwoNotImplemented
{SUMX2MY2? mpNum? the sum of the difference of squares of corresponding values in two arrays.}
{X? mpNum[]? A matrix of real numbers.}
{Y? mpNum[]? A matrix of real numbers.}
\end{mpFunctionsExtract}

\begin{mpFunctionsExtract}
\mpWorksheetFunctionTwoNotImplemented
{SUMX2PY2? mpNum? the sum of the sum of squares of corresponding values in two arrays.}
{X? mpNum[]? A matrix of real numbers.}
{Y? mpNum[]? A matrix of real numbers.}
\end{mpFunctionsExtract}

\begin{mpFunctionsExtract}
\mpWorksheetFunctionTwoNotImplemented
{SUMXMY2? mpNum? the sum of squares of differences of corresponding values in two arrays.}
{X? mpNum[]? A matrix of real numbers.}
{Y? mpNum[]? A matrix of real numbers.}
\end{mpFunctionsExtract}

\begin{mpFunctionsExtract}
\mpWorksheetFunctionOneNotImplemented
{SUMSQ? mpNum? the sum of the sum of the squares of up to 255 given arrays.}
{X? mpNumList? A list of up to 255 given arrays.}
\end{mpFunctionsExtract}

\begin{mpFunctionsExtract}
\mpWorksheetFunctionOneNotImplemented
{SUMPRODUCT? mpNum? the product of corresponding components in up to 255 given arrays.}
{X? mpNumList? A list of up to 255 given arrays.}
\end{mpFunctionsExtract}

\begin{mpFunctionsExtract}
\mpWorksheetFunctionFour
{SERIESSUM? mpNum? the sum of a (finite) power series.}
{x? mpNum? The input value to the power series.}
{n? Integer? The initial power to which you want to raise $x$..}
{m? Integer? The step by which to increase n for each term in the series..}
{a? mpNum[]? A set of $j$ coefficients by which each successive power of $x$ is multiplied.}
\end{mpFunctionsExtract}

\begin{mpFunctionsExtract}
\mpWorksheetFunctionTwoNotImplemented
{IMSUB? String? the difference of $z1$ and $z2$}
{z1? String? A Strings representing a complex number.}
{z2? String? A Strings representing a complex number.}
\end{mpFunctionsExtract}

\begin{mpFunctionsExtract}
\mpFunctionThree
{fsub? mpNum? the sum of the numbers x and y, giving a floating-point result, optionally using a custom precision and rounding mode..}
{x? mpNum? A complex number.}
{y? mpNum? A complex number.}
{Keywords? String? prec, dps, exact, rounding.}
\end{mpFunctionsExtract}

\begin{mpFunctionsExtract}
\mpFunctionTwo
{fneg? mpNum? the sum of the numbers x and y, giving a floating-point result, optionally using a custom precision and rounding mode..}
{x? mpNum? A complex number.}
{Keywords? String? prec, dps, exact, rounding.}
\end{mpFunctionsExtract}

\begin{mpFunctionsExtract}
\mpFunctionThree
{fmul? mpNum? the sum of the numbers x and y, giving a floating-point result, optionally using a custom precision and rounding mode..}
{x? mpNum? A complex number.}
{y? mpNum? A complex number.}
{Keywords? String? prec, dps, exact, rounding.}
\end{mpFunctionsExtract}

\begin{mpFunctionsExtract}
\mpWorksheetFunctionOneNotImplemented
{PRODUCT? mpNum? the product of all the numbers of up to 255 given arrays.}
{X? mpNumList? A list of up to 255 given arrays.}
\end{mpFunctionsExtract}

\begin{mpFunctionsExtract}
\mpWorksheetFunctionOneNotImplemented
{IMPRODUCT? String? the product of up to 255 complex numbers.}
{z? String[]? An array of Strings representing an array of complex numbers.}
\end{mpFunctionsExtract}

\begin{mpFunctionsExtract}
\mpFunctionTwo
{fprod? mpNum? a product containing a finite number of factors}
{factors? mpNum? a finite number of factors}
{Keywords? String? prec, dps, exact, rounding.}
\end{mpFunctionsExtract}

\begin{mpFunctionsExtract}
\mpFunctionTwo
{fdot? mpNum? a product containing a finite number of factors}
{factors? mpNum? a finite number of factors}
{Keywords? String? prec, dps, exact, rounding.}
\end{mpFunctionsExtract}

\begin{mpFunctionsExtract}
\mpWorksheetFunctionTwoNotImplemented
{BITLSHIFT? Integer? the product of $n$ and $2^k$.}
{n? Integer? An Integer.}
{k? Integer? An Integer.}
\end{mpFunctionsExtract}

\begin{mpFunctionsExtract}
\mpWorksheetFunctionTwoNotImplemented
{IMDIV? String? the quotient of $z1$ and $z2$}
{z1? String? A Strings representing a complex number.}
{z2? String? A Strings representing a complex number.}
\end{mpFunctionsExtract}

\begin{mpFunctionsExtract}
\mpFunctionThree
{fdiv? mpNum? the sum of the numbers x and y, giving a floating-point result, optionally using a custom precision and rounding mode..}
{x? mpNum? A complex number.}
{y? mpNum? A complex number.}
{Keywords? String? prec, dps, exact, rounding.}
\end{mpFunctionsExtract}

\begin{mpFunctionsExtract}
\mpWorksheetFunctionTwoNotImplemented
{BITRSHIFT? Integer? the quotient of $n$ and $2^k$.}
{n? Integer? An Integer.}
{k? Integer? An Integer.}
\end{mpFunctionsExtract}

\begin{mpFunctionsExtract}
\mpWorksheetFunctionTwoNotImplemented
{MOD? mpReal? the remainder of $x/y$}
{x? mpReal? A real number.}
{y? mpReal? A real number.}
\end{mpFunctionsExtract}

\begin{mpFunctionsExtract}
\mpFunctionTwo
{fmod? mpReal? the remainder of $x/y$}
{x? mpReal? A real number.}
{y? mpReal? A real number.}
\end{mpFunctionsExtract}

\section{Logical Operators }

\begin{mpFunctionsExtract}
\mpWorksheetFunctionTwoNotImplemented
{BITAND? Integer? $n_1$ bitwise-and $n_2$.}
{n1? Integer? An Integer.}
{n2? Integer? An Integer.}
\end{mpFunctionsExtract}

\begin{mpFunctionsExtract}
\mpWorksheetFunctionTwoNotImplemented
{BITOR? Integer? $n_1$ bitwise-inclusive-or $n_2$.}
{n1? Integer? An Integer.}
{n2? Integer? An Integer.}
\end{mpFunctionsExtract}

\begin{mpFunctionsExtract}
\mpWorksheetFunctionTwoNotImplemented
{BITXOR? Integer? $n_1$ bitwise-exclusive-or $n_2$.}
{n1? Integer? An Integer.}
{n2? Integer? An Integer.}
\end{mpFunctionsExtract}

\section{Comparison Operators and Sorting}

\begin{mpFunctionsExtract}
\mpFunctionTwo
{chop? mpNum? Chops off small real or imaginary parts, or converts numbers close to zero to exact zeros}
{x? mpNum? A real or complex number.}
{Keywords? String? tol=None}
\end{mpFunctionsExtract}

\begin{mpFunctionsExtract}
\mpFunctionThree
{almosteq? mpNum? Determine whether the difference between $s$ and $t$ is smaller than a given epsilon, either relatively or absolutely.}
{s? mpNum? A real or complex number.}
{t? mpNum? A real or complex number.}
{Keywords? String? rel\_eps=None, abs\_eps=None}
\end{mpFunctionsExtract}

\begin{mpFunctionsExtract}
\mpWorksheetFunctionTwoNotImplemented
{GESTEP? mpNum?  1 if number $\geq$ step; returns 0 (zero) otherwise. Use this function to filter a set of values. For example, by summing several GESTEP functions you calculate the count of values that exceed a threshold.}
{Number? mpNum? The value to test against step.}
{Step? mpNum? The threshold value. If you omit a value for step, GESTEP uses zero.}
\end{mpFunctionsExtract}

\begin{mpFunctionsExtract}
\mpWorksheetFunctionTwoNotImplemented
{DELTA? mpNum?  1 if Number1 = Number2; returns 0 otherwise.}
{Number1? mpNum? The first number to compare.}
{Number2? mpNum? The second number to compare.}
\end{mpFunctionsExtract}

\section{Properties of numbers}

\begin{mpFunctionsExtract}
\mpWorksheetFunctionOneNotImplemented
{ISNUMBER? Boolean? TRUE if $x$ is an ordinary number (i.e. neither NaN nor an infinity), and FALSE otherwise.}
{x? mpNum? A real number.}
\end{mpFunctionsExtract}

\begin{mpFunctionsExtract}
\mpFunctionOne
{isnormal? Boolean?  Determine whether x is 'normal' in the sense of floating-point representation; that is, return False if x is zero, an infinity or NaN; otherwise return True. By extension, a complex number x is considered 'normal' if its magnitude is normal}
{Number1? mpNum? A real or complex number.}
\end{mpFunctionsExtract}

\begin{mpFunctionsExtract}
\mpFunctionOne
{isfinite? Boolean?  Return True if x is a finite number, i.e. neither an infinity or a NaN}
{Number1? mpNum? A real or complex number.}
\end{mpFunctionsExtract}

\begin{mpFunctionsExtract}
\mpFunctionOne
{isinf? Boolean?  True if the absolute value of x is infinite; otherwise return False}
{Number1? mpNum? A real or complex number.}
\end{mpFunctionsExtract}

\begin{mpFunctionsExtract}
\mpFunctionOne
{isnan? Boolean?  Return True if x is a NaN (not-a-number), or for a complex number, whether either the real or complex part is NaN; otherwise return False}
{Number1? mpNum? A real or complex number.}
\end{mpFunctionsExtract}

\begin{mpFunctionsExtract}
\mpFunctionTwo
{isint? Boolean? Return True if x is integer-valued; otherwise return False.}
{x? mpNum? A real number.}
{Kexwords? String? gaussian=False.}
\end{mpFunctionsExtract}

\begin{mpFunctionsExtract}
\mpWorksheetFunctionOneNotImplemented
{ISEVEN? Boolean? TRUE if $n$ is an even integer, and FALSE otherwise.}
{x? mpNum? A real number.}
\end{mpFunctionsExtract}

\begin{mpFunctionsExtract}
\mpWorksheetFunctionOneNotImplemented
{ISODD? Boolean? TRUE if $n$ is an odd integer, and FALSE otherwise.}
{x? mpNum? A real number.}
\end{mpFunctionsExtract}

\begin{mpFunctionsExtract}
\mpFunctionOne
{mag? mpNum? Quick logarithmic magnitude estimate of a number.}
{x? mpNum? A real number.}
\end{mpFunctionsExtract}

\begin{mpFunctionsExtract}
\mpFunctionOne
{.nint\_distance? mpNum? Return $(n,d)$ where $n$ is the nearest integer to $x$ and $d$ is an estimate of $\log_2(|x-n|)$.}
{x? mpNum? A real number.}
\end{mpFunctionsExtract}

\section{Number generation}

\begin{mpFunctionsExtract}
\mpWorksheetFunctionZero
{RAND? mpNum? an evenly distributed random real number greater than or equal to 0 and less than 1.}
\end{mpFunctionsExtract}

\begin{mpFunctionsExtract}
\mpWorksheetFunctionTwoNotImplemented
{RANDBETWEEN? mpNum?  a random integer number between the numbers you specify.}
{Bottom? mpNum? The smallest integer RANDBETWEEN will return.}
{Top? mpNum? The largest integer RANDBETWEEN will return.}
\end{mpFunctionsExtract}

\begin{mpFunctionsExtract}
\mpFunctionZero
{rand? mpNum? Returns an mpf with value chosen randomly from $[0,1)$. The number of randomly generated bits in the mantissa is equal to the working precision.}
\end{mpFunctionsExtract}

\begin{mpFunctionsExtract}
\mpFunctionTwo
{fraction? mpNum?  Given Python integers $(p,q)$, returns a lazy mpf representing the fraction $p/q$. The value is updated with the precision.}
{p? mpNum? an integer.}
{q? mpNum? an integer.}
\end{mpFunctionsExtract}

\begin{mpFunctionsExtract}
\mpFunctionThree
{arange? mpNum?  This is a generalized version of Python's range() function that accepts fractional endpoints and step sizes and returns a list of mpf instance.}
{a? mpNum? a real number.}
{b? mpNum? a real number.}
{h? mpNum? a real number.}
\end{mpFunctionsExtract}

\begin{mpFunctionsExtract}
\mpFunctionFour
{linspace? mpNum?  This is a generalized version of Python's range() function that accepts fractional endpoints and step sizes and returns a list of mpf instance.}
{a? mpNum? a real number.}
{b? mpNum? a real number.}
{h? mpNum? a real number.}
{Keywords? String? endpoint=True.}
\end{mpFunctionsExtract}

\section{Matrices}

\begin{mpFunctionsExtract}
\mpWorksheetFunctionOneNotImplemented
{MUNIT? mpNum? the unity matrix of dimension $n$.}
{n? Integer? An integer specifying the dimension of the matrix.}
\end{mpFunctionsExtract}

\begin{mpFunctionsExtract}
\mpFunctionTwo
{matrix? mpNum?  This is a generalized version of Python's range() function that accepts fractional endpoints and step sizes and returns a list of mpf instance.}
{data? Object? an object specifying the matrix.}
{Keywords? String? random, random-symmetric, random-complex, random-hermitian, zeros, ones, eye, row-vector, col-vector, diagonal.}
\end{mpFunctionsExtract}

\begin{mpFunctionsExtract}
\mpFunctionThree
{MatrixAdd? mpNum? the sum of the numbers x and y, giving a floating-point result, optionally using a custom precision and rounding mode..}
{x? mpNum? A complex number.}
{y? mpNum? A complex number.}
{Keywords? String? prec, dps, exact, rounding.}
\end{mpFunctionsExtract}

\begin{mpFunctionsExtract}
\mpWorksheetFunctionTwoNotImplemented
{MMULT? mpNum[]? the matrix product of two arrays $X$ and $Y$. The result is an array with the same number of rows as $X$ and the same number of columns as $Y$.}
{X? mpNum[]? A matrix of real numbers.}
{Y? mpNum[]? A matrix of real numbers.}
\end{mpFunctionsExtract}

\begin{mpFunctionsExtract}
\mpFunctionThree
{MatrixMul? mpNum? the sum of the numbers x and y, giving a floating-point result, optionally using a custom precision and rounding mode..}
{x? mpNum? A complex number.}
{y? mpNum? A complex number.}
{Keywords? String? prec, dps, exact, rounding.}
\end{mpFunctionsExtract}

\chapter{Elementary Functions}

\section{Constants}

\begin{mpFunctionsExtract}
\mpWorksheetFunctionZero
{PI? mpNum? the value of $\pi = 3.1415926535897932...$.}
\end{mpFunctionsExtract}

\begin{mpFunctionsExtract}
\mpFunctionZero
{pi? mpNum?  pi: 3.14159...}
\end{mpFunctionsExtract}

\begin{mpFunctionsExtract}
\mpFunctionZero
{degree? mpNum?  degree = 1 deg = pi / 180: 0.0174533...}
\end{mpFunctionsExtract}

\begin{mpFunctionsExtract}
\mpFunctionZero
{e? mpNum?  the base of the natural logarithm, e = exp(1): 2.71828...}
\end{mpFunctionsExtract}

\begin{mpFunctionsExtract}
\mpFunctionZero
{phi? mpNum?  Golden ratio phi: 1.61803...}
\end{mpFunctionsExtract}

\begin{mpFunctionsExtract}
\mpFunctionZero
{euler? mpNum?  Euler's constant: 0.577216...}
\end{mpFunctionsExtract}

\begin{mpFunctionsExtract}
\mpFunctionZero
{catalan? mpNum?  Catalan's constant: 0.915966...}
\end{mpFunctionsExtract}

\begin{mpFunctionsExtract}
\mpFunctionZero
{apery? mpNum?  Apery's constant: 1.20206...}
\end{mpFunctionsExtract}

\begin{mpFunctionsExtract}
\mpFunctionZero
{khinchin? mpNum?  Khinchin's constant: 2.68545...}
\end{mpFunctionsExtract}

\begin{mpFunctionsExtract}
\mpFunctionZero
{glaisher? mpNum?  Glaisher's constant: 1.28243...}
\end{mpFunctionsExtract}

\begin{mpFunctionsExtract}
\mpFunctionZero
{mertens? mpNum?  Mertens' constant: 0.261497...}
\end{mpFunctionsExtract}

\begin{mpFunctionsExtract}
\mpFunctionZero
{twinprime? mpNum?  Twin prime constant: 0.660162...}
\end{mpFunctionsExtract}

\begin{mpFunctionsExtract}
\mpFunctionZero
{inf? mpNum? the value of the representation of  $+\infty$ in the current precision.}
\end{mpFunctionsExtract}

\begin{mpFunctionsExtract}
\mpFunctionZero
{nan? mpNum? the value of the representation of Not a Number (NaN) in the current precision.}
\end{mpFunctionsExtract}

\section{Exponential and Logarithmic Functions}

\begin{mpFunctionsExtract}
\mpWorksheetFunctionOneNotImplemented
{EXP? mpNum? the value of the exponential function, $\text{exp}(x) = e^x = \exp(x)$.}
{x? mpNum? A real number.}
\end{mpFunctionsExtract}

\begin{mpFunctionsExtract}
\mpWorksheetFunctionOneNotImplemented
{IMEXP? String? the complex exponential of $z$, as a String representing a complex number.}
{z? String? A String representing a complex number.}
\end{mpFunctionsExtract}

\begin{mpFunctionsExtract}
\mpFunctionOne
{exp? mpNum? the complex exponential of $z$}
{z? mpNum? A complex number.}
\end{mpFunctionsExtract}

\begin{mpFunctionsExtract}
\mpFunctionOne
{expj? mpNum?  $10^z$}
{z? mpNum? A complex number.}
\end{mpFunctionsExtract}

\begin{mpFunctionsExtract}
\mpFunctionOne
{expjpi? mpNum?  $10^z$}
{z? mpNum? A complex number.}
\end{mpFunctionsExtract}

\begin{mpFunctionsExtract}
\mpFunctionOne
{expm1? mpNum?  $10^z$}
{z? mpNum? A complex number.}
\end{mpFunctionsExtract}

\begin{mpFunctionsExtract}
\mpFunctionOne
{exp10? mpNum?  $10^z$}
{z? mpNum? A complex number.}
\end{mpFunctionsExtract}

\begin{mpFunctionsExtract}
\mpFunctionOne
{exp2? mpNum?  $2^z$}
{z? mpNum? A complex number.}
\end{mpFunctionsExtract}

\begin{mpFunctionsExtract}
\mpWorksheetFunctionOneNotImplemented
{LN? mpNum? the value of the natural logarithm $\text{ln}(x) = \log_e(x)$.}
{x? mpNum? A real number.}
\end{mpFunctionsExtract}

\begin{mpFunctionsExtract}
\mpWorksheetFunctionOneNotImplemented
{IMLN? String? the complex natural logarithm of $z$, as a String representing a complex number.}
{z? String? A String representing a complex number.}
\end{mpFunctionsExtract}

\begin{mpFunctionsExtract}
\mpWorksheetFunctionTwoNotImplemented
{LOG? mpNum? the value of the logarithm  to base $b$: $\text{logb}(x) = \log_{b}(x)$.}
{x? mpNum? A real number.}
{b? mpNum? A real number.}
\end{mpFunctionsExtract}

\begin{mpFunctionsExtract}
\mpFunctionTwo
{Logb? mpNum? the value of the logarithm  to base $b$: $\text{logb}(x) = \log_{b}(x)$.}
{x? mpNum? A real number.}
{b? mpNum? A real number.}
\end{mpFunctionsExtract}

\begin{mpFunctionsExtract}
\mpFunctionTwo
{log? mpNum? the complex natural logarithm of $z$}
{z? mpNum? A complex number.}
{base? mpNum? the base of the logarithm. A real number.}
\end{mpFunctionsExtract}

\begin{mpFunctionsExtract}
\mpFunctionOne
{ln? mpNum? the complex natural logarithm of $z$}
{z? mpNum? A complex number.}
\end{mpFunctionsExtract}

\begin{mpFunctionsExtract}
\mpWorksheetFunctionOneNotImplemented
{LOG10? mpNum? the value of the decadic logarithm $\text{log10}(x) = \log_{10}(x)$.}
{x? mpNum? A real number.}
\end{mpFunctionsExtract}

\begin{mpFunctionsExtract}
\mpWorksheetFunctionOneNotImplemented
{IMLOG10? String? $\log_{10}(z)$, as a String representing a complex number.}
{z? String? A String representing a complex number.}
\end{mpFunctionsExtract}

\begin{mpFunctionsExtract}
\mpFunctionOne
{log10? mpNum? $\log_{10}(z)$}
{z? mpNum? A complex number.}
\end{mpFunctionsExtract}

\begin{mpFunctionsExtract}
\mpWorksheetFunctionOneNotImplemented
{IMLOG2? String? $\log_{2}(z)$, as a String representing a complex number.}
{z? String? A String representing a complex number.}
\end{mpFunctionsExtract}

\begin{mpFunctionsExtract}
\mpFunctionOne
{log2? mpNum? $\log_{2}(z)$}
{z? mpNum? A complex number.}
\end{mpFunctionsExtract}

\begin{mpFunctionsExtract}
\mpFunctionOne
{lnp1? mpNum? the value of the function $\ln(1+x)$.}
{x? mpNum? A real number.}
\end{mpFunctionsExtract}

\section{Roots and Power Functions}

\begin{mpFunctionsExtract}
\mpFunctionOne
{square? mpNum? the square of $z$.}
{z? mpNum? A complex number.}
\end{mpFunctionsExtract}

\begin{mpFunctionsExtract}
\mpWorksheetFunctionTwoNotImplemented
{POWER? mpNum? the value of $x^y, y \in  \mathbb{R}$.}
{x? mpNum? A real number.}
{y? mpNum? A real number.}
\end{mpFunctionsExtract}

\begin{mpFunctionsExtract}
\mpWorksheetFunctionTwoNotImplemented
{IMPOWER? String? an integer power of $z$, as a String representing a complex number.}
{z? String? A String representing a complex number.}
{k? Integer? An integer.}
\end{mpFunctionsExtract}

\begin{mpFunctionsExtract}
\mpFunctionTwo
{power? mpNum? an complex power of $z$}
{z1? mpNum? A complex number.}
{z2? mpNum? A complex number.}
\end{mpFunctionsExtract}

\begin{mpFunctionsExtract}
\mpFunctionTwo
{powm1? mpNum? an integer power of $z$}
{z? mpNum? A complex number.}
{k? mpNum?  A complex number.}
\end{mpFunctionsExtract}

\begin{mpFunctionsExtract}
\mpWorksheetFunctionOneNotImplemented
{SQRT? mpNum? the absolute value of the square root of $x, \sqrt{x}$.}
{x? mpNum? A real number.}
\end{mpFunctionsExtract}

\begin{mpFunctionsExtract}
\mpWorksheetFunctionOneNotImplemented
{IMSQRT? String? the square root of $z$, as a String representing a complex number.}
{z? String? A String representing a complex number.}
\end{mpFunctionsExtract}

\begin{mpFunctionsExtract}
\mpFunctionOne
{sqrt? mpNum? the square root of $z$}
{z? mpNum? A complex number.}
\end{mpFunctionsExtract}

\begin{mpFunctionsExtract}
\mpFunctionTwo
{hypot? mpNum? the value of $\sqrt{x^2+y^2}$.}
{x? mpNum? A real number.}
{y? mpNum? A real number.}
\end{mpFunctionsExtract}

\begin{mpFunctionsExtract}
\mpFunctionOne
{cbrt? mpNum? the square root of $z$}
{z? mpNum? A complex number.}
\end{mpFunctionsExtract}

\begin{mpFunctionsExtract}
\mpFunctionTwo
{root? mpNum? the value of the $n^{th}$ root of $x$, $\sqrt[n]{x}, n=2,3,...$.}
{z? mpNum? A complex number.}
{n? mpNum? An integer.}
\end{mpFunctionsExtract}

\begin{mpFunctionsExtract}
\mpFunctionTwo
{nthroot? mpNum? the value of the $n^{th}$ root of $x$, $\sqrt[n]{x}, n=2,3,...$.}
{n? mpNum? An integer.}
{y? mpNum? A real number.}
\end{mpFunctionsExtract}

\section{Trigonometric Functions}

\begin{mpFunctionsExtract}
\mpWorksheetFunctionOneNotImplemented
{DEGREES? mpNum? the value of $x$ converted to degrees, with the input $x$ in radians.}
{x? mpNum? A real number.}
\end{mpFunctionsExtract}

\begin{mpFunctionsExtract}
\mpFunctionOne
{degrees? mpNum? the value of $x$ converted to degrees, with the input $x$ in radians.}
{x? mpNum? A real number.}
\end{mpFunctionsExtract}

\begin{mpFunctionsExtract}
\mpWorksheetFunctionOneNotImplemented
{RADIANS? mpNum? the value of $x$ converted to radians, with the input $x$ in degrees.}
{x? mpNum? A real number.}
\end{mpFunctionsExtract}

\begin{mpFunctionsExtract}
\mpFunctionOne
{radians? mpNum? the value of $x$ converted to radians, with the input $x$ in degrees.}
{x? mpNum? A real number.}
\end{mpFunctionsExtract}

\begin{mpFunctionsExtract}
\mpWorksheetFunctionOneNotImplemented
{SQRTPI? mpNum? the value of , $\sqrt{n \cdot \pi}$.}
{x? mpNum? A real number.}
\end{mpFunctionsExtract}

\begin{mpFunctionsExtract}
\mpWorksheetFunctionOneNotImplemented
{SIN? mpNum? the value of the sine of $x$, with $x$ in radians.}
{x? mpNum? A real number.}
\end{mpFunctionsExtract}

\begin{mpFunctionsExtract}
\mpWorksheetFunctionOneNotImplemented
{IMSIN? String? complex sine of $z$, as a String representing a complex number.}
{z? String? A String representing a complex number.}
\end{mpFunctionsExtract}

\begin{mpFunctionsExtract}
\mpFunctionOne
{sin? mpNum? complex sine of $z$}
{z? mpNum? A complex number.}
\end{mpFunctionsExtract}

\begin{mpFunctionsExtract}
\mpWorksheetFunctionOneNotImplemented
{COS? mpNum? the value of the cosine of $x$, with $x$ in radians.}
{x? mpNum? A real number.}
\end{mpFunctionsExtract}

\begin{mpFunctionsExtract}
\mpWorksheetFunctionOneNotImplemented
{IMCOS? String? complex cosine of $z$, as a String representing a complex number.}
{z? String? A String representing a complex number.}
\end{mpFunctionsExtract}

\begin{mpFunctionsExtract}
\mpFunctionOne
{cos? mpNum? complex cosine of $z$}
{z? mpNum? A complex number.}
\end{mpFunctionsExtract}

\begin{mpFunctionsExtract}
\mpWorksheetFunctionOneNotImplemented
{TAN? mpNum? the value of the tangent of $x$, with $x$ in radians.}
{x? mpNum? A real number.}
\end{mpFunctionsExtract}

\begin{mpFunctionsExtract}
\mpWorksheetFunctionOneNotImplemented
{IMTAN? String? complex tangent of $z$, as a String representing a complex number.}
{z? String? A String representing a complex number.}
\end{mpFunctionsExtract}

\begin{mpFunctionsExtract}
\mpFunctionOne
{tan? mpNum? complex tangent of $z$}
{z? mpNum? A complex number.}
\end{mpFunctionsExtract}

\begin{mpFunctionsExtract}
\mpWorksheetFunctionOneNotImplemented
{SEC? mpNum? the value of the secant of $x$, with $x$ in radians.}
{x? mpNum? A real number.}
\end{mpFunctionsExtract}

\begin{mpFunctionsExtract}
\mpWorksheetFunctionOneNotImplemented
{IMSEC? String? the complex secant of $z$, as a String representing a complex number.}
{z? String? A String representing a complex number.}
\end{mpFunctionsExtract}

\begin{mpFunctionsExtract}
\mpFunctionOne
{sec? mpNum? the complex secant of $z$}
{z? mpNum? A complex number.}
\end{mpFunctionsExtract}

\begin{mpFunctionsExtract}
\mpWorksheetFunctionOneNotImplemented
{CSC? mpNum? the value of the cosecant of $x$, with $x$ in radians.}
{x? mpNum? A real number.}
\end{mpFunctionsExtract}

\begin{mpFunctionsExtract}
\mpWorksheetFunctionOneNotImplemented
{IMCSC? String? the complex cosecant of $z$, as a String representing a complex number.}
{z? String? A String representing a complex number.}
\end{mpFunctionsExtract}

\begin{mpFunctionsExtract}
\mpFunctionOne
{csc? mpNum? the complex cosecant of $z$}
{z? mpNum? A complex number.}
\end{mpFunctionsExtract}

\begin{mpFunctionsExtract}
\mpWorksheetFunctionOneNotImplemented
{COT? mpNum? the value of the cotangent of $x$, with $x$ in radians.}
{x? mpNum? A real number.}
\end{mpFunctionsExtract}

\begin{mpFunctionsExtract}
\mpWorksheetFunctionOneNotImplemented
{IMCOT? String? the complex cotangent of $z$, as a String representing a complex number.}
{z? String? A String representing a complex number.}
\end{mpFunctionsExtract}

\begin{mpFunctionsExtract}
\mpFunctionOne
{cot? mpNum? the complex cotangent of $z$}
{z? mpNum? A complex number.}
\end{mpFunctionsExtract}

\section{Hyperbolic Functions}

\begin{mpFunctionsExtract}
\mpWorksheetFunctionOneNotImplemented
{SINH? mpNum? the value of the hyperbolic sine of $x$, with $x$ in radians.}
{x? mpNum? A real number.}
\end{mpFunctionsExtract}

\begin{mpFunctionsExtract}
\mpWorksheetFunctionOneNotImplemented
{IMSINH? String? the complex hyperbolic sine of $z$, as a String representing a complex number.}
{z? String? A String representing a complex number.}
\end{mpFunctionsExtract}

\begin{mpFunctionsExtract}
\mpFunctionOne
{sinh? mpNum? the complex hyperbolic sine of $z$}
{z? mpNum? A complex number.}
\end{mpFunctionsExtract}

\begin{mpFunctionsExtract}
\mpWorksheetFunctionOneNotImplemented
{COSH? mpNum? the value of the hyperbolic cosine of $x$, with $x$ in radians.}
{x? mpNum? A real number.}
\end{mpFunctionsExtract}

\begin{mpFunctionsExtract}
\mpWorksheetFunctionOneNotImplemented
{IMCOSH? String? the complex hyperbolic cosine of $z$, as a String representing a complex number.}
{z? String? A String representing a complex number.}
\end{mpFunctionsExtract}

\begin{mpFunctionsExtract}
\mpFunctionOne
{cosh? mpNum? the complex hyperbolic cosine of $z$}
{z? mpNum? A complex number.}
\end{mpFunctionsExtract}

\begin{mpFunctionsExtract}
\mpWorksheetFunctionOneNotImplemented
{TANH? mpNum? the value of the hyperbolic cosine of $x$, with $x$ in radians.}
{x? mpNum? A real number.}
\end{mpFunctionsExtract}

\begin{mpFunctionsExtract}
\mpWorksheetFunctionOneNotImplemented
{IMTANH? String? the complex hyperbolic tangent of $z$, as a String representing a complex number.}
{z? String? A String representing a complex number.}
\end{mpFunctionsExtract}

\begin{mpFunctionsExtract}
\mpFunctionOne
{tanh? mpNum? the complex hyperbolic tangent of $z$}
{z? mpNum? A complex number.}
\end{mpFunctionsExtract}

\begin{mpFunctionsExtract}
\mpWorksheetFunctionOneNotImplemented
{SECH? mpNum? the value of the hyperbolic cosecant of $x$, with $x$ in radians.}
{x? mpNum? A real number.}
\end{mpFunctionsExtract}

\begin{mpFunctionsExtract}
\mpWorksheetFunctionOneNotImplemented
{IMSECH? String? the complex hyperbolic secant of $z$, as a String representing a complex number.}
{z? String? A String representing a complex number.}
\end{mpFunctionsExtract}

\begin{mpFunctionsExtract}
\mpFunctionOne
{sech? mpNum? the complex hyperbolic secant of $z$}
{z? mpNum? A complex number.}
\end{mpFunctionsExtract}

\begin{mpFunctionsExtract}
\mpWorksheetFunctionOneNotImplemented
{CSCH? mpNum? the value of the hyperbolic cosecant of $x$, with $x$ in radians.}
{x? mpNum? A real number.}
\end{mpFunctionsExtract}

\begin{mpFunctionsExtract}
\mpWorksheetFunctionOneNotImplemented
{IMCSCH? String? the complex hyperbolic cosecant of $z$, as a String representing a complex number.}
{z? String? A String representing a complex number.}
\end{mpFunctionsExtract}

\begin{mpFunctionsExtract}
\mpFunctionOne
{csch? mpNum? the complex hyperbolic cosecant of $z$}
{z? mpNum? A complex number.}
\end{mpFunctionsExtract}

\begin{mpFunctionsExtract}
\mpWorksheetFunctionOneNotImplemented
{COTH? mpNum? the value of the hyperbolic cotangent of $x$, with $x$ in radians.}
{x? mpNum? A real number.}
\end{mpFunctionsExtract}

\begin{mpFunctionsExtract}
\mpWorksheetFunctionOneNotImplemented
{IMCOTH? String? the complex hyperbolic cotangent of $z$, as a String representing a complex number.}
{z? String? A String representing a complex number.}
\end{mpFunctionsExtract}

\begin{mpFunctionsExtract}
\mpFunctionOne
{coth? mpNum? the complex hyperbolic cotangent of $z$}
{z? mpNum? A complex number.}
\end{mpFunctionsExtract}

\section{Inverse Trigonometric Functions}

\begin{mpFunctionsExtract}
\mpWorksheetFunctionOneNotImplemented
{ASIN? mpNum? the value of the arc-sine of $x$ in radians.}
{x? mpNum? A real number.}
\end{mpFunctionsExtract}

\begin{mpFunctionsExtract}
\mpFunctionOne
{asin? mpNum? the inverse complex sine of $z$}
{z? mpNum? A complex number.}
\end{mpFunctionsExtract}

\begin{mpFunctionsExtract}
\mpWorksheetFunctionOneNotImplemented
{ACOS? mpNum? the value of the arc-cosine of $x$ in radians.}
{x? mpNum? A real number.}
\end{mpFunctionsExtract}

\begin{mpFunctionsExtract}
\mpFunctionOne
{acos? mpNum? the inverse complex cosine of $z$}
{z? mpNum? A complex number.}
\end{mpFunctionsExtract}

\begin{mpFunctionsExtract}
\mpWorksheetFunctionOneNotImplemented
{ATAN? mpNum? the value of the arc-tangent of $x$ in radians.}
{x? mpNum? A real number.}
\end{mpFunctionsExtract}

\begin{mpFunctionsExtract}
\mpFunctionOne
{atan? mpNum? the inverse complex tangent of $z$}
{z? mpNum? A complex number.}
\end{mpFunctionsExtract}

\begin{mpFunctionsExtract}
\mpWorksheetFunctionTwoNotImplemented
{ATAN2? mpNum? the value of the arc-tangent of $x$ in radians.}
{x? mpNum? A real number.}
{y? mpNum? A real number.}
\end{mpFunctionsExtract}

\begin{mpFunctionsExtract}
\mpFunctionTwo
{Atan2? mpNum? the value of the arc-tangent of $x$ in radians.}
{x? mpNum? A real number.}
{y? mpNum? A real number.}
\end{mpFunctionsExtract}

\begin{mpFunctionsExtract}
\mpWorksheetFunctionOneNotImplemented
{ACOT? mpNum? the value of the arc-cotangent of $x$ in radians.}
{x? mpNum? A real number.}
\end{mpFunctionsExtract}

\begin{mpFunctionsExtract}
\mpFunctionOne
{acot? mpNum? the inverse complex cotangent of $z$}
{z? mpNum? A complex number.}
\end{mpFunctionsExtract}

\section{Inverse Hyperbolic Functions}

\begin{mpFunctionsExtract}
\mpWorksheetFunctionOneNotImplemented
{ASINH? mpNum? the value of the hyperbolic arc-sine  of $x$ in radians.}
{x? mpNum? A real number.}
\end{mpFunctionsExtract}

\begin{mpFunctionsExtract}
\mpFunctionOne
{asinh? mpNum? the inverse complex hyperbolic sine of $z$}
{z? mpNum? A complex number.}
\end{mpFunctionsExtract}

\begin{mpFunctionsExtract}
\mpWorksheetFunctionOneNotImplemented
{ACOSH? mpNum? the value of the hyperbolic arc-cosine  of $x$ in radians.}
{x? mpNum? A real number.}
\end{mpFunctionsExtract}

\begin{mpFunctionsExtract}
\mpFunctionOne
{acosh? mpNum? the inverse complex hyperbolic cosine of $z$}
{z? mpNum? A complex number.}
\end{mpFunctionsExtract}

\begin{mpFunctionsExtract}
\mpWorksheetFunctionOneNotImplemented
{ATANH? mpNum? the value of the hyperbolic arc-tangent  of $x$ in radians.}
{x? mpNum? A real number.}
\end{mpFunctionsExtract}

\begin{mpFunctionsExtract}
\mpFunctionOne
{atanh? mpNum? the inverse complex hyperbolic tangent of $z$}
{z? mpNum? A complex number.}
\end{mpFunctionsExtract}

\begin{mpFunctionsExtract}
\mpWorksheetFunctionOneNotImplemented
{ACOTH? mpNum? the value of the hyperbolic arc-cotangent  of $x$ in radians.}
{x? mpNum? A real number.}
\end{mpFunctionsExtract}

\begin{mpFunctionsExtract}
\mpFunctionOne
{acoth? mpNum? the inverse complex hyperbolic cotangent of $z$}
{z? mpNum? A complex number.}
\end{mpFunctionsExtract}

\section{Elementary Functions of Mathematical Physics}

\begin{mpFunctionsExtract}
\mpWorksheetFunctionTwoNotImplemented
{BESSELJ? mpNum? $J_{\nu}(z)$, the Bessel function of the first kind of real order $\nu$.}
{x? mpNum? A real number.}
{$\nu$? mpNum? A real number.}
\end{mpFunctionsExtract}

\begin{mpFunctionsExtract}
\mpWorksheetFunctionTwoNotImplemented
{BESSELY? mpNum? $Y_{\nu}(z)$, the Bessel function of the second kind of order $\nu$.}
{x? mpNum? A real number.}
{$\nu$? mpNum? A real number.}
\end{mpFunctionsExtract}

\begin{mpFunctionsExtract}
\mpWorksheetFunctionTwoNotImplemented
{BESSELI? mpNum? $J_{\nu}(z)$, the Bessel function of the first kind of real order $\nu$.}
{x? mpNum? A real number.}
{$\nu$? mpNum? A real number.}
\end{mpFunctionsExtract}

\begin{mpFunctionsExtract}
\mpWorksheetFunctionTwoNotImplemented
{BESSELK? mpNum?  $K_{\nu}(x)$, the modified Bessel function of the second kind of order $\nu$.}
{x? mpNum? A real number.}
{$\nu$? mpNum? A real number.}
\end{mpFunctionsExtract}

\begin{mpFunctionsExtract}
\mpWorksheetFunctionOneNotImplemented
{ERF? mpNum? the value of the error function.}
{x? mpNum? A real number.}
\end{mpFunctionsExtract}

\begin{mpFunctionsExtract}
\mpWorksheetFunctionOneNotImplemented
{ERF.PRECISE? mpNum? the value of the error function.}
{x? mpNum? A real number.}
\end{mpFunctionsExtract}

\begin{mpFunctionsExtract}
\mpWorksheetFunctionOneNotImplemented
{ERFC? mpNum? the value of the complementary error function.}
{x? mpNum? A real number.}
\end{mpFunctionsExtract}

\begin{mpFunctionsExtract}
\mpWorksheetFunctionOneNotImplemented
{ERFC.PRECISE? mpNum? the value of the complementary error function.}
{x? mpNum? A real number.}
\end{mpFunctionsExtract}

\begin{mpFunctionsExtract}
\mpWorksheetFunctionOneNotImplemented
{GAMMA? mpNum? the gamma function for $x \neq 0, -1, -2,\ldots$.}
{x? mpNum? A real number.}
\end{mpFunctionsExtract}

\begin{mpFunctionsExtract}
\mpWorksheetFunctionOneNotImplemented
{GAMMALN? mpNum? the logarithm of the gamma function.}
{x? mpNum? A real number.}
\end{mpFunctionsExtract}

\begin{mpFunctionsExtract}
\mpWorksheetFunctionOneNotImplemented
{GAMMALN.PRECISE? mpNum? the logarithm of the gamma function.}
{x? mpNum? A real number.}
\end{mpFunctionsExtract}

\begin{mpFunctionsExtract}
\mpFunctionTwoNotImplemented
{Beta? mpNum? the Beta function.}
{a? mpNum? A real number.}
{b? mpNum? A real number.}
\end{mpFunctionsExtract}

\section{Factorial and Related Functions}

\begin{mpFunctionsExtract}
\mpWorksheetFunctionOneNotImplemented
{FACT? Integer?  $n!$, the factorial of $n$}
{n? Integer? An Integer.}
\end{mpFunctionsExtract}

\begin{mpFunctionsExtract}
\mpWorksheetFunctionOneNotImplemented
{FACTDOUBLE? Integer?  $n!!$, the double factorial of $n$}
{n? Integer? An Integer.}
\end{mpFunctionsExtract}

\begin{mpFunctionsExtract}
\mpWorksheetFunctionTwoNotImplemented
{COMBIN? Integer? the binomial coefficient}
{n? Integer? An Integer.}
{k? Integer? An Integer.}
\end{mpFunctionsExtract}

\begin{mpFunctionsExtract}
\mpWorksheetFunctionTwoNotImplemented
{COMBINA? Integer? the binomial coefficient}
{n? Integer? An Integer.}
{k? Integer? An Integer.}
\end{mpFunctionsExtract}

\begin{mpFunctionsExtract}
\mpWorksheetFunctionOneNotImplemented
{MULTINOMIAL? mpReal? the multinomial}
{a[]? mpReal? An array of integers.}
\end{mpFunctionsExtract}

\begin{mpFunctionsExtract}
\mpWorksheetFunctionTwoNotImplemented
{PERMUT? Integer? the number of permutations for a given number $k$ of objects that can be selected from $n$ objects.}
{n? Integer? An Integer.}
{k? Integer? An Integer.}
\end{mpFunctionsExtract}

\begin{mpFunctionsExtract}
\mpWorksheetFunctionTwoNotImplemented
{PERMUTATIONA? Integer? the number of permutations for a given number $k$ of objects that can be selected from $n$ objects.}
{n? Integer? An Integer.}
{k? Integer? An Integer.}
\end{mpFunctionsExtract}

\begin{mpFunctionsExtract}
\mpWorksheetFunctionTwoNotImplemented
{GCD? Integer? the greatest common divisor of $n_1$ and $n_2$}
{n1? Integer? An Integer.}
{n2? Integer? An Integer.}
\end{mpFunctionsExtract}

\begin{mpFunctionsExtract}
\mpWorksheetFunctionTwoNotImplemented
{LCM? Integer? the least common multiple of $n_1$ and $n_2$.}
{n1? Integer? An Integer.}
{n2? Integer? An Integer.}
\end{mpFunctionsExtract}

\chapter{Linear Algebra}

\section{Multiple Linear Regression}

\begin{mpFunctionsExtract}
\mpWorksheetFunctionOneNotImplemented
{MDETERM? mpNum? the matrix determinant of a numeric array $X$ with an equal number of rows and columns.}
{X? mpNum[]? A matrix of real numbers.}
\end{mpFunctionsExtract}

\begin{mpFunctionsExtract}
\mpWorksheetFunctionOneNotImplemented
{MINVERSE? mpNum? the  inverse matrix for the matrix stored in the numeric array  $X$ with an equal number of rows and columns.}
{X? mpNum[]? A matrix of real numbers.}
\end{mpFunctionsExtract}

\begin{mpFunctionsExtract}
\mpWorksheetFunctionFourNotImplemented
{LINEST? mpNumList? information obtained by performing multiple liner regression.}
{Y? mpNum[]? An array of real numbers.}
{X? mpNum[]? An array of real numbers.}
{Const? Boolean? A logical value.}
{Stats? Boolean? A logical value.}
\end{mpFunctionsExtract}

\begin{mpFunctionsExtract}
\mpWorksheetFunctionFourNotImplemented
{TREND? mpNumList? values along a linear trend.}
{Y? mpNum[]? An array of real numbers.}
{X? mpNum[]? An array of real numbers.}
{NewX? mpNum[]? An array of real numbers.}
{Const? Boolean? A logical value.}
\end{mpFunctionsExtract}

\section{Exponential Growth Curves}

\begin{mpFunctionsExtract}
\mpWorksheetFunctionFourNotImplemented
{LOGEST? mpNumList? an exponential curve that fits your data and returns an array of values that describes the curve.}
{Y? mpNum[]? An array of real numbers.}
{X? mpNum[]? An array of real numbers.}
{Const? Boolean? A logical value.}
{Stats? Boolean? A logical value.}
\end{mpFunctionsExtract}

\begin{mpFunctionsExtract}
\mpWorksheetFunctionFourNotImplemented
{GROWTH? mpNumList? predicted exponential growth by using existing data.}
{Y? mpNum[]? An array of real numbers.}
{X? mpNum[]? An array of real numbers.}
{NewX? mpNum[]? An array of real numbers.}
{Const? Boolean? A logical value.}
\end{mpFunctionsExtract}

\section{Norms}

\begin{mpFunctionsExtract}
\mpFunctionTwo
{norm? mpNumList? the entrywise $p$-norm of an iterable x, i.e. the vector norm.}
{Y? mpNum[]? An array of real numbers.}
{Keywords? String?  p=2.}
\end{mpFunctionsExtract}

\begin{mpFunctionsExtract}
\mpFunctionTwo
{mnorm? mpNumList? the matrix (operator) $p$-norm of A. Currently p=1 and p=inf are supported.}
{A? mpNum[]? An array of real numbers.}
{Keywords? String?  p=1.}
\end{mpFunctionsExtract}

\section{Decompositions}

\begin{mpFunctionsExtract}
\mpFunctionTwo
{cholesky? mpNum? the Cholesky decomposition of a symmetric positive-definite matrix $A$.}
{A? mpNum[]? A symmetric matrix.}
{Keywords? String?  tol=None.}
\end{mpFunctionsExtract}

\section{Linear Equations}

\begin{mpFunctionsExtract}
\mpFunctionThree
{lu\_solve? mpNum? solves a linear equation system using a LU decomposition.}
{A? mpNum[]? A symmetric matrix.}
{b? mpNum[]? A symmetric matrix.}
{Keywords? String?  tol=None.}
\end{mpFunctionsExtract}

\begin{mpFunctionsExtract}
\mpFunctionFour
{residual? mpNum? the residual  $||Ax-b||$.}
{A? mpNum[]? A square matrix.}
{b? mpNum[]? A vector.}
{x? mpNum[]? A vector.}
{Keywords? String?  tol=None.}
\end{mpFunctionsExtract}

\section{Matrix Factorization}

\begin{mpFunctionsExtract}
\mpFunctionTwo
{lu? mpNum? an explicit LU factorization of a matrix, returning P, L, U}
{A? mpNum[]? A square matrix.}
{Keywords? String?  tol=None.}
\end{mpFunctionsExtract}

\begin{mpFunctionsExtract}
\mpFunctionTwo
{qr? mpNum? an explicit QR factorization of a matrix, returning Q, R}
{A? mpNum[]? A square matrix.}
{Keywords? String?  tol=None.}
\end{mpFunctionsExtract}

\chapter{Distribution Functions}

\section{Introduction to Distribution Functions}

\section{Beta-Distribution}

\begin{mpFunctionsExtract}
\mpFunctionFourNotImplemented
{BetaDist? mpNumList? pdf, CDF and related information for the central Beta-distribution}
{x? mpNum? A real number}
{a? mpNum? A real number greater 0, representing the numerator  degrees of freedom}
{b? mpNum? A real number greater 0, representing the denominator degrees of freedom}
{Output? String? A string describing the output choices}
\end{mpFunctionsExtract}

\begin{mpFunctionsExtract}
\mpWorksheetFunctionThreeNotImplemented
{BETADIST? mpReal? the CDF and of the central Beta-distribution}
{x? mpReal? A real number. The numeric value at which to evaluate the distribution}
{a? mpNum? A real number greater 0, representing the numerator  degrees of freedom}
{b? mpNum? A real number greater 0, representing the denominator degrees of freedom}
\end{mpFunctionsExtract}

\begin{mpFunctionsExtract}
\mpWorksheetFunctionFourNotImplemented
{BETA.DIST? mpReal? the CDF and of the central Beta-distribution}
{x? mpReal? A real number. The numeric value at which to evaluate the distribution}
{a? mpNum? A real number greater 0, representing the numerator  degrees of freedom}
{b? mpNum? A real number greater 0, representing the denominator degrees of freedom}
{Cumulative ? Boolean? A logical value that determines the form of the function. If cumulative is TRUE, T.DIST returns the cumulative distribution function; if FALSE, it returns the probability density function}
\end{mpFunctionsExtract}

\begin{mpFunctionsExtract}
\mpFunctionFourNotImplemented
{BetaDistInv? mpNumList? quantiles and related information for the the central Beta-distribution}
{Prob? mpNum? A real number between 0 and 1.}
{m? mpNum? A real number greater 0, representing the numerator  degrees of freedom}
{n? mpNum? A real number greater 0, representing the denominator degrees of freedom}
{Output? String? A string describing the output choices}
\end{mpFunctionsExtract}

\begin{mpFunctionsExtract}
\mpWorksheetFunctionThree
{BETAINV? mpReal? the two-tailed inverse of the central Beta-distribution}
{Prob? mpReal? A real number}
{a? mpNum? A real number greater 0, representing the numerator  degrees of freedom}
{b? mpNum? A real number greater 0, representing the denominator degrees of freedom}
\end{mpFunctionsExtract}

\begin{mpFunctionsExtract}
\mpWorksheetFunctionThree
{BETA.INV? mpReal? the left-tailed inverse of the central Beta-distribution}
{Prob? mpReal? A real number}
{a? mpNum? A real number greater 0, representing the numerator  degrees of freedom}
{b? mpNum? A real number greater 0, representing the denominator degrees of freedom}
\end{mpFunctionsExtract}

\begin{mpFunctionsExtract}
\mpFunctionThreeNotImplemented
{BetaDistInfo? mpNumList? moments and related information for the central Beta-distribution}
{a? mpNum? A real number greater 0, representing the degrees of freedom}
{b? mpNum? A real number greater 0, representing the degrees of freedom}
{Output? String? A string describing the output choices}
\end{mpFunctionsExtract}

\begin{mpFunctionsExtract}
\mpFunctionFiveNotImplemented
{BetaDistRandom? mpNumList? random numbers following a central Beta-distribution}
{Size? mpNum? A positive integer up to $10^7$}
{a? mpNum? A real number greater 0, representing the numerator  degrees of freedom}
{b? mpNum? A real number greater 0, representing the denominator degrees of freedom}
{Generator? String? A string describing the random generator}
{Output? String? A string describing the output choices}
\end{mpFunctionsExtract}

\section{Binomial Distribution}

\begin{mpFunctionsExtract}
\mpFunctionFourNotImplemented
{BinomialDist? mpNumList? pdf, CDF and related information for the central Binomial-distribution}
{x? mpNum? The number of successes in trials.}
{n? mpNum? The number of independent trials.}
{p? mpNum? The probability of success on each trial}
{Output? String? A string describing the output choices}
\end{mpFunctionsExtract}

\begin{mpFunctionsExtract}
\mpWorksheetFunctionFourNotImplemented
{BINOMDIST? mpReal? pdf, CDF, and related information of the central Binomial-distribution}
{x? mpNum? The number of successes in trials.}
{n? mpNum? The number of independent trials.}
{p? mpNum? The probability of success on each trial}
{Cumulative ? Boolean? A logical value that determines the form of the function. If cumulative is TRUE, T.DIST returns the cumulative distribution function; if FALSE, it returns the probability density function}
\end{mpFunctionsExtract}

\begin{mpFunctionsExtract}
\mpWorksheetFunctionFourNotImplemented
{BINOM.DIST? mpReal? the CDF and pdf of the central Binomial-distribution}
{x? mpNum? The number of successes in trials.}
{n? mpNum? The number of independent trials.}
{p? mpNum? The probability of success on each trial}
{Cumulative ? Boolean? A logical value that determines the form of the function. If cumulative is TRUE, T.DIST returns the cumulative distribution function; if FALSE, it returns the probability density function}
\end{mpFunctionsExtract}

\begin{mpFunctionsExtract}
\mpWorksheetFunctionFourNotImplemented
{BINOM.DIST.RANGE? mpReal?  the probability that the number of successful trials will fall between x1 and x22}
{n? mpNum? The number of independent trials.}
{p? mpNum? The probability of success on each trial}
{x1? mpNum? The number x1 of successes in trials.}
{x2? mpNum? The number x2 of successes in trials.}
\end{mpFunctionsExtract}

\begin{mpFunctionsExtract}
\mpFunctionFourNotImplemented
{BinomialDistInv? mpNumList? quantiles and related information for the the central binomial-distribution}
{Prob? mpNum? A real number between 0 and 1.}
{n? mpNum? The number of Bernoulli trials.}
{p? mpNum? The probability of a success on each trial.}
{Output? String? A string describing the output choices}
\end{mpFunctionsExtract}

\begin{mpFunctionsExtract}
\mpWorksheetFunctionThreeNotImplemented
{CRITBINOM? mpReal? the smallest value for which the cumulative binomial distribution is greater than or equal to a criterion value.}
{n? mpNum? The number of Bernoulli trials.}
{p? mpNum? The probability of a success on each trial.}
{Alpha? mpReal? The criterion value.}
\end{mpFunctionsExtract}

\begin{mpFunctionsExtract}
\mpWorksheetFunctionThreeNotImplemented
{BINOM.INV? mpReal? the smallest value for which the cumulative binomial distribution is greater than or equal to a criterion value.}
{n? mpNum? The number of Bernoulli trials.}
{p? mpNum? The probability of a success on each trial.}
{Alpha? mpReal? The criterion value.}
\end{mpFunctionsExtract}

\begin{mpFunctionsExtract}
\mpFunctionThreeNotImplemented
{BinomialDistInfo? mpNumList? moments and related information for the central Binomial-distribution}
{n? mpNum? The number of Bernoulli trials.}
{p? mpNum? The probability of a success on each trial.}
{Output? String? A string describing the output choices}
\end{mpFunctionsExtract}

\begin{mpFunctionsExtract}
\mpFunctionFiveNotImplemented
{BinomialDistRandom? mpNumList? random numbers following a central Binomial-distribution}
{Size? mpNum? A positive integer up to $10^7$}
{n? mpNum? The number of Bernoulli trials.}
{p? mpNum? The probability of a success on each trial.}
{Generator? String? A string describing the random generator}
{Output? String? A string describing the output choices}
\end{mpFunctionsExtract}

\section{Chi-Square Distribution}

\begin{mpFunctionsExtract}
\mpFunctionThreeNotImplemented
{CDist? mpNumList? pdf, CDF and related information for the central $\chi^2$-distribution}
{x? mpNum? A real number}
{n? mpNum? A real number greater 0, representing the degrees of freedom}
{Output? String? A string describing the output choices}
\end{mpFunctionsExtract}

\begin{mpFunctionsExtract}
\mpWorksheetFunctionThreeNotImplemented
{CHIDIST? mpReal? the CDF and of the central $\chi^2$-distribution}
{x? mpReal? A real number. The numeric value at which to evaluate the distribution}
{deg\_freedom? mpReal? An integer  greater 0, indicating the degrees of freedom}
{Tails? Integer? Specifies the number of distribution tails to return. If tails = 1, TDIST returns the one-tailed distribution. If tails = 2, TDIST returns the two-tailed distribution.}
\end{mpFunctionsExtract}

\begin{mpFunctionsExtract}
\mpWorksheetFunctionThreeNotImplemented
{CHISQDIST? mpReal? the CDF and of the central $\chi^2$-distribution}
{x? mpReal? A real number. The numeric value at which to evaluate the distribution}
{deg\_freedom? mpReal? An integer  greater 0, indicating the degrees of freedom}
{Tails? Integer? Specifies the number of distribution tails to return. If tails = 1, TDIST returns the one-tailed distribution. If tails = 2, TDIST returns the two-tailed distribution.}
\end{mpFunctionsExtract}

\begin{mpFunctionsExtract}
\mpWorksheetFunctionThreeNotImplemented
{CHISQ.DIST? mpReal? the CDF and of the central $\chi^2$-distribution}
{x? mpReal? A real number. The numeric value at which to evaluate the distribution}
{deg\_freedom? mpReal? An integer  greater 0, indicating the degrees of freedom}
{Cumulative ? Boolean? A logical value that determines the form of the function. If cumulative is TRUE, T.DIST returns the cumulative distribution function; if FALSE, it returns the probability density function}
\end{mpFunctionsExtract}

\begin{mpFunctionsExtract}
\mpWorksheetFunctionTwoNotImplemented
{CHISQ.DIST.RT? mpReal? the complement of the CDF and of the central $\chi^2$-distribution}
{x? mpReal? A real number}
{deg\_freedom? mpReal? An integer  greater 0, indicating the degrees of freedom}
\end{mpFunctionsExtract}

\begin{mpFunctionsExtract}
\mpWorksheetFunctionTwoNotImplemented
{CHISQ.DIST.2T? mpReal? the two-sided CDF of the central $\chi^2$-distribution}
{x? mpReal? A real number}
{deg\_freedom? mpReal? An integer  greater 0, indicating the degrees of freedom}
\end{mpFunctionsExtract}

\begin{mpFunctionsExtract}
\mpFunctionThreeNotImplemented
{CDistInv? mpNumList? quantiles and related information for the the central $\chi^2$-distribution}
{Prob? mpNum? A real number between 0 and 1.}
{n? mpNum? A real number greater 0, representing the degrees of freedom}
{Output? String? A string describing the output choices}
\end{mpFunctionsExtract}

\begin{mpFunctionsExtract}
\mpWorksheetFunctionTwoNotImplemented
{CHIINV? mpReal? the two-tailed inverse of the central $\chi^2$-distribution}
{x? mpReal? A real number}
{deg\_freedom? mpReal? An integer  greater 0, indicating the degrees of freedom}
\end{mpFunctionsExtract}

\begin{mpFunctionsExtract}
\mpWorksheetFunctionTwoNotImplemented
{CHISQ.INV? mpReal? the left-tailed inverse of the central $\chi^2$-distribution}
{x? mpReal? A real number}
{deg\_freedom? mpReal? An integer  greater 0, indicating the degrees of freedom}
\end{mpFunctionsExtract}

\begin{mpFunctionsExtract}
\mpWorksheetFunctionTwoNotImplemented
{CHISQ.INV.RT? mpReal? the two-tailed inverse of the central $\chi^2$-distribution}
{x? mpReal? A real number}
{deg\_freedom? mpReal? An integer  greater 0, indicating the degrees of freedom}
\end{mpFunctionsExtract}

\begin{mpFunctionsExtract}
\mpWorksheetFunctionTwoNotImplemented
{CHISQINV? mpReal? the two-tailed inverse of the central $\chi^2$-distribution}
{x? mpReal? A real number}
{deg\_freedom? mpReal? An integer  greater 0, indicating the degrees of freedom}
\end{mpFunctionsExtract}

\begin{mpFunctionsExtract}
\mpFunctionTwoNotImplemented
{CDistInfo? mpNumList? moments and related information for the central $\chi^2$-distribution}
{n? mpNum? A real number greater 0, representing the degrees of freedom}
{Output? String? A string describing the output choices}
\end{mpFunctionsExtract}

\begin{mpFunctionsExtract}
\mpFunctionFourNotImplemented
{CDistRan? mpNumList? random numbers following a central $\chi^2$-distribution}
{Size? mpNum? A positive integer up to $10^7$}
{n? mpNum? A real number greater 0, representing the degrees of freedom}
{Generator? String? A string describing the random generator}
{Output? String? A string describing the output choices}
\end{mpFunctionsExtract}

\section{Exponential Distribution}

\begin{mpFunctionsExtract}
\mpFunctionThreeNotImplemented
{ExponentialDist? mpNumList? pdf, CDF and related information for the central Exponential distribution}
{x? mpNum? The value of the distribution.}
{lambda? mpNum? The parameter of the distribution.}
{Output? String? A string describing the output choices}
\end{mpFunctionsExtract}

\begin{mpFunctionsExtract}
\mpWorksheetFunctionThreeNotImplemented
{EXPONDIST? mpReal? pdf, CDF, and related information of the central Binomial-distribution}
{x? mpNum? The value of the distribution.}
{lambda? mpNum? The parameter of the distribution.}
{Cumulative ? Boolean? A logical value that determines the form of the function. If cumulative is TRUE, T.DIST returns the cumulative distribution function; if FALSE, it returns the probability density function}
\end{mpFunctionsExtract}

\begin{mpFunctionsExtract}
\mpWorksheetFunctionThreeNotImplemented
{EXPON.DIST? mpReal? the CDF and pdf of the central Binomial-distribution}
{x? mpNum? The value of the distribution.}
{lambda? mpNum? The parameter of the distribution.}
{Cumulative ? Boolean? A logical value that determines the form of the function. If cumulative is TRUE, T.DIST returns the cumulative distribution function; if FALSE, it returns the probability density function}
\end{mpFunctionsExtract}

\begin{mpFunctionsExtract}
\mpFunctionThreeNotImplemented
{ExponentialDistInv? mpNumList? quantiles and related information for the the central Exponential distribution}
{Prob? mpNum? A real number between 0 and 1.}
{lambda? mpNum? The number of Bernoulli trials.}
{Output? String? A string describing the output choices}
\end{mpFunctionsExtract}

\begin{mpFunctionsExtract}
\mpFunctionTwoNotImplemented
{ExponentialDistInfo? mpNumList? moments and related information for the central $t$-distribution}
{lambda? mpNum? A real number greater 0, representing the parameter of the distribution}
{Output? String? A string describing the output choices}
\end{mpFunctionsExtract}

\begin{mpFunctionsExtract}
\mpFunctionFourNotImplemented
{ExponentialDistRandom? mpNumList? random numbers following a central Beta-distribution}
{Size? mpNum? A positive integer up to $10^7$}
{lambda? mpNum? A real number greater 0, representing the numerator  degrees of freedom}
{Generator? String? A string describing the random generator}
{Output? String? A string describing the output choices}
\end{mpFunctionsExtract}

\section{Fisher's F-Distribution}

\begin{mpFunctionsExtract}
\mpFunctionFourNotImplemented
{FDist? mpNumList? pdf, CDF and related information for the central $F$-distribution}
{x? mpNum? A real number}
{m? mpNum? A real number greater 0, representing the numerator  degrees of freedom}
{n? mpNum? A real number greater 0, representing the denominator degrees of freedom}
{Output? String? A string describing the output choices}
\end{mpFunctionsExtract}

\begin{mpFunctionsExtract}
\mpWorksheetFunctionThreeNotImplemented
{FDIST? mpReal? the CDF and of the central $F$-distribution}
{x? mpReal? A real number. The numeric value at which to evaluate the distribution}
{m? mpNum? A real number greater 0, representing the numerator  degrees of freedom}
{n? mpNum? A real number greater 0, representing the denominator degrees of freedom}
\end{mpFunctionsExtract}

\begin{mpFunctionsExtract}
\mpWorksheetFunctionFourNotImplemented
{F.DIST? mpReal? the CDF and of the central $F$-distribution}
{x? mpReal? A real number. The numeric value at which to evaluate the distribution}
{m? mpNum? A real number greater 0, representing the numerator  degrees of freedom}
{n? mpNum? A real number greater 0, representing the denominator degrees of freedom}
{Cumulative ? Boolean? A logical value that determines the form of the function. If cumulative is TRUE, F.DIST returns the cumulative distribution function; if FALSE, it returns the probability density function}
\end{mpFunctionsExtract}

\begin{mpFunctionsExtract}
\mpWorksheetFunctionThreeNotImplemented
{F.DIST.RT? mpReal? the complement of the CDF and of the central $F$-distribution}
{x? mpReal? A real number}
{m? mpNum? A real number greater 0, representing the numerator  degrees of freedom}
{n? mpNum? A real number greater 0, representing the denominator degrees of freedom}
\end{mpFunctionsExtract}

\begin{mpFunctionsExtract}
\mpFunctionThreeNotImplemented
{FDistInv? mpNumList? quantiles and related information for the the central $t$-distribution}
{Prob? mpNum? A real number between 0 and 1.}
{m? mpNum? A real number greater 0, representing the numerator  degrees of freedom}
{n? mpNum? A real number greater 0, representing the denominator degrees of freedom}
{Output? String? A string describing the output choices}
\end{mpFunctionsExtract}

\begin{mpFunctionsExtract}
\mpWorksheetFunctionThreeNotImplemented
{FINV? mpReal? the two-tailed inverse of the central $t$-distribution}
{x? mpReal? A real number}
{m? mpNum? A real number greater 0, representing the numerator  degrees of freedom}
{n? mpNum? A real number greater 0, representing the denominator degrees of freedom}
\end{mpFunctionsExtract}

\begin{mpFunctionsExtract}
\mpWorksheetFunctionThreeNotImplemented
{F.INV? mpReal? the left-tailed inverse of the central $t$-distribution}
{x? mpReal? A real number}
{m? mpNum? A real number greater 0, representing the numerator  degrees of freedom}
{n? mpNum? A real number greater 0, representing the denominator degrees of freedom}
\end{mpFunctionsExtract}

\begin{mpFunctionsExtract}
\mpWorksheetFunctionThreeNotImplemented
{F.INV.RT? mpReal? the right-tailed inverse of the central $t$-distribution}
{x? mpReal? A real number}
{m? mpNum? A real number greater 0, representing the numerator  degrees of freedom}
{n? mpNum? A real number greater 0, representing the denominator degrees of freedom}
\end{mpFunctionsExtract}

\begin{mpFunctionsExtract}
\mpFunctionThreeNotImplemented
{FDistInfo? mpNumList? moments and related information for the central $t$-distribution}
{m? mpNum? A real number greater 0, representing the numerator  degrees of freedom}
{n? mpNum? A real number greater 0, representing the denominator degrees of freedom}
{Output? String? A string describing the output choices}
\end{mpFunctionsExtract}

\begin{mpFunctionsExtract}
\mpFunctionFiveNotImplemented
{FDistRan? mpNumList? random numbers following a central $F$-distribution}
{Size? mpNum? A positive integer up to $10^7$}
{m? mpNum? A real number greater 0, representing the numerator  degrees of freedom}
{n? mpNum? A real number greater 0, representing the denominator degrees of freedom}
{Generator? String? A string describing the random generator}
{Output? String? A string describing the output choices}
\end{mpFunctionsExtract}

\section{Gamma (and Erlang) Distribution}

\begin{mpFunctionsExtract}
\mpFunctionFourNotImplemented
{GammaDist? mpNumList? pdf, CDF and related information for the central Gamma-distribution}
{x? mpNum? A real number}
{a? mpNum? A real number greater 0, a parameter to the distribution}
{b? mpNum? A real number greater 0, a parameter to the distribution}
{Output? String? A string describing the output choices}
\end{mpFunctionsExtract}

\begin{mpFunctionsExtract}
\mpWorksheetFunctionFourNotImplemented
{GAMMADIST? mpReal? the CDF and of the central Gamma-distribution}
{x? mpReal? A real number. The numeric value at which to evaluate the distribution}
{a? mpNum? A real number greater 0, a parameter to the distribution}
{b? mpNum? A real number greater 0, a parameter to the distribution}
{Cumulative ? Boolean? A logical value that determines the form of the function. If cumulative is TRUE, GAMMA.DIST returns the cumulative distribution function; if FALSE, it returns the probability density function.}
\end{mpFunctionsExtract}

\begin{mpFunctionsExtract}
\mpWorksheetFunctionFourNotImplemented
{GAMMA.DIST? mpReal? the CDF and of the central Gamma-distribution}
{x? mpReal? A real number. The numeric value at which to evaluate the distribution}
{a? mpNum? A real number greater 0, a parameter to the distribution}
{b? mpNum? A real number greater 0, a parameter to the distribution}
{Cumulative ? Boolean? A logical value that determines the form of the function. If cumulative is TRUE, GAMMA.DIST returns the cumulative distribution function; if FALSE, it returns the probability density function.}
\end{mpFunctionsExtract}

\begin{mpFunctionsExtract}
\mpFunctionThreeNotImplemented
{GammaDistInv? mpNumList? quantiles and related information for the the central Gamma-distribution}
{Prob? mpNum? A real number between 0 and 1.}
{m? mpNum? A real number greater 0, a parameter to the distribution}
{n? mpNum? A real number greater 0, a parameter to the distribution}
{Output? String? A string describing the output choices}
\end{mpFunctionsExtract}

\begin{mpFunctionsExtract}
\mpWorksheetFunctionThreeNotImplemented
{GAMMAINV? mpReal? the two-tailed inverse of the central Gamma-distribution}
{Prob? mpReal? A real number}
{a? mpNum? A real number greater 0, a parameter to the distribution}
{b? mpNum? A real number greater 0, a parameter to the distribution}
\end{mpFunctionsExtract}

\begin{mpFunctionsExtract}
\mpWorksheetFunctionThreeNotImplemented
{GAMMA.INV? mpReal? the left-tailed inverse of the central Gamma-distribution}
{Prob? mpReal? A real number}
{a? mpNum? A real number greater 0, a parameter to the distribution}
{b? mpNum? A real number greater 0, a parameter to the distribution}
\end{mpFunctionsExtract}

\begin{mpFunctionsExtract}
\mpFunctionTwoNotImplemented
{GammaDistInfo? mpNumList? moments and related information for the central Gamma-distribution}
{a? mpNum? A real number greater 0, representing the degrees of freedom}
{b? mpNum? A real number greater 0, representing the degrees of freedom}
{Output? String? A string describing the output choices}
\end{mpFunctionsExtract}

\begin{mpFunctionsExtract}
\mpFunctionFiveNotImplemented
{GammaDistRandom? mpNumList? random numbers following a central Beta-distribution}
{Size? mpNum? A positive integer up to $10^7$}
{a? mpNum? A real number greater 0, a parameter to the distribution}
{b? mpNum? A real number greater 0, a parameter to the distribution}
{Generator? String? A string describing the random generator}
{Output? String? A string describing the output choices}
\end{mpFunctionsExtract}

\section{Hypergeometric Distribution}

\begin{mpFunctionsExtract}
\mpFunctionFiveNotImplemented
{HypergeometricDist? mpNumList? pdf, CDF and related information for the central hypergeometric distribution}
{x? mpNum? The number of successes in the sample.}
{n? mpNum? The size of the sample.}
{M? mpNum? The number of successes in the population}
{N? mpNum? The population size}
{Output? String? A string describing the output choices}
\end{mpFunctionsExtract}

\begin{mpFunctionsExtract}
\mpWorksheetFunctionFiveNotImplemented
{HYPGEOMDIST? mpReal? pdf, CDF, and related information of the central hypergeometric distribution}
{x? mpNum? The number of successes in the sample.}
{n? mpNum? The size of the sample.}
{M? mpNum? The number of successes in the population}
{N? mpNum? The population size}
{Cumulative ? Boolean? A logical value that determines the form of the function. If cumulative is TRUE, T.DIST returns the cumulative distribution function; if FALSE, it returns the probability density function}
\end{mpFunctionsExtract}

\begin{mpFunctionsExtract}
\mpWorksheetFunctionFiveNotImplemented
{HYPGEOM.DIST? mpReal? the CDF and pdf of the central hypergeometric distribution}
{x? mpNum? The number of successes in the sample.}
{n? mpNum? The size of the sample.}
{M? mpNum? The number of successes in the population}
{N? mpNum? The population size}
{Cumulative ? Boolean? A logical value that determines the form of the function. If cumulative is TRUE, T.DIST returns the cumulative distribution function; if FALSE, it returns the probability density function}
\end{mpFunctionsExtract}

\begin{mpFunctionsExtract}
\mpFunctionFiveNotImplemented
{HypergeometricDistInv? mpNumList? quantiles and related information for the the central hypergeometric distribution}
{Prob? mpNum? A real number between 0 and 1.}
{n? mpNum? The size of the sample.}
{M? mpNum? The number of successes in the population}
{N? mpNum? The population size}
{Output? String? A string describing the output choices}
\end{mpFunctionsExtract}

\begin{mpFunctionsExtract}
\mpFunctionFourNotImplemented
{HypergeometricDistInfo? mpNumList? moments and related information for the central hypergeometric distribution}
{n? mpNum? The size of the sample.}
{M? mpNum? The number of successes in the population}
{N? mpNum? The population size}
{Output? String? A string describing the output choices}
\end{mpFunctionsExtract}

\begin{mpFunctionsExtract}
\mpFunctionSixNotImplemented
{HypergeometricDistRandom? mpNumList? random numbers following a central hypergeometric distribution}
{Size? mpNum? A positive integer up to $10^7$}
{n? mpNum? The size of the sample.}
{M? mpNum? The number of successes in the population}
{N? mpNum? The population size}
{Generator? String? A string describing the random generator}
{Output? String? A string describing the output choices}
\end{mpFunctionsExtract}

\section{Lognormal Distribution}

\begin{mpFunctionsExtract}
\mpFunctionFourNotImplemented
{LogNormalDist? mpNumList? pdf, CDF and related information for the Lognormal-distribution}
{x? mpNum? A real number}
{mean? mpNum? A real number greater 0, representing the mean of the distribution}
{stdev? mpNum? A real number greater 0, representing the standard deviation of the distribution}
{Output? String? A string describing the output choices}
\end{mpFunctionsExtract}

\begin{mpFunctionsExtract}
\mpWorksheetFunctionThreeNotImplemented
{LOGNORMDIST? mpReal? the CDF and of the Lognormal-distribution}
{x? mpReal? A real number. The numeric value at which to evaluate the distribution}
{mean? mpNum? A real number greater 0, representing the mean of the distribution}
{stdev? mpNum? A real number greater 0, representing the standard deviation of the distribution}
\end{mpFunctionsExtract}

\begin{mpFunctionsExtract}
\mpWorksheetFunctionFourNotImplemented
{LOGNORM.DIST? mpReal? the CDF and of the Lognormal-distribution}
{x? mpReal? A real number. The numeric value at which to evaluate the distribution}
{mean? mpNum? A real number greater 0, representing the mean of the distribution}
{stdev? mpNum? A real number greater 0, representing the standard deviation of the distribution}
{Cumulative ? Boolean? A logical value that determines the form of the function. If cumulative is TRUE, T.DIST returns the cumulative distribution function; if FALSE, it returns the probability density function}
\end{mpFunctionsExtract}

\begin{mpFunctionsExtract}
\mpFunctionFourNotImplemented
{LognormalDistInv? mpNumList? quantiles and related information for the the Lognormal-distribution}
{Prob? mpNum? A real number between 0 and 1.}
{mean? mpNum? A real number greater 0, representing the mean of the distribution}
{stdev? mpNum? A real number greater 0, representing the standard deviation of the distribution}
{Output? String? A string describing the output choices}
\end{mpFunctionsExtract}

\begin{mpFunctionsExtract}
\mpWorksheetFunctionThreeNotImplemented
{LOGINV? mpReal? the two-tailed inverse of the Lognormal-distribution}
{Prob? mpReal? A real number}
{mean? mpNum? A real number greater 0, representing the mean of the distribution}
{stdev? mpNum? A real number greater 0, representing the standard deviation of the distribution}
\end{mpFunctionsExtract}

\begin{mpFunctionsExtract}
\mpWorksheetFunctionThreeNotImplemented
{LOGNORM.INV? mpReal? the left-tailed inverse of the Lognormal-distribution}
{Prob? mpReal? A real number}
{mean? mpNum? A real number greater 0, representing the mean of the distribution}
{stdev? mpNum? A real number greater 0, representing the standard deviation of the distribution}
\end{mpFunctionsExtract}

\begin{mpFunctionsExtract}
\mpFunctionThreeNotImplemented
{LognormalDistInfo? mpNumList? moments and related information for the central Lognormal-distribution}
{mean? mpNum? A real number greater 0, representing the mean of the distribution}
{stdev? mpNum? A real number greater 0, representing the standard deviation of the distribution}
{Output? String? A string describing the output choices}
\end{mpFunctionsExtract}

\begin{mpFunctionsExtract}
\mpFunctionFiveNotImplemented
{LognormalRandom? mpNumList? random numbers following a central Beta-distribution}
{Size? mpNum? A positive integer up to $10^7$}
{mean? mpNum? A real number greater 0, representing the mean of the distribution}
{stdev? mpNum? A real number greater 0, representing the standard deviation of the distribution}
{Generator? String? A string describing the random generator}
{Output? String? A string describing the output choices}
\end{mpFunctionsExtract}

\section{Negative Binomial Distribution}

\begin{mpFunctionsExtract}
\mpFunctionFourNotImplemented
{NegativeBinomialDist? mpNumList? pdf, CDF and related information for the central negative binomial distribution}
{x? mpNum? The number of failures in trials.}
{r? mpNum? The threshold number of successes.}
{p? mpNum? The probability of a success}
{Output? String? A string describing the output choices}
\end{mpFunctionsExtract}

\begin{mpFunctionsExtract}
\mpWorksheetFunctionFourNotImplemented
{NEGBINOMDIST? mpReal? pdf, CDF, and related information of the central negative binomial distribution}
{x? mpNum? The number of failures in trials.}
{r? mpNum? The threshold number of successes.}
{p? mpNum? The probability of a success}
{Cumulative ? Boolean? A logical value that determines the form of the function. If cumulative is TRUE, T.DIST returns the cumulative distribution function; if FALSE, it returns the probability density function}
\end{mpFunctionsExtract}

\begin{mpFunctionsExtract}
\mpWorksheetFunctionFourNotImplemented
{NEGBINOM.DIST? mpReal? the CDF and pdf of the central negative binomial distribution}
{x? mpNum? The number of failures in trials.}
{r? mpNum? The threshold number of successes.}
{p? mpNum? The probability of a success}
{Cumulative ? Boolean? A logical value that determines the form of the function. If cumulative is TRUE, T.DIST returns the cumulative distribution function; if FALSE, it returns the probability density function}
\end{mpFunctionsExtract}

\begin{mpFunctionsExtract}
\mpFunctionFourNotImplemented
{NegativeBinomialDistInv? mpNumList? quantiles and related information for the the central binomial-distribution}
{Prob? mpNum? A real number between 0 and 1.}
{r? mpNum? The threshold number of successes.}
{p? mpNum? The probability of a success}
{Output? String? A string describing the output choices}
\end{mpFunctionsExtract}

\begin{mpFunctionsExtract}
\mpFunctionThreeNotImplemented
{NegativeBinomialDistInfo? mpNumList? moments and related information for the central Binomial-distribution}
{r? mpNum? The threshold number of successes.}
{p? mpNum? The probability of a success}
{Output? String? A string describing the output choices}
\end{mpFunctionsExtract}

\begin{mpFunctionsExtract}
\mpFunctionFiveNotImplemented
{NegativeBinomialDistRandom? mpNumList? random numbers following a central Binomial-distribution}
{Size? mpNum? A positive integer up to $10^7$}
{r? mpNum? The threshold number of successes.}
{p? mpNum? The probability of a success}
{Generator? String? A string describing the random generator}
{Output? String? A string describing the output choices}
\end{mpFunctionsExtract}

\section{Normal Distribution}

\begin{mpFunctionsExtract}
\mpFunctionFourNotImplemented
{NDist? mpNumList? pdf, CDF and related information for the normal-distribution}
{x? mpNum? A real number}
{mean? mpNum? A real number greater 0, representing the mean of the distribution}
{stdev? mpNum? A real number greater 0, representing the standard deviation of the distribution}
{Output? String? A string describing the output choices}
\end{mpFunctionsExtract}

\begin{mpFunctionsExtract}
\mpWorksheetFunctionThreeNotImplemented
{NORMDIST? mpReal? the CDF and of the Lognormal-distribution}
{x? mpReal? A real number. The numeric value at which to evaluate the distribution}
{mean? mpNum? A real number greater 0, representing the mean of the distribution}
{stdev? mpNum? A real number greater 0, representing the standard deviation of the distribution}
\end{mpFunctionsExtract}

\begin{mpFunctionsExtract}
\mpWorksheetFunctionFourNotImplemented
{NORM.DIST? mpReal? the CDF and of the Lognormal-distribution}
{x? mpReal? A real number. The numeric value at which to evaluate the distribution}
{mean? mpNum? A real number greater 0, representing the mean of the distribution}
{stdev? mpNum? A real number greater 0, representing the standard deviation of the distribution}
{Cumulative ? Boolean? A logical value that determines the form of the function. If cumulative is TRUE, T.DIST returns the cumulative distribution function; if FALSE, it returns the probability density function}
\end{mpFunctionsExtract}

\begin{mpFunctionsExtract}
\mpWorksheetFunctionOneNotImplemented
{NORMSDIST? mpReal? the CDF and of the  standard normal distribution}
{x? mpReal? A real number. The numeric value at which to evaluate the distribution}
\end{mpFunctionsExtract}

\begin{mpFunctionsExtract}
\mpWorksheetFunctionTwoNotImplemented
{NORM.S.DIST? mpReal? the CDF and of the  standard normal distribution}
{x? mpReal? A real number. The numeric value at which to evaluate the distribution}
{Cumulative ? Boolean? A logical value that determines the form of the function. If cumulative is TRUE, NORM.S.DIST returns the cumulative distribution function; if FALSE, it returns the probability density function}
\end{mpFunctionsExtract}

\begin{mpFunctionsExtract}
\mpWorksheetFunctionOneNotImplemented
{GAUSS? mpReal? the CDF of the  standard normal distribution}
{x? mpReal? A real number. The numeric value at which to evaluate the distribution}
\end{mpFunctionsExtract}

\begin{mpFunctionsExtract}
\mpWorksheetFunctionOneNotImplemented
{PHI? mpReal? the pdf of the  standard normal distribution}
{x? mpReal? A real number. The numeric value at which to evaluate the distribution}
\end{mpFunctionsExtract}

\begin{mpFunctionsExtract}
\mpFunctionFourNotImplemented
{NDistInv? mpNumList? quantiles and related information for the the Lognormal-distribution}
{Prob? mpNum? A real number between 0 and 1.}
{mean? mpNum? A real number greater 0, representing the mean of the distribution}
{stdev? mpNum? A real number greater 0, representing the standard deviation of the distribution}
{Output? String? A string describing the output choices}
\end{mpFunctionsExtract}

\begin{mpFunctionsExtract}
\mpWorksheetFunctionThreeNotImplemented
{NORMINV? mpReal? the two-tailed inverse of the normal distribution}
{Prob? mpReal? A real number}
{mean? mpNum? A real number greater 0, representing the mean of the distribution}
{stdev? mpNum? A real number greater 0, representing the standard deviation of the distribution}
\end{mpFunctionsExtract}

\begin{mpFunctionsExtract}
\mpWorksheetFunctionThreeNotImplemented
{NORM.INV? mpReal? the left-tailed inverse of the normal distribution}
{Prob? mpReal? A real number}
{mean? mpNum? A real number greater 0, representing the mean of the distribution}
{stdev? mpNum? A real number greater 0, representing the standard deviation of the distribution}
\end{mpFunctionsExtract}

\begin{mpFunctionsExtract}
\mpWorksheetFunctionOneNotImplemented
{NORMSINV? mpReal? the two-tailed inverse of the standardized normal distribution}
{Prob? mpReal? A real number}
\end{mpFunctionsExtract}

\begin{mpFunctionsExtract}
\mpWorksheetFunctionOneNotImplemented
{NORM.S.INV? mpReal? the left-tailed inverse of the standardized normal distribution}
{Prob? mpReal? A real number}
\end{mpFunctionsExtract}

\begin{mpFunctionsExtract}
\mpFunctionThreeNotImplemented
{NormalDistInfo? mpNumList? moments and related information for the central Lognormal-distribution}
{mean? mpNum? A real number greater 0, representing the mean of the distribution}
{stdev? mpNum? A real number greater 0, representing the standard deviation of the distribution}
{Output? String? A string describing the output choices}
\end{mpFunctionsExtract}

\begin{mpFunctionsExtract}
\mpFunctionFiveNotImplemented
{NormalRandom? mpNumList? random numbers following a central Beta-distribution}
{Size? mpNum? A positive integer up to $10^7$}
{mean? mpNum? A real number greater 0, representing the mean of the distribution}
{stdev? mpNum? A real number greater 0, representing the standard deviation of the distribution}
{Generator? String? A string describing the random generator}
{Output? String? A string describing the output choices}
\end{mpFunctionsExtract}

\section{Poisson Distribution}

\begin{mpFunctionsExtract}
\mpFunctionThreeNotImplemented
{PoissonDist? mpNumList? pdf, CDF and related information for the Poisson distribution}
{x? mpNum? A real number}
{lambda? mpNum? A real number greater 0, representing the degrees of freedom}
{Output? String? A string describing the output choices}
\end{mpFunctionsExtract}

\begin{mpFunctionsExtract}
\mpWorksheetFunctionThreeNotImplemented
{POISSON? mpReal? the CDF and of the Poisson distribution}
{x? mpReal? A real number. The numeric value at which to evaluate the distribution}
{deg\_freedom? mpReal? An integer  greater 0, indicating the degrees of freedom}
{Tails? Integer? Specifies the number of distribution tails to return. If tails = 1, TDIST returns the one-tailed distribution. If tails = 2, TDIST returns the two-tailed distribution.}
\end{mpFunctionsExtract}

\begin{mpFunctionsExtract}
\mpWorksheetFunctionThreeNotImplemented
{POISSON.DIST? mpReal? the CDF and of the Poisson distribution}
{x? mpReal? A real number. The numeric value at which to evaluate the distribution}
{deg\_freedom? mpReal? An integer  greater 0, indicating the degrees of freedom}
{Cumulative ? Boolean? A logical value that determines the form of the function. If cumulative is TRUE, POISSON.DIST returns the cumulative distribution function; if FALSE, it returns the probability density function}
\end{mpFunctionsExtract}

\begin{mpFunctionsExtract}
\mpFunctionThreeNotImplemented
{PoissonDistInv? mpNumList? quantiles and related information for the the Poisson distribution}
{Prob? mpNum? A real number between 0 and 1.}
{lambda? mpNum? A real number greater 0, representing the degrees of freedom}
{Output? String? A string describing the output choices}
\end{mpFunctionsExtract}

\begin{mpFunctionsExtract}
\mpFunctionTwoNotImplemented
{PoissonDistInfo? mpNumList? moments and related information for the Poisson distribution}
{lambda? mpNum? A real number greater 0, representing the degrees of freedom}
{Output? String? A string describing the output choices}
\end{mpFunctionsExtract}

\begin{mpFunctionsExtract}
\mpFunctionFourNotImplemented
{PoissonDistRan? mpNumList? random numbers following a Poisson distribution}
{Size? mpNum? A positive integer up to $10^7$}
{lambda? mpNum? A real number greater 0, representing the degrees of freedom}
{Generator? String? A string describing the random generator}
{Output? String? A string describing the output choices}
\end{mpFunctionsExtract}

\section{Student's t-Distribution}

\begin{mpFunctionsExtract}
\mpFunctionThreeNotImplemented
{TDist? mpNumList? pdf, CDF and related information for the central $t$-distribution}
{x? mpNum? A real number}
{n? mpNum? A real number greater 0, representing the degrees of freedom}
{Output? String? A string describing the output choices}
\end{mpFunctionsExtract}

\begin{mpFunctionsExtract}
\mpWorksheetFunctionThreeNotImplemented
{TDIST? mpReal? the CDF and of the central $t$-distribution}
{x? mpReal? A real number. The numeric value at which to evaluate the distribution}
{deg\_freedom? mpReal? An integer  greater 0, indicating the degrees of freedom}
{Tails? Integer? Specifies the number of distribution tails to return. If tails = 1, TDIST returns the one-tailed distribution. If tails = 2, TDIST returns the two-tailed distribution.}
\end{mpFunctionsExtract}

\begin{mpFunctionsExtract}
\mpWorksheetFunctionThreeNotImplemented
{T.DIST? mpReal? the CDF and of the central $t$-distribution}
{x? mpReal? A real number. The numeric value at which to evaluate the distribution}
{deg\_freedom? mpReal? An integer  greater 0, indicating the degrees of freedom}
{Cumulative ? Boolean? A logical value that determines the form of the function. If cumulative is TRUE, T.DIST returns the cumulative distribution function; if FALSE, it returns the probability density function}
\end{mpFunctionsExtract}

\begin{mpFunctionsExtract}
\mpWorksheetFunctionTwoNotImplemented
{T.DIST.RT? mpReal? the complement of the CDF and of the central $t$-distribution}
{x? mpReal? A real number}
{deg\_freedom? mpReal? An integer  greater 0, indicating the degrees of freedom}
\end{mpFunctionsExtract}

\begin{mpFunctionsExtract}
\mpWorksheetFunctionTwoNotImplemented
{T.DIST.2T? mpReal? the two-sided CDF of the central $t$-distribution}
{x? mpReal? A real number}
{deg\_freedom? mpReal? An integer  greater 0, indicating the degrees of freedom}
\end{mpFunctionsExtract}

\begin{mpFunctionsExtract}
\mpFunctionThreeNotImplemented
{TDistInv? mpNumList? quantiles and related information for the the central $t$-distribution}
{Prob? mpNum? A real number between 0 and 1.}
{n? mpNum? A real number greater 0, representing the degrees of freedom}
{Output? String? A string describing the output choices}
\end{mpFunctionsExtract}

\begin{mpFunctionsExtract}
\mpWorksheetFunctionTwoNotImplemented
{TINV? mpReal? the two-tailed inverse of the central $t$-distribution}
{x? mpReal? A real number}
{deg\_freedom? mpReal? An integer  greater 0, indicating the degrees of freedom}
\end{mpFunctionsExtract}

\begin{mpFunctionsExtract}
\mpWorksheetFunctionTwoNotImplemented
{T.INV? mpReal? the left-tailed inverse of the central $t$-distribution}
{x? mpReal? A real number}
{deg\_freedom? mpReal? An integer  greater 0, indicating the degrees of freedom}
\end{mpFunctionsExtract}

\begin{mpFunctionsExtract}
\mpWorksheetFunctionTwoNotImplemented
{T.INV.2T? mpReal? the two-tailed inverse of the central $t$-distribution}
{x? mpReal? A real number}
{deg\_freedom? mpReal? An integer  greater 0, indicating the degrees of freedom}
\end{mpFunctionsExtract}

\begin{mpFunctionsExtract}
\mpFunctionTwoNotImplemented
{TDistInfo? mpNumList? moments and related information for the central $t$-distribution}
{n? mpNum? A real number greater 0, representing the degrees of freedom}
{Output? String? A string describing the output choices}
\end{mpFunctionsExtract}

\begin{mpFunctionsExtract}
\mpFunctionFourNotImplemented
{TDistRan? mpNumList? random numbers following a central $t$-distribution}
{Size? mpNum? A positive integer up to $10^7$}
{n? mpNum? A real number greater 0, representing the degrees of freedom}
{Generator? String? A string describing the random generator}
{Output? String? A string describing the output choices}
\end{mpFunctionsExtract}

\section{Weibull Distribution}

\begin{mpFunctionsExtract}
\mpFunctionFourNotImplemented
{WeibullDist? mpNumList? pdf, CDF and related information for the Weibull distribution}
{x? mpNum? A real number}
{a? mpNum? A real number greater 0, representing the numerator  degrees of freedom}
{b? mpNum? A real number greater 0, representing the denominator degrees of freedom}
{Output? String? A string describing the output choices}
\end{mpFunctionsExtract}

\begin{mpFunctionsExtract}
\mpWorksheetFunctionThreeNotImplemented
{WEIBULL? mpReal? the CDF and of the Weibull distribution}
{x? mpReal? A real number. The numeric value at which to evaluate the distribution}
{a? mpNum? A real number greater 0, representing the numerator  degrees of freedom}
{b? mpNum? A real number greater 0, representing the denominator degrees of freedom}
\end{mpFunctionsExtract}

\begin{mpFunctionsExtract}
\mpWorksheetFunctionFourNotImplemented
{WEIBULL.DIST? mpReal? the CDF and of the Weibull distribution}
{x? mpReal? A real number. The numeric value at which to evaluate the distribution}
{a? mpNum? A real number greater 0, representing the numerator  degrees of freedom}
{b? mpNum? A real number greater 0, representing the denominator degrees of freedom}
{Cumulative ? Boolean? A logical value that determines the form of the function. If cumulative is TRUE, T.DIST returns the cumulative distribution function; if FALSE, it returns the probability density function}
\end{mpFunctionsExtract}

\begin{mpFunctionsExtract}
\mpFunctionFourNotImplemented
{WeibullDistInv? mpNumList? quantiles and related information for the the central Beta-distribution}
{Prob? mpNum? A real number between 0 and 1.}
{a? mpNum? A real number greater 0, representing the numerator  degrees of freedom}
{b? mpNum? A real number greater 0, representing the denominator degrees of freedom}
{Output? String? A string describing the output choices}
\end{mpFunctionsExtract}

\begin{mpFunctionsExtract}
\mpFunctionThreeNotImplemented
{WeibullDistInfo? mpNumList? moments and related information for the central Beta-distribution}
{a? mpNum? A real number greater 0, representing the degrees of freedom}
{b? mpNum? A real number greater 0, representing the degrees of freedom}
{Output? String? A string describing the output choices}
\end{mpFunctionsExtract}

\begin{mpFunctionsExtract}
\mpFunctionFiveNotImplemented
{WeibullDistRandom? mpNumList? random numbers following a central Beta-distribution}
{Size? mpNum? A positive integer up to $10^7$}
{a? mpNum? A real number greater 0, representing the numerator  degrees of freedom}
{b? mpNum? A real number greater 0, representing the denominator degrees of freedom}
{Generator? String? A string describing the random generator}
{Output? String? A string describing the output choices}
\end{mpFunctionsExtract}

\chapter{Statistical Functions}

\section{Frequencies and Percentages}

\begin{mpFunctionsExtract}
\mpWorksheetFunctionOneNotImplemented
{COUNT? mpNum? the number of cells that contain numbers, and counts numbers within the list of arguments.}
{x? mpNum[]? An array of real numbers.}
\end{mpFunctionsExtract}

\begin{mpFunctionsExtract}
\mpWorksheetFunctionOneNotImplemented
{COUNTA? mpNum? the number of cells that are not empty in a range.}
{x? mpNum[]? An array of real numbers.}
\end{mpFunctionsExtract}

\begin{mpFunctionsExtract}
\mpWorksheetFunctionOneNotImplemented
{COUNTBLANK? mpNum? the number of empty cells in a specified range of cells.}
{x? mpNum[]? An array of real numbers.}
\end{mpFunctionsExtract}

\begin{mpFunctionsExtract}
\mpWorksheetFunctionTwoNotImplemented
{COUNTIF? mpNum? the number of cells within a range that meet a single criterion that you specify.}
{x? mpNum[]? An array of real numbers.}
{Criteria? String? A String specifying the criteria.}
\end{mpFunctionsExtract}

\begin{mpFunctionsExtract}
\mpWorksheetFunctionTwoNotImplemented
{COUNTIFS? mpNum? the number of cells within multiple ranges that meet all criteria that you specify.}
{x? mpNumList? An array of real numbers.}
{Criteria? String[]? An array of strings specifying the criteria.}
\end{mpFunctionsExtract}

\begin{mpFunctionsExtract}
\mpWorksheetFunctionTwoNotImplemented
{FREQUENCY? mpNum? a vertical array of numbers, calculating how often values occur within a range of values.}
{DataArray? mpNum[]? An array of a set of values for which you want to count frequencies.}
{BinsArray? mpNum[]? An array of intervals into which you want to group the values in \textsf{DataArray}}
\end{mpFunctionsExtract}

\begin{mpFunctionsExtract}
\mpWorksheetFunctionTwoNotImplemented
{Histogram? mpNum? a vertical array of numbers, calculating how often values occur within a range of values.}
{DataArray? mpNum[]? An array of a set of values for which you want to count frequencies.}
{BinsArray? mpNum[]? An array of intervals into which you want to group the values in \textsf{DataArray}}
\end{mpFunctionsExtract}

\section{Transformations}

\begin{mpFunctionsExtract}
\mpWorksheetFunctionThreeNotImplemented
{CONVERT? mpNum? a number converted from one measurement system to another.}
{Number? mpNum? The value in \textsf{FromUnit} to convert.}
{FromUnit? String? The units for \textsf{Number}.}
{ToUnit? String? The units for the result.}
\end{mpFunctionsExtract}

\begin{mpFunctionsExtract}
\mpWorksheetFunctionThreeNotImplemented
{STANDARDIZE? mpNum? a normalized value  $Z$ from a distribution with mean $\mu$ and standard deviation $\sigma$.}
{Number? mpNum? The value you want to normalize.}
{Mean? mpNum? The arithmetic mean $\mu$ of the distribution.}
{StDev? mpNum? The standard deviation  $\sigma$ of the distribution.}
\end{mpFunctionsExtract}

\begin{mpFunctionsExtract}
\mpWorksheetFunctionTwoNotImplemented
{TRIMMEAN? mpNum? the mean of the interior of a data set.}
{X? mpNum[]? The array or range of values to trim and average.}
{Percent? mpNum? The fractional number of data points to exclude from the calculation.}
\end{mpFunctionsExtract}

\section{Sums, Means, Moments and Cumulants}

\begin{mpFunctionsExtract}
\mpWorksheetFunctionOneNotImplemented
{SUM? mpNum? the sum of the numbers in an array}
{x? mpNum[]? An array of real numbers.}
\end{mpFunctionsExtract}

\begin{mpFunctionsExtract}
\mpWorksheetFunctionOneNotImplemented
{SUMA? mpNum? the sum of the numbers in a list of arguments}
{x? mpNumList? An array of real numbers.}
\end{mpFunctionsExtract}

\begin{mpFunctionsExtract}
\mpWorksheetFunctionTwoNotImplemented
{SUMIF? mpNum? the sum of cells within a range that meet a single criterion that you specify.}
{x? mpNum[]? An array of real numbers.}
{Criteria? String? A String specifying the criteria.}
\end{mpFunctionsExtract}

\begin{mpFunctionsExtract}
\mpWorksheetFunctionTwoNotImplemented
{SUMIFS? mpNum? the sum of cells within multiple ranges that meet all criteria that you specify.}
{x? mpNumList? An array of real numbers.}
{Criteria? String[]? An array of strings specifying the criteria.}
\end{mpFunctionsExtract}

\begin{mpFunctionsExtract}
\mpWorksheetFunctionOneNotImplemented
{AVERAGE? mpNum? the average (arithmetic mean) of the numbers in an array}
{x? mpNum[]? An array of real numbers.}
\end{mpFunctionsExtract}

\begin{mpFunctionsExtract}
\mpWorksheetFunctionOneNotImplemented
{AVERAGEA? mpNum? the average (arithmetic mean) of the numbers in a list of arguments}
{x? mpNumList? An array of real numbers.}
\end{mpFunctionsExtract}

\begin{mpFunctionsExtract}
\mpWorksheetFunctionTwoNotImplemented
{AVERAGEIF? mpNum? the average (arithmetic mean) of cells within a range that meet a single criterion that you specify.}
{x? mpNum[]? An array of real numbers.}
{Criteria? String? A String specifying the criteria.}
\end{mpFunctionsExtract}

\begin{mpFunctionsExtract}
\mpWorksheetFunctionTwoNotImplemented
{AVERAGEIFS? mpNum? the average (arithmetic mean) of cells within multiple ranges that meet all criteria that you specify.}
{x? mpNumList? An array of real numbers.}
{Criteria? String[]? An array of strings specifying the criteria.}
\end{mpFunctionsExtract}

\begin{mpFunctionsExtract}
\mpWorksheetFunctionOneNotImplemented
{GEOMEAN? mpNum? the geometric mean of an array or range of positive data.}
{x? mpNum[]? An array of real numbers.}
\end{mpFunctionsExtract}

\begin{mpFunctionsExtract}
\mpWorksheetFunctionOneNotImplemented
{HARMEAN? mpNum? the harmonic mean of an array or range of positive data.}
{x? mpNum[]? An array of real numbers.}
\end{mpFunctionsExtract}

\begin{mpFunctionsExtract}
\mpWorksheetFunctionOneNotImplemented
{VAR? mpNum? the sample variance $s^2$ of an array or range of numerical data, ignoring non-numeric entries.}
{x? mpNum[]? An array of real numbers.}
\end{mpFunctionsExtract}

\begin{mpFunctionsExtract}
\mpWorksheetFunctionOneNotImplemented
{VAR.S? mpNum? the sample variance $s^2$ of an array or range of numerical data, ignoring non-numeric entries.}
{x? mpNum[]? An array of real numbers.}
\end{mpFunctionsExtract}

\begin{mpFunctionsExtract}
\mpWorksheetFunctionOneNotImplemented
{VARA? mpNum? the sample variance $s^2$ of an array or range of data, including text entries and FALSE as 0 and TRUE as 1.}
{x? mpNum[]? An array of real numbers.}
\end{mpFunctionsExtract}

\begin{mpFunctionsExtract}
\mpWorksheetFunctionOneNotImplemented
{VARP? mpNum? the population variance $S^2$ of an array or range of numerical data, ignoring non-numeric entries.}
{x? mpNum[]? An array of real numbers.}
\end{mpFunctionsExtract}

\begin{mpFunctionsExtract}
\mpWorksheetFunctionOneNotImplemented
{VAR.P? mpNum? the population variance $S^2$ of an array or range of numerical data, ignoring non-numeric entries.}
{x? mpNum[]? An array of real numbers.}
\end{mpFunctionsExtract}

\begin{mpFunctionsExtract}
\mpWorksheetFunctionOneNotImplemented
{VARPA? mpNum? the population variance $S^2$ of an array or range of data, including text entries and FALSE as 0 and TRUE as 1.}
{x? mpNum[]? An array of real numbers.}
\end{mpFunctionsExtract}

\begin{mpFunctionsExtract}
\mpWorksheetFunctionOneNotImplemented
{STDEV? mpNum? the sample standard deviation $s$ of an array or range of numerical data, ignoring non-numeric entries.}
{x? mpNum[]? An array of real numbers.}
\end{mpFunctionsExtract}

\begin{mpFunctionsExtract}
\mpWorksheetFunctionOneNotImplemented
{STDEV.S? mpNum? the sample standard deviation $s$ of an array or range of numerical data, ignoring non-numeric entries.}
{x? mpNum[]? An array of real numbers.}
\end{mpFunctionsExtract}

\begin{mpFunctionsExtract}
\mpWorksheetFunctionOneNotImplemented
{STDEVA? mpNum? the sample standard deviation $s$ of an array or range of data, including text entries and FALSE as 0 and TRUE as 1.}
{x? mpNum[]? An array of real numbers.}
\end{mpFunctionsExtract}

\begin{mpFunctionsExtract}
\mpWorksheetFunctionOneNotImplemented
{STDEVP? mpNum? the population standard deviation $S$ of an array or range of numerical data, ignoring non-numeric entries.}
{x? mpNum[]? An array of real numbers.}
\end{mpFunctionsExtract}

\begin{mpFunctionsExtract}
\mpWorksheetFunctionOneNotImplemented
{STDEV.P? mpNum? the population standard deviation $S$ of an array or range of numerical data, ignoring non-numeric entries.}
{x? mpNum[]? An array of real numbers.}
\end{mpFunctionsExtract}

\begin{mpFunctionsExtract}
\mpWorksheetFunctionOneNotImplemented
{STDEVPA? mpNum? the population standard deviation $S$ of an array or range of data, including text entries and FALSE as 0 and TRUE as 1.}
{x? mpNum[]? An array of real numbers.}
\end{mpFunctionsExtract}

\begin{mpFunctionsExtract}
\mpWorksheetFunctionOneNotImplemented
{AVEDEV? mpNum? the average of the absolute deviations of data points from their mean.}
{x? mpNum[]? An array of real numbers.}
\end{mpFunctionsExtract}

\begin{mpFunctionsExtract}
\mpWorksheetFunctionOneNotImplemented
{DEVSQ? mpNum? the sum of squares of deviations of data points from their mean.}
{x? mpNum[]? An array of real numbers.}
\end{mpFunctionsExtract}

\begin{mpFunctionsExtract}
\mpWorksheetFunctionOneNotImplemented
{SKEW? mpNum? the skewness of a sample.}
{x? mpNum[]? An array of real numbers.}
\end{mpFunctionsExtract}

\begin{mpFunctionsExtract}
\mpWorksheetFunctionOneNotImplemented
{SKEW.P? mpNum? the skewness of a population.}
{x? mpNum[]? An array of real numbers.}
\end{mpFunctionsExtract}

\begin{mpFunctionsExtract}
\mpWorksheetFunctionOneNotImplemented
{KURT? mpNum? the kurtosis of a sample.}
{x? mpNum[]? An array of real numbers.}
\end{mpFunctionsExtract}

\section{Min, Max, Median, Percentiles}

\begin{mpFunctionsExtract}
\mpWorksheetFunctionOneNotImplemented
{MIN? mpNum? the smallest number in an array, ignoring non-numerical values,}
{x? mpNum[]? An array of real numbers.}
\end{mpFunctionsExtract}

\begin{mpFunctionsExtract}
\mpWorksheetFunctionOneNotImplemented
{MINA? mpNum? the smallest number in an array; Text and FALSE in arguments are evaluated as zero; TRUE is evaluated as 1.}
{x? mpNumList? An array of real numbers.}
\end{mpFunctionsExtract}

\begin{mpFunctionsExtract}
\mpWorksheetFunctionOneNotImplemented
{MAX? mpNum? the largest number in an array, ignoring non-numerical values,}
{x? mpNum[]? An array of real numbers.}
\end{mpFunctionsExtract}

\begin{mpFunctionsExtract}
\mpWorksheetFunctionOneNotImplemented
{MAXA? mpNum? the largest number in an array; Text and FALSE in arguments are evaluated as zero; TRUE is evaluated as 1.}
{x? mpNumList? An array of real numbers.}
\end{mpFunctionsExtract}

\begin{mpFunctionsExtract}
\mpWorksheetFunctionOneNotImplemented
{MEDIAN? mpNum? the median in an array, ignoring non-numerical values.}
{x? mpNum[]? An array of real numbers.}
\end{mpFunctionsExtract}

\begin{mpFunctionsExtract}
\mpWorksheetFunctionOneNotImplemented
{MODE? mpNum? the most frequently occurring, or repetitive, value in an array or range of data.}
{x? mpNum[]? An array of real numbers.}
\end{mpFunctionsExtract}

\begin{mpFunctionsExtract}
\mpWorksheetFunctionOneNotImplemented
{MODE.SNGL? mpNum? the most frequently occurring, or repetitive, value in an array or range of data.}
{x? mpNum[]? An array of real numbers.}
\end{mpFunctionsExtract}

\begin{mpFunctionsExtract}
\mpWorksheetFunctionOneNotImplemented
{MODE.MULT? mpNum? a vertical array of the most frequently occurring, or repetitive values in an array or range of data.}
{x? mpNum[]? An array of real numbers.}
\end{mpFunctionsExtract}

\begin{mpFunctionsExtract}
\mpWorksheetFunctionTwoNotImplemented
{LARGE? mpNum? the $k^{th}$ largest value in a data set.}
{x? mpNum[]?  The array or range of data for which you want to determine the $k^{th}$ largest value.}
{k? mpNum? The position (from the largest) in the array or cell range of data to return.}
\end{mpFunctionsExtract}

\begin{mpFunctionsExtract}
\mpWorksheetFunctionTwoNotImplemented
{SMALL? mpNum? the $k^{th}$ smallest value in a data set.}
{x? mpNum[]?  The array or range of data for which you want to determine the $k^{th}$ smallest value.}
{k? mpNum? The position (from the smallest) in the array or cell range of data to return.}
\end{mpFunctionsExtract}

\begin{mpFunctionsExtract}
\mpWorksheetFunctionTwoNotImplemented
{PERCENTILE? mpNum? the k-th percentile of values  in a data set as a percentage (0..1, inclusive) of the data set.}
{x? mpNum[]?  The array or range of data with numeric values that defines relative standing.}
{k? mpNum? The percentile value in the range 0..1, inclusive.}
\end{mpFunctionsExtract}

\begin{mpFunctionsExtract}
\mpWorksheetFunctionTwoNotImplemented
{PERCENTILE.INC? mpNum? the k-th percentile of values  in a data set as a percentage (0..1, inclusive) of the data set.}
{x? mpNum[]?  The array or range of data with numeric values that defines relative standing.}
{k? mpNum? The percentile value in the range 0..1, inclusive.}
\end{mpFunctionsExtract}

\begin{mpFunctionsExtract}
\mpWorksheetFunctionTwoNotImplemented
{PERCENTILE.EXC? mpNum? the k-th percentile of values  in a data set as a percentage (0..1, exclusive) of the data set.}
{x? mpNum[]?  The array or range of data with numeric values that defines relative standing.}
{k? mpNum? The percentile value in the range 0..1, exclusive.}
\end{mpFunctionsExtract}

\begin{mpFunctionsExtract}
\mpWorksheetFunctionThreeNotImplemented
{PERCENTRANK? mpNum? the rank of a value in a data set as a percentage (0..1, inclusive) of the data set.}
{Data? mpNum[]?  The array or range of data for which you want to determine the $k^{th}$ smallest  value.}
{x? mpNum? The value for which you want to know the rank.}
{digits? mpNum? A value that identifies the number of significant digits for the returned percentage value. If omitted, three digits (0.xxx) are used.}
\end{mpFunctionsExtract}

\begin{mpFunctionsExtract}
\mpWorksheetFunctionThreeNotImplemented
{PERCENTRANK.INC? mpNum? the rank of a value in a data set as a percentage (0..1, inclusive) of the data set.}
{Data? mpNum[]?  The array or range of data for which you want to determine the $k^{th}$ smallest  value.}
{x? mpNum? The value for which you want to know the rank.}
{digits? mpNum? A value that identifies the number of significant digits for the returned percentage value. If omitted, three digits (0.xxx) are used.}
\end{mpFunctionsExtract}

\begin{mpFunctionsExtract}
\mpWorksheetFunctionThreeNotImplemented
{PERCENTRANK.EXC? mpNum? the rank of a value in a data set as a percentage (0..1, exclusive) of the data set.}
{Data? mpNum[]?  The array or range of data for which you want to determine the $k^{th}$ smallest  value.}
{x? mpNum? The value for which you want to know the rank.}
{digits? mpNum? A value that identifies the number of significant digits for the returned percentage value. If omitted, three digits (0.xxx) are used.}
\end{mpFunctionsExtract}

\begin{mpFunctionsExtract}
\mpWorksheetFunctionTwoNotImplemented
{QUARTILE? mpNum? the quartile of a data set, based on percentile values from 0..1, inclusive.}
{x? mpNum[]?  The array or cell range of numeric values for which you want the quartile value.}
{Quart? mpNum? Indicates which quartile to return.}
\end{mpFunctionsExtract}

\begin{mpFunctionsExtract}
\mpWorksheetFunctionTwoNotImplemented
{QUARTILE.INC? mpNum? quartile of a data set, based on percentile values from 0..1, inclusive.}
{x? mpNum[]?  The array or cell range of numeric values for which you want the quartile value.}
{Quart? mpNum? Indicates which quartile to return.}
\end{mpFunctionsExtract}

\begin{mpFunctionsExtract}
\mpWorksheetFunctionTwoNotImplemented
{QUARTILE.EXC? mpNum? quartile of a data set, based on percentile values from 0..1, exclusive.}
{x? mpNum[]?  The array or cell range of numeric values for which you want the quartile value.}
{Quart? mpNum? Indicates which quartile to return.}
\end{mpFunctionsExtract}

\begin{mpFunctionsExtract}
\mpWorksheetFunctionThreeNotImplemented
{RANK? mpNum? the rank of a number in a list of numbers.}
{x? mpNum? The number whose rank you want to find.}
{Data? mpNum[]?  The array or cell range of numeric values for which you want the rank value.}
{order? mpNum? A number specifying how to rank number.}
\end{mpFunctionsExtract}

\begin{mpFunctionsExtract}
\mpWorksheetFunctionThreeNotImplemented
{RANK.EQ? mpNum? the rank of a number in a list of numbers.}
{x? mpNum? The number whose rank you want to find.}
{Data? mpNum[]?  The array or cell range of numeric values for which you want the rank value.}
{order? mpNum? A number specifying how to rank number.}
\end{mpFunctionsExtract}

\begin{mpFunctionsExtract}
\mpWorksheetFunctionThreeNotImplemented
{RANK.AVG? mpNum? the (average) rank of a number in a list of numbers.}
{x? mpNum? The number whose rank you want to find.}
{Data? mpNum[]?  The array or cell range of numeric values for which you want the rank value.}
{order? mpNum? A number specifying how to rank number.}
\end{mpFunctionsExtract}

\begin{mpFunctionsExtract}
\mpWorksheetFunctionFourNotImplemented
{PROB? mpNum? the probability that values in a range are between two limits.}
{XRange? mpNum[]? The range of numeric values of $x$ with which there are associated probabilities.}
{ProbRange? mpNum[]?  A set of probabilities associated with values in \textsf{XRange}.}
{LowerLimit? mpNum? The lower bound on the value for which you want a probability.}
{UpperLimit? mpNum? The optional upper bound on the value for which you want a probability.}
\end{mpFunctionsExtract}

\section{Summary Tables of Aggregates}

\begin{mpFunctionsExtract}
\mpWorksheetFunctionTwoNotImplemented
{SUBTOTAL? mpNum? a subtotal in a list or database.}
{FunctionNum? Integer?  The number that specifies which function to use in calculating subtotals within a list.}
{Data? mpNumList? An array of real numbers.}
\end{mpFunctionsExtract}

\begin{mpFunctionsExtract}
\mpWorksheetFunctionFourNotImplemented
{AGGREGATE? mpNum? a subtotal in a list or database.}
{FunctionNum? Integer?  The number that specifies which function to use in calculating subtotals within a list.}
{Options? Integer?  A numerical value that determines which values to ignore in the evaluation range for the function}
{Data? mpNumList? An array of real numbers.}
{k? Integer?  A selection parameter required for the certain functions.}
\end{mpFunctionsExtract}

\section{Confidence intervals and tests}

\begin{mpFunctionsExtract}
\mpWorksheetFunctionThreeNotImplemented
{CONFIDENCE? mpNum? the confidence interval for a population mean.}
{Alpha? mpNum? The significance level used to compute the confidence level. The confidence level equals 100*(1 - alpha)\%, or in other words, an alpha of 0.05 indicates a 95 percent confidence level.
}
{SteDev? mpNum? The population standard deviation for the data range and is assumed to be known.}
{N? mpNum? The sample size.}
\end{mpFunctionsExtract}

\begin{mpFunctionsExtract}
\mpWorksheetFunctionThreeNotImplemented
{CONFIDENCE.NORM? mpNum? the confidence interval for a population mean.}
{Alpha? mpNum? The significance level used to compute the confidence level. The confidence level equals 100*(1 - alpha)\%, or in other words, an alpha of 0.05 indicates a 95 percent confidence level.
}
{SteDev? mpNum? The population standard deviation for the data range and is assumed to be known.}
{N? mpNum? The sample size.}
\end{mpFunctionsExtract}

\begin{mpFunctionsExtract}
\mpWorksheetFunctionThreeNotImplemented
{CONFIDENCE.T? mpNum? the confidence interval for a population mean.}
{Alpha? mpNum? The significance level used to compute the confidence level. The confidence level equals 100*(1 - alpha)\%, or in other words, an alpha of 0.05 indicates a 95 percent confidence level.
}
{SteDev? mpNum? The population standard deviation for the data range}
{N? mpNum? The sample size.}
\end{mpFunctionsExtract}

\begin{mpFunctionsExtract}
\mpWorksheetFunctionThreeNotImplemented
{ZTEST? mpNum? the one-tailed P-value of a z-test.}
{X? mpNum[]? The array or range of data against which to test Mean.}
{Mean? mpNum? The value to test.}
{Sigma? mpNum? The population (known) standard deviation. If omitted, the sample standard deviation is used.}
\end{mpFunctionsExtract}

\begin{mpFunctionsExtract}
\mpWorksheetFunctionThreeNotImplemented
{Z.TEST? mpNum? the one-tailed P-value of a z-test.}
{X? mpNum[]? The array or range of data against which to test Mean.}
{Mean? mpNum? The value to test.}
{Sigma? mpNum? The population (known) standard deviation. If omitted, the sample standard deviation is used.}
\end{mpFunctionsExtract}

\begin{mpFunctionsExtract}
\mpWorksheetFunctionFourNotImplemented
{TTEST? mpNum? the probability associated with a Student's t-Test.}
{X? mpNum[]? the first data set.}
{Y? mpNum[]? the second data set.}
{Tails? Integer?  specifies the number of distribution tails. If tails = 1, TTEST uses the one-tailed distribution. If tails = 2, TTEST uses the two-tailed distribution.}
{Type? mpNum? the kind of t-Test to perform. type = 1 paired, type = 2 unpaired, equal variance (homoscedastic), type = 3 unpaired, unequal variance (heteroscedastic).}
\end{mpFunctionsExtract}

\begin{mpFunctionsExtract}
\mpWorksheetFunctionFourNotImplemented
{T.TEST? mpNum? the probability associated with a Student's t-Test.}
{X? mpNum[]? the first data set.}
{Y? mpNum[]? the second data set.}
{Tails? Integer?  specifies the number of distribution tails. If tails = 1, TTEST uses the one-tailed distribution. If tails = 2, TTEST uses the two-tailed distribution.}
{Type? mpNum? the kind of t-Test to perform. type = 1 paired, type = 2 unpaired, equal variance (homoscedastic), type = 3 unpaired, unequal variance (heteroscedastic).}
\end{mpFunctionsExtract}

\begin{mpFunctionsExtract}
\mpWorksheetFunctionTwoNotImplemented
{FTEST? mpNum? the two-tailed probability that the variances in array1 and array2 are not significantly different.}
{X? mpNum[]? the first data set.}
{Y? mpNum[]? the second data set.}
\end{mpFunctionsExtract}

\begin{mpFunctionsExtract}
\mpWorksheetFunctionTwoNotImplemented
{F.TEST? mpNum? the two-tailed probability that the variances in array1 and array2 are not significantly different.}
{X? mpNum[]? the first data set.}
{Y? mpNum[]? the second data set.}
\end{mpFunctionsExtract}

\begin{mpFunctionsExtract}
\mpFunctionTwoNotImplemented
{ANOVA1? mpNum? the two-tailed probability that the variances in array1 and array2 are not significantly different.}
{X? mpNum[]? the first data set.}
{Y? mpNum[]? the second data set.}
\end{mpFunctionsExtract}

\begin{mpFunctionsExtract}
\mpFunctionTwoNotImplemented
{ANOVA2? mpNum? the two-tailed probability that the variances in array1 and array2 are not significantly different.}
{X? mpNum[]? the first data set.}
{Y? mpNum[]? the second data set.}
\end{mpFunctionsExtract}

\begin{mpFunctionsExtract}
\mpWorksheetFunctionTwoNotImplemented
{CHITEST? mpNum? the probability that a value of the $\chi^2$ statistic at least as high as the value calculated by the above formula could have happened by chance under the assumption of independence.}
{A? mpNum[]? The range of data that contains observations to test against expected values.}
{E? mpNum[]? The range of data that contains the ratio of the product of row totals and column totals to the grand total.}
\end{mpFunctionsExtract}

\begin{mpFunctionsExtract}
\mpWorksheetFunctionTwoNotImplemented
{CHISQ.TEST? mpNum? the probability that a value of the $\chi^2$ statistic at least as high as the value calculated by the above formula could have happened by chance under the assumption of independence.}
{A? mpNum[]? The range of data that contains observations to test against expected values.}
{E? mpNum[]? The range of data that contains the ratio of the product of row totals and column totals to the grand total.}
\end{mpFunctionsExtract}

\section{Covariance and Correlation}

\begin{mpFunctionsExtract}
\mpWorksheetFunctionTwoNotImplemented
{COVAR? mpNum? the sample covariance $\textsf{cov}(X,Y)$}
{X? mpNum[]? An array of real numbers.}
{Y? mpNum[]? An array of real numbers.}
\end{mpFunctionsExtract}

\begin{mpFunctionsExtract}
\mpWorksheetFunctionTwoNotImplemented
{COVARIANCE.S? mpNum? the sample covariance $\textsf{cov}(X,Y)$}
{X? mpNum[]? An array of real numbers.}
{Y? mpNum[]? An array of real numbers.}
\end{mpFunctionsExtract}

\begin{mpFunctionsExtract}
\mpWorksheetFunctionTwoNotImplemented
{COVARIANCE.P? mpNum? the sample covariance $\textsf{cov}(X,Y)$}
{X? mpNum[]? An array of real numbers.}
{Y? mpNum[]? An array of real numbers.}
\end{mpFunctionsExtract}

\begin{mpFunctionsExtract}
\mpWorksheetFunctionTwoNotImplemented
{CORREL? mpNum? the Pearson product moment correlation coefficient  $r = \textsf{corr}(X,Y)$}
{X? mpNum[]? An array of real numbers.}
{Y? mpNum[]? An array of real numbers.}
\end{mpFunctionsExtract}

\begin{mpFunctionsExtract}
\mpWorksheetFunctionTwoNotImplemented
{PEARSON? mpNum? the Pearson product moment correlation coefficient  $r = \textsf{corr}(X,Y)$}
{X? mpNum[]? An array of real numbers.}
{Y? mpNum[]? An array of real numbers.}
\end{mpFunctionsExtract}

\begin{mpFunctionsExtract}
\mpWorksheetFunctionTwoNotImplemented
{RSQ? mpNum? $r^2$, the square of the Pearson product moment correlation coefficient $r$, with $r = \textsf{corr}(X,Y)$}
{X? mpNum[]? An array of real numbers.}
{Y? mpNum[]? An array of real numbers.}
\end{mpFunctionsExtract}

\begin{mpFunctionsExtract}
\mpWorksheetFunctionOneNotImplemented
{FISHER? mpNum? Fisher's z-transform}
{X? mpNum? Areal numbers}
\end{mpFunctionsExtract}

\begin{mpFunctionsExtract}
\mpWorksheetFunctionOneNotImplemented
{FISHERINV? mpNum? Fisher's inverse z-transform}
{X? mpNum? Areal numbers}
\end{mpFunctionsExtract}

\section{Linear Regression}

\begin{mpFunctionsExtract}
\mpWorksheetFunctionTwoNotImplemented
{INTERCEPT? mpNum? the point at which a line will intersect the y-axis by using linear regression.}
{X? mpNum[]? An array of real numbers.}
{Y? mpNum[]? An array of real numbers.}
\end{mpFunctionsExtract}

\begin{mpFunctionsExtract}
\mpWorksheetFunctionTwoNotImplemented
{SLOPE? mpNum? the slope of the linear regression line through data points in $X$ and $Y$.}
{X? mpNum[]? An array of real numbers.}
{Y? mpNum[]? An array of real numbers.}
\end{mpFunctionsExtract}

\begin{mpFunctionsExtract}
\mpWorksheetFunctionTwoNotImplemented
{FORECAST? mpNum? an $y_0$-value for a given $x_0$-value, using linear regression.}
{X? mpNum[]? An array of real numbers.}
{Y? mpNum[]? An array of real numbers.}
\end{mpFunctionsExtract}

\begin{mpFunctionsExtract}
\mpWorksheetFunctionTwoNotImplemented
{STEYX? mpNum? the standard error of the predicted $y$-value for each $x$ in the regression.}
{X? mpNum[]? An array of real numbers.}
{Y? mpNum[]? An array of real numbers.}
\end{mpFunctionsExtract}

\section{Database related functions}

\begin{mpFunctionsExtract}
\mpWorksheetFunctionThreeNotImplemented
{DGET? mpNum? a single value from a column of a list or database that matches conditions that you specify.}
{Table? mpNum[]? An array of real numbers, using Strings as headers.}
{Field? String? Indicates which column is used in the function.}
{Criteria? String? A String containing the criteria.}
\end{mpFunctionsExtract}

\begin{mpFunctionsExtract}
\mpWorksheetFunctionThreeNotImplemented
{DPRODUCT? mpNum? the product of the values from a column of a list or database that matches conditions that you specify.}
{Table? mpNum[]? An array of real numbers, using Strings as headers.}
{Field? String? Indicates which column is used in the function.}
{Criteria? String? A String containing the criteria.}
\end{mpFunctionsExtract}

\begin{mpFunctionsExtract}
\mpWorksheetFunctionThreeNotImplemented
{DCOUNT? mpNum? the number of cells that contain numbers in a field (column) of records in a list or database that match conditions that you specify.}
{Table? mpNum[]? An array of real numbers, using Strings as headers.}
{Field? String? Indicates which column is used in the function.}
{Criteria? String? A String containing the criteria.}
\end{mpFunctionsExtract}

\begin{mpFunctionsExtract}
\mpWorksheetFunctionThreeNotImplemented
{DCOUNTA? mpNum? the number of nonblank cells that contain numbers in a field (column) of records in a list or database that match conditions that you specify.}
{Table? mpNum[]? An array of real numbers, using Strings as headers.}
{Field? String? Indicates which column is used in the function.}
{Criteria? String? A String containing the criteria.}
\end{mpFunctionsExtract}

\begin{mpFunctionsExtract}
\mpWorksheetFunctionThreeNotImplemented
{DSUM? mpNum? the sum of cells that contain numbers in a field (column) of records in a list or database that match conditions that you specify.}
{Table? mpNum[]? An array of real numbers, using Strings as headers.}
{Field? String? Indicates which column is used in the function.}
{Criteria? String? A String containing the criteria.}
\end{mpFunctionsExtract}

\begin{mpFunctionsExtract}
\mpWorksheetFunctionThreeNotImplemented
{DAVERAGE? mpNum? the arithmetic mean of cells that contain numbers in a field (column) of records in a list or database that match conditions that you specify.}
{Table? mpNum[]? An array of real numbers, using Strings as headers.}
{Field? String? Indicates which column is used in the function.}
{Criteria? String? A String containing the criteria.}
\end{mpFunctionsExtract}

\begin{mpFunctionsExtract}
\mpWorksheetFunctionThreeNotImplemented
{DVAR? mpNum? the sample variance of cells that contain numbers in a field (column) of records in a list or database that match conditions that you specify.}
{Table? mpNum[]? An array of real numbers, using Strings as headers.}
{Field? String? Indicates which column is used in the function.}
{Criteria? String? A String containing the criteria.}
\end{mpFunctionsExtract}

\begin{mpFunctionsExtract}
\mpWorksheetFunctionThreeNotImplemented
{DVARP? mpNum? the population variance of cells that contain numbers in a field (column) of records in a list or database that match conditions that you specify.}
{Table? mpNum[]? An array of real numbers, using Strings as headers.}
{Field? String? Indicates which column is used in the function.}
{Criteria? String? A String containing the criteria.}
\end{mpFunctionsExtract}

\begin{mpFunctionsExtract}
\mpWorksheetFunctionThreeNotImplemented
{DSTDEV? mpNum? the sample standard deviation of cells that contain numbers in a field (column) of records in a list or database that match conditions that you specify.}
{Table? mpNum[]? An array of real numbers, using Strings as headers.}
{Field? String? Indicates which column is used in the function.}
{Criteria? String? A String containing the criteria.}
\end{mpFunctionsExtract}

\begin{mpFunctionsExtract}
\mpWorksheetFunctionThreeNotImplemented
{DSTDEVP? mpNum? the population standard deviation of cells that contain numbers in a field (column) of records in a list or database that match conditions that you specify.}
{Table? mpNum[]? An array of real numbers, using Strings as headers.}
{Field? String? Indicates which column is used in the function.}
{Criteria? String? A String containing the criteria.}
\end{mpFunctionsExtract}

\begin{mpFunctionsExtract}
\mpWorksheetFunctionThreeNotImplemented
{DMIN? mpNum? the smallest number in a field (column) of records in a list or database that match conditions that you specify.}
{Table? mpNum[]? An array of real numbers, using Strings as headers.}
{Field? String? Indicates which column is used in the function.}
{Criteria? String? A String containing the criteria.}
\end{mpFunctionsExtract}

\begin{mpFunctionsExtract}
\mpWorksheetFunctionThreeNotImplemented
{DMAX? mpNum? the largest number in a field (column) of records in a list or database that match conditions that you specify.}
{Table? mpNum[]? An array of real numbers, using Strings as headers.}
{Field? String? Indicates which column is used in the function.}
{Criteria? String? A String containing the criteria.}
\end{mpFunctionsExtract}

\chapter{Factorials and gamma functions}

\section{Factorials}

\begin{mpFunctionsExtract}
\mpFunctionOne
{Factorial? mpNum? the factorial, $x!$.}
{z? mpNum? A real or complex number.}
\end{mpFunctionsExtract}

\begin{mpFunctionsExtract}
\mpFunctionOne
{fac? mpNum? the factorial, $x!$.}
{z? mpNum? A real or complex number.}
\end{mpFunctionsExtract}

\begin{mpFunctionsExtract}
\mpFunctionOne
{fac2? mpNum? the double factorial $x!!$.}
{z? mpNum? A real or complex number.}
\end{mpFunctionsExtract}

\section{Binomial coefficient}

\begin{mpFunctionsExtract}
\mpFunctionTwo
{binomial? mpNum? the binomial coefficient.}
{n? mpNum? A real or complex number.}
{k? mpNum? A real or complex number.}
\end{mpFunctionsExtract}

\section{Pochhammer symbol, Rising and falling factorials}

\begin{mpFunctionsExtract}
\mpFunctionTwoNotImplemented
{RelativePochhammerMpMath? mpNum? the relative Pochhammer symbol.}
{a? mpNum? An integer.}
{x? mpNum? An integer.}
\end{mpFunctionsExtract}

\begin{mpFunctionsExtract}
\mpFunctionTwo
{rf? mpNum? the rising factorial.}
{x? mpNum? A real or complex number.}
{n? mpNum? A real or complex number.}
\end{mpFunctionsExtract}

\begin{mpFunctionsExtract}
\mpFunctionTwo
{ff? mpNum? the falling factorial.}
{x? mpNum? A real or complex number.}
{n? mpNum? A real or complex number.}
\end{mpFunctionsExtract}

\section{Super- and hyperfactorials}

\begin{mpFunctionsExtract}
\mpFunctionOne
{superfac? mpNum? the superfactorial.}
{z? mpNum? A real or complex number.}
\end{mpFunctionsExtract}

\begin{mpFunctionsExtract}
\mpFunctionOne
{hyperfac? mpNum? the hyperfactorial.}
{z? mpNum? A real or complex number.}
\end{mpFunctionsExtract}

\begin{mpFunctionsExtract}
\mpFunctionOne
{barnesg? mpNum? the Barnes G-function.}
{z? mpNum? A real or complex number.}
\end{mpFunctionsExtract}

\section{Gamma functions}

\begin{mpFunctionsExtract}
\mpFunctionOne
{gamma? mpNum? the gamma function, $\Gamma(x)$.}
{z? mpNum? A real or complex number.}
\end{mpFunctionsExtract}

\begin{mpFunctionsExtract}
\mpFunctionOne
{rgamma? mpNum? the reciprocal of the gamma function, $1/\Gamma(z)$.}
{z? mpNum? A real or complex number.}
\end{mpFunctionsExtract}

\begin{mpFunctionsExtract}
\mpFunctionTwo
{gammaprod? mpNum?  the  product / quotient of gamma functions.}
{a? mpNum? A real or complex iterables.}
{b? mpNum? A real or complex iterables.}
\end{mpFunctionsExtract}

\begin{mpFunctionsExtract}
\mpFunctionOne
{loggamma? mpNum? the principal branch of the log-gamma function, $\ln\Gamma(z)$.}
{z? mpNum? A real or complex number.}
\end{mpFunctionsExtract}

\begin{mpFunctionsExtract}
\mpFunctionFour
{gammainc? mpNum? the incomplete gamma function with integration limits $[a, b]$.}
{z? mpNum? A real or complex number.}
{a? mpNum? A real or complex number (default = 0).}
{b? mpNum? A real or complex number (default = inf).}
{Keywords? String?  regularized=False.}
\end{mpFunctionsExtract}

\begin{mpFunctionsExtract}
\mpFunctionTwoNotImplemented
{GammaPDerivativeMpMath? mpNum? the partial derivative with respect to $x$ of the incomplete gamma function $P(a,x)$.}
{a? mpNum? A real number.}
{x? mpNum? A real number.}
\end{mpFunctionsExtract}

\begin{mpFunctionsExtract}
\mpFunctionTwoNotImplemented
{GammaPMpMath? mpNum? the normalised incomplete gamma function $P(a,x)$.}
{a? mpNum? A real number.}
{x? mpNum? A real number.}
\end{mpFunctionsExtract}

\begin{mpFunctionsExtract}
\mpFunctionTwoNotImplemented
{GammaQMpMath? mpNum? the normalised incomplete gamma function $Q(a,x)$.}
{a? mpNum? A real number.}
{x? mpNum? A real number.}
\end{mpFunctionsExtract}

\begin{mpFunctionsExtract}
\mpFunctionTwoNotImplemented
{NonNormalisedGammaPMpMath? mpNum? the non-normalised incomplete gamma function $\Gamma(a,x)$.}
{a? mpNum? A real number.}
{x? mpNum? A real number.}
\end{mpFunctionsExtract}

\begin{mpFunctionsExtract}
\mpFunctionTwoNotImplemented
{NonNormalisedGammaQMpMath? mpNum? the non-normalised incomplete gamma function $\gamma(a,x)$.}
{a? mpNum? A real number.}
{x? mpNum? A real number.}
\end{mpFunctionsExtract}

\begin{mpFunctionsExtract}
\mpFunctionTwoNotImplemented
{TricomiGammaMpMath? mpNum? Tricomi's entire incomplete gamma function $\gamma^*(a,x)$.}
{a? mpNum? A real number.}
{x? mpNum? A real number.}
\end{mpFunctionsExtract}

\begin{mpFunctionsExtract}
\mpFunctionTwoNotImplemented
{GammaPinvMpMath? mpNum? the inverse of the normalised incomplete gamma function $P(a,x)$.}
{a? mpNum? A real number.}
{p? mpNum? A real number.}
\end{mpFunctionsExtract}

\begin{mpFunctionsExtract}
\mpFunctionTwoNotImplemented
{GammaQinvMpMath? mpNum? the inverse of the normalised incomplete gamma function $Q(a,x)$.}
{a? mpNum? A real number.}
{q? mpNum? A real number.}
\end{mpFunctionsExtract}

\section{Polygamma functions and harmonic numbers}

\begin{mpFunctionsExtract}
\mpFunctionTwo
{polygamma? mpNum? the polygamma function of order $m$ of $z$, $\psi^{(m)}(z)$.}
{m? mpNum? A real or complex number.}
{z? mpNum? A real or complex number.}
\end{mpFunctionsExtract}

\begin{mpFunctionsExtract}
\mpFunctionTwo
{psi? mpNum? the polygamma function of order $m$ of $z$, $\psi^{(m)}(z)$.}
{m? mpNum? A real or complex number.}
{z? mpNum? A real or complex number.}
\end{mpFunctionsExtract}

\begin{mpFunctionsExtract}
\mpFunctionOne
{digamma? mpNum? the digamma function.}
{z? mpNum? A real or complex number.}
\end{mpFunctionsExtract}

\begin{mpFunctionsExtract}
\mpFunctionOne
{harmonic? mpNum? a floating-point approximation of the $n$-th harmonic number $H(n)$.}
{n? mpNum? An  real or complex number.}
\end{mpFunctionsExtract}

\section{Beta Functions}

\begin{mpFunctionsExtract}
\mpFunctionTwo
{beta? mpNum? the beta function, $B(x,y)=\Gamma(x) \Gamma(y)/\Gamma(x+y)$.}
{x? mpNum? A real or complex number.}
{y? mpNum? A real or complex number.}
\end{mpFunctionsExtract}

\begin{mpFunctionsExtract}
\mpFunctionTwoNotImplemented
{LnBetaMpMath? mpNum? the logarithm of the beta function $\ln B(a,b)|$ with $a,b \neq 0,-1,-2,\ldots$.}
{a? mpNum? A real number.}
{b? mpNum? A real number.}
\end{mpFunctionsExtract}

\begin{mpFunctionsExtract}
\mpFunctionFive
{betainc? mpNum? the generalized incomplete beta function.}
{a? mpNum? A real or complex number.}
{b? mpNum? A real or complex number.}
{x1? mpNum? A real or complex number (default = 0).}
{x2? mpNum? A real or complex number (default = 1).}
{Keywords? String?  regularized=False.}
\end{mpFunctionsExtract}

\begin{mpFunctionsExtract}
\mpFunctionThreeNotImplemented
{IBetaNonNormalizedMpMath? mpNum? the non-normalised incomplete beta function.}
{a? mpNum? A real number.}
{b? mpNum? A real number.}
{x? mpNum? A real number.}
\end{mpFunctionsExtract}

\begin{mpFunctionsExtract}
\mpFunctionThreeNotImplemented
{IBetaMpMath? mpNum? the normalised incomplete beta function.}
{a? mpNum? A real number.}
{b? mpNum? A real number.}
{x? mpNum? A real number.}
\end{mpFunctionsExtract}

\chapter{Exponential integrals and error functions}

\section{Exponential integrals}

\begin{mpFunctionsExtract}
\mpFunctionOne
{ei? mpNum? the exponential integral.}
{z? mpNum? A real or complex number.}
\end{mpFunctionsExtract}

\begin{mpFunctionsExtract}
\mpFunctionOne
{e1? mpNum? the exponential integral $\text{E}_1(x)$.}
{z? mpNum? A real or complex number.}
\end{mpFunctionsExtract}

\begin{mpFunctionsExtract}
\mpFunctionTwo
{expint? mpNum? the generalized exponential integral or En-function.}
{n? mpNum? A real or complex number.}
{z? mpNum? A real or complex number.}
\end{mpFunctionsExtract}

\begin{mpFunctionsExtract}
\mpFunctionTwoNotImplemented
{GeneralizedExponentialIntegralEpMpMath? mpNum? the generalized exponential integrals $\text{E}_n(x)$ of real order $p$.}
{x? mpNum? A real number.}
{p? mpNum? A real number.}
\end{mpFunctionsExtract}

\section{Logarithmic integral}

\begin{mpFunctionsExtract}
\mpFunctionOne
{li? mpNum? the logarithmic integral.}
{z? mpNum? A real or complex number.}
\end{mpFunctionsExtract}

\section{Trigonometric integrals}

\begin{mpFunctionsExtract}
\mpFunctionOne
{ci? mpNum? the cosine integral.}
{z? mpNum? A real or complex number.}
\end{mpFunctionsExtract}

\begin{mpFunctionsExtract}
\mpFunctionOne
{si? mpNum? the sine integral.}
{z? mpNum? A real or complex number.}
\end{mpFunctionsExtract}

\section{Hyperbolic integrals}

\begin{mpFunctionsExtract}
\mpFunctionOne
{chi? mpNum? the hyperbolic cosine integral.}
{z? mpNum? A real or complex number.}
\end{mpFunctionsExtract}

\begin{mpFunctionsExtract}
\mpFunctionOne
{shi? mpNum? the hyperbolic sine integral.}
{z? mpNum? A real or complex number.}
\end{mpFunctionsExtract}

\section{Error functions}

\begin{mpFunctionsExtract}
\mpFunctionOne
{erf? mpNum? the error function, $\text{erf}(x)$.}
{z? mpNum? A real or complex number.}
\end{mpFunctionsExtract}

\begin{mpFunctionsExtract}
\mpFunctionOne
{erfc? mpNum? the complementary error function, $\text{erfc}(x)=1-\text{erf}(x)$.}
{z? mpNum? A real or complex number.}
\end{mpFunctionsExtract}

\begin{mpFunctionsExtract}
\mpFunctionOne
{erfi? mpNum? the imaginary error function, $\text{erfi}(x)$.}
{z? mpNum? A real or complex number.}
\end{mpFunctionsExtract}

\begin{mpFunctionsExtract}
\mpFunctionOne
{erfinv? mpNum? the inverse error function, $\text{erfinv}(x)$.}
{x? mpNum? A real number.}
\end{mpFunctionsExtract}

\section{The normal distribution}

\begin{mpFunctionsExtract}
\mpFunctionThree
{npdf? mpNum? the normal probability density function.}
{x? mpNum? A real number.}
{mu? mpNum? A real number.}
{sigma? mpNum? A real number.}
\end{mpFunctionsExtract}

\begin{mpFunctionsExtract}
\mpFunctionThree
{ncdf? mpNum? the normal cumulative distribution function.}
{x? mpNum? A real number.}
{mu? mpNum? A real number.}
{sigma? mpNum? A real number.}
\end{mpFunctionsExtract}

\section{Fresnel integrals}

\begin{mpFunctionsExtract}
\mpFunctionOne
{fresnels? mpNum? the Fresnel sine integral.}
{z? mpNum? A real or complex number.}
\end{mpFunctionsExtract}

\begin{mpFunctionsExtract}
\mpFunctionOne
{fresnelc? mpNum? the Fresnel cosine integral.}
{z? mpNum? A real or complex number.}
\end{mpFunctionsExtract}

\section{Other Special Functions}

\begin{mpFunctionsExtract}
\mpFunctionTwo
{lambertw? mpNum? the Lambert W function.}
{z? mpNum? A real or complex number.}
{Keywords? String? k=0.}
\end{mpFunctionsExtract}

\begin{mpFunctionsExtract}
\mpFunctionTwo
{agm? mpNum? the arithmetic-geometric mean of $a$ and $b$.}
{a? mpNum? A real or complex number.}
{b? mpNum? A real or complex number.}
\end{mpFunctionsExtract}

\chapter{Bessel functions and related functions}

\section{Bessel functions}

\begin{mpFunctionsExtract}
\mpFunctionTwoNotImplemented
{BesselIeMpMath? mpNum? $I_{\nu, e}(x) = I_{\nu}(x) \exp(-|x|)$, the exponentially scaled modified Bessel function $I_{\nu}(z)$ of the first kind of order $\nu$, $x \geq 0$ if $\nu$ is not an integer.}
{x? mpNum? A real number.}
{$\nu$? mpNum? A real number.}
\end{mpFunctionsExtract}

\begin{mpFunctionsExtract}
\mpFunctionTwoNotImplemented
{BesselKeMpMath? mpNum? $K_{\nu, e}(x) = K_{\nu}(x) \exp(x)$, the exponentially scaled modified Bessel function $K_{\nu}(z)$ of the first kind of order $\nu$, $x > 0$.}
{x? mpNum? A real number.}
{$\nu$? mpNum? A real number.}
\end{mpFunctionsExtract}

\begin{mpFunctionsExtract}
\mpFunctionThree
{besselj? mpNum? the Bessel function of the first kind $J_n(x)$.}
{n? mpNum? A real or complex number.}
{x? mpNum? A real or complex number.}
{Keywords? String? derivative=0.}
\end{mpFunctionsExtract}

\begin{mpFunctionsExtract}
\mpFunctionOne
{j0? mpNum? the Bessel function $J_0(x)$.}
{x? mpNum? A real or complex number.}
\end{mpFunctionsExtract}

\begin{mpFunctionsExtract}
\mpFunctionOne
{j1? mpNum? the Bessel function $J_1(x)$.}
{x? mpNum? A real or complex number.}
\end{mpFunctionsExtract}

\begin{mpFunctionsExtract}
\mpFunctionThree
{bessely? mpNum? the Bessel function of the second kind $Y_n(x)$.}
{n? mpNum? A real or complex number.}
{x? mpNum? A real or complex number.}
{Keywords? String? derivative=0.}
\end{mpFunctionsExtract}

\begin{mpFunctionsExtract}
\mpFunctionThree
{besseli? mpNum? the modified Bessel function of the first kind $J_n(x)$.}
{n? mpNum? A real or complex number.}
{x? mpNum? A real or complex number.}
{Keywords? String? derivative=0.}
\end{mpFunctionsExtract}

\begin{mpFunctionsExtract}
\mpFunctionTwo
{besselk? mpNum? the modified Bessel function of the second kind $K_n(x)$.}
{n? mpNum? A real or complex number.}
{x? mpNum? A real or complex number.}
\end{mpFunctionsExtract}

\section{Bessel function zeros}

\begin{mpFunctionsExtract}
\mpFunctionThree
{besseljzero? mpNum? the m-th positive zero of the Bessel function of the first kind}
{v? mpNum? A real or complex number.}
{m? mpNum? A real or complex number.}
{Keywords? String? derivative=0.}
\end{mpFunctionsExtract}

\begin{mpFunctionsExtract}
\mpFunctionThree
{besselyzero? mpNum? the m-th positive zero of the Bessel function of the second kind}
{v? mpNum? A real or complex number.}
{m? mpNum? A real or complex number.}
{Keywords? String? derivative=0.}
\end{mpFunctionsExtract}

\section{Hankel functions}

\begin{mpFunctionsExtract}
\mpFunctionTwo
{hankel1? mpNum? the Hankel function of the first kind}
{n? mpNum? A real or complex number.}
{x? mpNum? A real or complex number.}
\end{mpFunctionsExtract}

\begin{mpFunctionsExtract}
\mpFunctionTwo
{hankel2? mpNum? the Hankel function of the second kind}
{n? mpNum? A real or complex number.}
{x? mpNum? A real or complex number.}
\end{mpFunctionsExtract}

\section{Kelvin functions}

\begin{mpFunctionsExtract}
\mpFunctionTwo
{ber? mpNum? the Kelvin function ber}
{n? mpNum? A real or complex number.}
{z? mpNum? A real or complex number.}
\end{mpFunctionsExtract}

\begin{mpFunctionsExtract}
\mpFunctionTwo
{bei? mpNum? the Kelvin function bei}
{n? mpNum? A real or complex number.}
{z? mpNum? A real or complex number.}
\end{mpFunctionsExtract}

\begin{mpFunctionsExtract}
\mpFunctionTwo
{ker? mpNum? the Kelvin function ker}
{n? mpNum? A real or complex number.}
{z? mpNum? A real or complex number.}
\end{mpFunctionsExtract}

\begin{mpFunctionsExtract}
\mpFunctionTwo
{kei? mpNum? the Kelvin function kei}
{n? mpNum? A real or complex number.}
{z? mpNum? A real or complex number.}
\end{mpFunctionsExtract}

\section{Struve Functions}

\begin{mpFunctionsExtract}
\mpFunctionTwo
{struveh? mpNum? the Struve function H}
{n? mpNum? A real or complex number.}
{z? mpNum? A real or complex number.}
\end{mpFunctionsExtract}

\begin{mpFunctionsExtract}
\mpFunctionTwo
{struvel? mpNum? the modified Struve function L}
{n? mpNum? A real or complex number.}
{z? mpNum? A real or complex number.}
\end{mpFunctionsExtract}

\section{Anger-Weber functions}

\begin{mpFunctionsExtract}
\mpFunctionTwo
{angerj? mpNum? the Anger function J}
{v? mpNum? A real or complex number.}
{z? mpNum? A real or complex number.}
\end{mpFunctionsExtract}

\begin{mpFunctionsExtract}
\mpFunctionTwo
{webere? mpNum? the Weber function E}
{v? mpNum? A real or complex number.}
{z? mpNum? A real or complex number.}
\end{mpFunctionsExtract}

\section{Lommel functions}

\begin{mpFunctionsExtract}
\mpFunctionThree
{lommels1? mpNum? the First Lommel functions s}
{u? mpNum? A real or complex number.}
{v? mpNum? A real or complex number.}
{z? mpNum? A real or complex number.}
\end{mpFunctionsExtract}

\begin{mpFunctionsExtract}
\mpFunctionThree
{lommels2? mpNum? the Second Lommel functions S}
{u? mpNum? A real or complex number.}
{v? mpNum? A real or complex number.}
{z? mpNum? A real or complex number.}
\end{mpFunctionsExtract}

\section{Airy and Scorer functions}

\begin{mpFunctionsExtract}
\mpFunctionTwo
{airyai? mpNum? the Airy function Ai}
{z? mpNum? A real or complex number.}
{Keywords? String? derivative=0.}
\end{mpFunctionsExtract}

\begin{mpFunctionsExtract}
\mpFunctionTwo
{airybi? mpNum? the Airy function Bi}
{z? mpNum? A real or complex number.}
{Keywords? String? derivative=0.}
\end{mpFunctionsExtract}

\begin{mpFunctionsExtract}
\mpFunctionTwo
{airyaizero? mpNum? the $k$-th zero of the Airy Ai-function}
{k? mpNum? An integer.}
{Keywords? String? derivative=0.}
\end{mpFunctionsExtract}

\begin{mpFunctionsExtract}
\mpFunctionTwo
{airybizero? mpNum? the $k$-th zero of the Airy Bi-function}
{k? mpNum? An integer.}
{Keywords? String? derivative=0, complex=0.}
\end{mpFunctionsExtract}

\begin{mpFunctionsExtract}
\mpFunctionOne
{scorergi? mpNum? the Scorer function Gi}
{z? mpNum? A real or complex number.}
\end{mpFunctionsExtract}

\begin{mpFunctionsExtract}
\mpFunctionOne
{scorerhi? mpNum? the Scorer function Gi}
{z? mpNum? A real or complex number.}
\end{mpFunctionsExtract}

\section{Coulomb wave functions}

\begin{mpFunctionsExtract}
\mpFunctionThree
{coulombf? mpNum? the regular Coulomb wave function}
{l? mpNum? A real or complex number.}
{eta? mpNum? A real or complex number.}
{z? mpNum? A real or complex number.}
\end{mpFunctionsExtract}

\begin{mpFunctionsExtract}
\mpFunctionThree
{coulombg? mpNum? the irregular Coulomb wave function}
{l? mpNum? A real or complex number.}
{eta? mpNum? A real or complex number.}
{z? mpNum? A real or complex number.}
\end{mpFunctionsExtract}

\begin{mpFunctionsExtract}
\mpFunctionTwo
{coulombc? mpNum? the normalizing Gamow constant for Coulomb wave functions}
{l? mpNum? A real or complex number.}
{eta? mpNum? A real or complex number.}
\end{mpFunctionsExtract}

\section{Parabolic cylinder functions}

\begin{mpFunctionsExtract}
\mpFunctionTwo
{pcfd? mpNum? the parabolic cylinder function D}
{n? mpNum? A real or complex number.}
{z? mpNum? A real or complex number.}
\end{mpFunctionsExtract}

\begin{mpFunctionsExtract}
\mpFunctionTwo
{pcfu? mpNum? the parabolic cylinder function U}
{a? mpNum? A real or complex number.}
{z? mpNum? A real or complex number.}
\end{mpFunctionsExtract}

\begin{mpFunctionsExtract}
\mpFunctionTwo
{pcfv? mpNum? the parabolic cylinder function V}
{a? mpNum? A real or complex number.}
{z? mpNum? A real or complex number.}
\end{mpFunctionsExtract}

\begin{mpFunctionsExtract}
\mpFunctionTwo
{pcfw? mpNum? Computes the parabolic cylinder function W}
{a? mpNum? A real or complex number.}
{z? mpNum? A real or complex number.}
\end{mpFunctionsExtract}

\chapter{Orthogonal polynomials}

\section{Legendre functions}

\begin{mpFunctionsExtract}
\mpFunctionTwo
{legendre? mpNum? the Legendre polynomial $P_n(x)$}
{n? mpNum? A real or complex number.}
{x? mpNum? A real or complex number.}
\end{mpFunctionsExtract}

\begin{mpFunctionsExtract}
\mpFunctionFour
{legenp? mpNum? the (associated) Legendre function of the first kind of degree $n$ and order $m$, $P_n^m(z)$.}
{n? mpNum? A real or complex number.}
{m? mpNum? A real or complex number.}
{z? mpNum? A real or complex number.}
{Keywords? String? type=2.}
\end{mpFunctionsExtract}

\begin{mpFunctionsExtract}
\mpFunctionFour
{legenq? mpNum? the (associated) Legendre function of the second kind of degree $n$ and order $m$, $Q_n^m(z)$.}
{n? mpNum? A real or complex number.}
{m? mpNum? A real or complex number.}
{z? mpNum? A real or complex number.}
{Keywords? String? type=2.}
\end{mpFunctionsExtract}

\begin{mpFunctionsExtract}
\mpFunctionFour
{spherharm? mpNum? the spherical harmonic $Y_l^m(\theta,\phi)$}
{l? mpNum? A real or complex number.}
{m? mpNum? A real or complex number.}
{theta? mpNum? A real or complex number.}
{phi? mpNum? A real or complex number.}
\end{mpFunctionsExtract}

\section{Chebyshev polynomials}

\begin{mpFunctionsExtract}
\mpFunctionTwo
{chebyt? mpNum? the Chebyshev polynomial of the first kind $T_n(x)$}
{n? mpNum? A real or complex number.}
{x? mpNum? A real or complex number.}
\end{mpFunctionsExtract}

\begin{mpFunctionsExtract}
\mpFunctionTwo
{chebyu? mpNum? the Chebyshev polynomial of the second kind $U_n(x)$}
{n? mpNum? A real or complex number.}
{x? mpNum? A real or complex number.}
\end{mpFunctionsExtract}

\section{Jacobi and Gegenbauer polynomials}

\begin{mpFunctionsExtract}
\mpFunctionFour
{jacobi? mpNum? the Jacobi polynomial $P_n^{(a,b)}$}
{n? mpNum? A real or complex number.}
{a? mpNum? A real or complex number.}
{b? mpNum? A real or complex number.}
{z? mpNum? A real or complex number.}
\end{mpFunctionsExtract}

\begin{mpFunctionsExtract}
\mpFunctionThreeNotImplemented
{ZernikeRadialMpMath? mpNum? the Zernike radial polynomials $R^m_n(r)$.}
{n? mpNum? An Integer.}
{m? mpNum? An Integer.}
{x? mpNum? A real number.}
\end{mpFunctionsExtract}

\begin{mpFunctionsExtract}
\mpFunctionThree
{gegenbauer? mpNum? the Gegenbauer polynomial $C_n^{(a)}(z)$}
{n? mpNum? A real or complex number.}
{a? mpNum? A real or complex number.}
{z? mpNum? A real or complex number.}
\end{mpFunctionsExtract}

\section{Hermite and Laguerre polynomials}

\begin{mpFunctionsExtract}
\mpFunctionTwo
{hermite? mpNum? the Hermite polynomial $H_n(z)$}
{n? mpNum? A real or complex number.}
{z? mpNum? A real or complex number.}
\end{mpFunctionsExtract}

\begin{mpFunctionsExtract}
\mpFunctionThree
{laguerre? mpNum? the generalized Laguerre polynomial $L_n^{\alpha}(z)$}
{n? mpNum? A real or complex number.}
{a? mpNum? A real or complex number.}
{z? mpNum? A real or complex number.}
\end{mpFunctionsExtract}

\begin{mpFunctionsExtract}
\mpFunctionTwoNotImplemented
{LaguerreLMpMath? mpNum? $L_n (x)$, the Laguerre polynomials of degree $n \geq 0$.}
{n? mpNum? An Integer.}
{x? mpNum? A real number.}
\end{mpFunctionsExtract}

\begin{mpFunctionsExtract}
\mpFunctionThreeNotImplemented
{AssociatedLaguerreMpMath? mpNum? $L^m_n (x)$, the associated Laguerre polynomials of degree $n \geq 0$ and order $m \geq 0$.}
{n? mpNum? An Integer.}
{m? mpNum? An Integer.}
{x? mpNum? A real number.}
\end{mpFunctionsExtract}

\chapter{Hypergeometric functions}

\section{Confluent Hypergeometric Limit Function}

\begin{mpFunctionsExtract}
\mpFunctionTwo
{hyp0f1? mpNum? the hypergeometric function ${}_0F_1$}
{a? mpNum? A real or complex number.}
{z? mpNum? A real or complex number.}
\end{mpFunctionsExtract}

\begin{mpFunctionsExtract}
\mpFunctionTwoNotImplemented
{Hypergeometric0F1RegularizedMpMath? mpNum? the regularized confluent hypergeometric limit function ${}_0F_1(b; x)$.}
{b? mpNum? A real number.}
{x? mpNum? A real number.}
\end{mpFunctionsExtract}

\section{Kummer's Confluent Hypergeometric Functions and related functions}

\begin{mpFunctionsExtract}
\mpFunctionThree
{hyp1f1? mpNum? the confluent hypergeometric function of the first kind ${}_1F_1(a,b;z)$}
{a? mpNum? A real or complex number.}
{b? mpNum? A real or complex number.}
{z? mpNum? A real or complex number.}
\end{mpFunctionsExtract}

\begin{mpFunctionsExtract}
\mpFunctionThreeNotImplemented
{Hypergeometric1F1RegularizedMpMath? mpNum? Kummer's regularized confluent hypergeometric function ${}_1F_1(a, b; x)$.}
{a? mpNum? A real number.}
{b? mpNum? A real number.}
{z? mpNum? A real number.}
\end{mpFunctionsExtract}

\begin{mpFunctionsExtract}
\mpFunctionThree
{hyperu? mpNum? the Tricomi confluent hypergeometric function $U$}
{a? mpNum? A real or complex number.}
{b? mpNum? A real or complex number.}
{z? mpNum? A real or complex number.}
\end{mpFunctionsExtract}

\begin{mpFunctionsExtract}
\mpFunctionThree
{hyp2f0? mpNum? the hypergeometric function ${}_2F_0$}
{a? mpNum? A real or complex number.}
{b? mpNum? A real or complex number.}
{z? mpNum? A real or complex number.}
\end{mpFunctionsExtract}

\section{Whittaker functions M and W}

\begin{mpFunctionsExtract}
\mpFunctionThree
{whitm? mpNum? the Whittaker function $M$}
{k? mpNum? A real or complex number.}
{m? mpNum? A real or complex number.}
{z? mpNum? A real or complex number.}
\end{mpFunctionsExtract}

\begin{mpFunctionsExtract}
\mpFunctionThree
{whitw? mpNum? the Whittaker function $W$}
{k? mpNum? A real or complex number.}
{m? mpNum? A real or complex number.}
{z? mpNum? A real or complex number.}
\end{mpFunctionsExtract}

\section{Gauss Hypergeometric Function}

\begin{mpFunctionsExtract}
\mpFunctionFour
{hyp2f1? mpNum? the square of $z$.}
{a? mpNum? A real or complex number.}
{b? mpNum? A real or complex number.}
{c? mpNum? A real or complex number.}
{z? mpNum? A real or complex number.}
\end{mpFunctionsExtract}

\begin{mpFunctionsExtract}
\mpFunctionFourNotImplemented
{Hypergeometric2F1RegularizedMpMath? mpNum? the regularized Gauss hypergeometric function ${}_2F_1(a, b; c; x)$.}
{a? mpNum? A real number.}
{b? mpNum? A real number.}
{c? mpNum? A real number.}
{z? mpNum? A real number.}
\end{mpFunctionsExtract}

\section{Additional Hypergeometric Functions}

\begin{mpFunctionsExtract}
\mpFunctionFour
{hyp1f2? mpNum? the the hypergeometric function ${}_1F_2(a_1; b_1, b_2; z)$}
{a1? mpNum? A real or complex number.}
{b1? mpNum? A real or complex number.}
{b2? mpNum? A real or complex number.}
{z? mpNum? A real or complex number.}
\end{mpFunctionsExtract}

\begin{mpFunctionsExtract}
\mpFunctionFive
{hyp2f2? mpNum? the hypergeometric function  ${}_2F_2(a_1; b_1, b_2; z)$.}
{a1? mpNum? A real or complex number.}
{a2? mpNum? A real or complex number.}
{b1? mpNum? A real or complex number.}
{b2? mpNum? A real or complex number.}
{z? mpNum? A real or complex number.}
\end{mpFunctionsExtract}

\begin{mpFunctionsExtract}
\mpFunctionSix
{hyp2f3? mpNum? the hypergeometric function ${}_2F_3(a_1,a2;b_1,b_2,b_3;z)$.}
{a1? mpNum? A real or complex number.}
{a2? mpNum? A real or complex number.}
{b1? mpNum? A real or complex number.}
{b2? mpNum? A real or complex number.}
{b3? mpNum? A real or complex number.}
{z? mpNum? A real or complex number.}
\end{mpFunctionsExtract}

\begin{mpFunctionsExtract}
\mpFunctionSix
{hyp3f2? mpNum? hypergeometric function ${}_3F_2$.}
{a1? mpNum? A real or complex number.}
{a2? mpNum? A real or complex number.}
{a3? mpNum? A real or complex number.}
{b1? mpNum? A real or complex number.}
{b2? mpNum? A real or complex number.}
{z? mpNum? A real or complex number.}
\end{mpFunctionsExtract}

\section{Generalized hypergeometric functions}

\begin{mpFunctionsExtract}
\mpFunctionThree
{hyper? mpNum? the generalized hypergeometric function${}_pF_q$}
{as? mpNum? list of real or complex numbers.}
{bs? mpNum? list of real or complex numbers.}
{z? mpNum? A real or complex number.}
\end{mpFunctionsExtract}

\begin{mpFunctionsExtract}
\mpFunctionFour
{hypercomb? mpNum? a weighted combination of hypergeometric functions}
{f? mpFunction? a real or function.}
{params? mpNum? list of real or complex numbers.}
{z? mpNum? A real or complex number.}
{Keywords? String ? discardknownzeros=True.}
\end{mpFunctionsExtract}

\section{Meijer G-function}

\begin{mpFunctionsExtract}
\mpFunctionFour
{meijerg? mpNum? the  Meijer G-function}
{as? mpNum? list of real or complex numbers.}
{bs? mpNum? list of real or complex numbers.}
{z? mpNum? A real or complex number.}
{Keywords? String?  r=1, series=1.}
\end{mpFunctionsExtract}

\section{Bilateral hypergeometric series}

\begin{mpFunctionsExtract}
\mpFunctionFour
{bihyper? mpNum? the  bilateral hypergeometric series}
{as? mpNum? list of real or complex numbers.}
{bs? mpNum? list of real or complex numbers.}
{z? mpNum? A real or complex number.}
{Keywords? String?  r=1, series=1.}
\end{mpFunctionsExtract}

\section{Hypergeometric functions of two variables}

\begin{mpFunctionsExtract}
\mpFunctionFour
{hyper2d? mpNum? the sum the generalized 2D hypergeometric series}
{a? mpNum? A real or complex number.}
{b? mpNum? A real or complex number.}
{x? mpNum? A real or complex number.}
{y? mpNum? A real or complex number.}
\end{mpFunctionsExtract}

\begin{mpFunctionsExtract}
\mpFunctionSix
{appellf1? mpNum? the Appell F1 hypergeometric function of two variables.}
{a? mpNum? A real or complex number.}
{b1? mpNum? A real or complex number.}
{b2? mpNum? A real or complex number.}
{c? mpNum? A real or complex number.}
{x? mpNum? A real or complex number.}
{y? mpNum? A real or complex number.}
\end{mpFunctionsExtract}

\begin{mpFunctionsExtract}
\mpFunctionSeven
{appellf2? mpNum? the Appell F2 hypergeometric function of two variables.}
{a? mpNum? A real or complex number.}
{b1? mpNum? A real or complex number.}
{b2? mpNum? A real or complex number.}
{c1? mpNum? A real or complex number.}
{c2? mpNum? A real or complex number.}
{x? mpNum? A real or complex number.}
{y? mpNum? A real or complex number.}
\end{mpFunctionsExtract}

\begin{mpFunctionsExtract}
\mpFunctionSeven
{appellf3? mpNum? the Appell F3 hypergeometric function of two variables.}
{a1? mpNum? A real or complex number.}
{a2? mpNum? A real or complex number.}
{b1? mpNum? A real or complex number.}
{b2? mpNum? A real or complex number.}
{c? mpNum? A real or complex number.}
{x? mpNum? A real or complex number.}
{y? mpNum? A real or complex number.}
\end{mpFunctionsExtract}

\begin{mpFunctionsExtract}
\mpFunctionSix
{appellf4? mpNum? the Appell F4 hypergeometric function of two variables.}
{a? mpNum? A real or complex number.}
{b? mpNum? A real or complex number.}
{c1? mpNum? A real or complex number.}
{c2? mpNum? A real or complex number.}
{x? mpNum? A real or complex number.}
{y? mpNum? A real or complex number.}
\end{mpFunctionsExtract}

\chapter{Elliptic functions}

\section{Elliptic arguments}

\begin{mpFunctionsExtract}
\mpFunctionOne
{qfrom? mpNum? the elliptic nome $q$.}
{Keywords? String? m=x; k=x; tau=x; qbar=x.}
\end{mpFunctionsExtract}

\begin{mpFunctionsExtract}
\mpFunctionOne
{qbarfrom? mpNum? the number-theoretic nome $\overline{q}$.}
{Keywords? String? m=x; k=x; tau=x; q=x.}
\end{mpFunctionsExtract}

\begin{mpFunctionsExtract}
\mpFunctionOne
{mfrom? mpNum? the elliptic parameter $m$.}
{Keywords? String? k=x; tau=x; q=x; qbar=x.}
\end{mpFunctionsExtract}

\begin{mpFunctionsExtract}
\mpFunctionOne
{kfrom? mpNum? the elliptic modulus $k$.}
{Keywords? String? m=x; tau=x; q=x; qbar=x.}
\end{mpFunctionsExtract}

\begin{mpFunctionsExtract}
\mpFunctionOne
{taufrom? mpNum? the elliptic half-period ratio $\tau$.}
{Keywords? String? m=x; k=x; q=x; qbar=x.}
\end{mpFunctionsExtract}

\section{Legendre elliptic integrals}

\begin{mpFunctionsExtract}
\mpFunctionOne
{ellipk? mpNum? the complete elliptic integral of the first kind, $K(m)$.}
{m? mpNum? A real or complex number.}
\end{mpFunctionsExtract}

\begin{mpFunctionsExtract}
\mpFunctionTwo
{ellipf? mpNum? the Legendre incomplete elliptic integral of the first kind $F(\phi,m)$.}
{phi? mpNum? A real or complex number.}
{m? mpNum? A real or complex number.}
\end{mpFunctionsExtract}

\begin{mpFunctionsExtract}
\mpFunctionOne
{ellipe? mpNum? the Legendre complete elliptic integral of the second kind $E(m)$.}
{m? mpNum? A real or complex number.}
\end{mpFunctionsExtract}

\begin{mpFunctionsExtract}
\mpFunctionTwo
{ellipef? mpNum? the Legendre incomplete elliptic integral of the second kind $E(\phi,m)$.}
{phi? mpNum? A real or complex number.}
{m? mpNum? A real or complex number.}
\end{mpFunctionsExtract}

\begin{mpFunctionsExtract}
\mpFunctionTwo
{ellippi? mpNum? the complete elliptic integral of the third kind $\Pi(n,m$.}
{n? mpNum? A real or complex number.}
{m? mpNum? A real or complex number.}
\end{mpFunctionsExtract}

\begin{mpFunctionsExtract}
\mpFunctionThree
{ellippif? mpNum? the Legendre incomplete elliptic integral of the third kind $\Pi(n;\phi,m)$.}
{n? mpNum? A real or complex number.}
{phi? mpNum? A real or complex number.}
{m? mpNum? A real or complex number.}
\end{mpFunctionsExtract}

\section{Carlson symmetric elliptic integrals}

\begin{mpFunctionsExtract}
\mpFunctionThree
{elliprf? mpNum? the Carlson symmetric elliptic integral of the first kind.}
{x? mpNum? A real or complex number.}
{y? mpNum? A real or complex number.}
{z? mpNum? A real or complex number.}
\end{mpFunctionsExtract}

\begin{mpFunctionsExtract}
\mpFunctionThree
{elliprc? mpNum? the degenerate Carlson symmetric elliptic integral of the first kind.}
{x? mpNum? A real or complex number.}
{y? mpNum? A real or complex number.}
{Keywords? String? pv=True.}
\end{mpFunctionsExtract}

\begin{mpFunctionsExtract}
\mpFunctionFour
{elliprj? mpNum? the Carlson symmetric elliptic integral of the third kind.}
{x? mpNum? A real or complex number.}
{y? mpNum? A real or complex number.}
{z? mpNum? A real or complex number.}
{p? mpNum? A real or complex number.}
\end{mpFunctionsExtract}

\begin{mpFunctionsExtract}
\mpFunctionThree
{elliprd? mpNum? the Carlson symmetric elliptic integral of the second kind.}
{x? mpNum? A real or complex number.}
{y? mpNum? A real or complex number.}
{z? mpNum? A real or complex number.}
\end{mpFunctionsExtract}

\begin{mpFunctionsExtract}
\mpFunctionThree
{elliprg? mpNum? the Carlson completely symmetric elliptic integral of the second kind.}
{x? mpNum? A real or complex number.}
{y? mpNum? A real or complex number.}
{z? mpNum? A real or complex number.}
\end{mpFunctionsExtract}

\section{Jacobi theta functions}

\begin{mpFunctionsExtract}
\mpFunctionFour
{jtheta? mpNum? the Jacobi theta function $\vartheta_n(z,q)$.}
{n? mpNum? An integer, where $n=1,2,3,4$.}
{z? mpNum? A real or complex number.}
{q? mpNum? A real or complex number.}
{Keywords? String? derivative=0.}
\end{mpFunctionsExtract}

\section{Jacobi elliptic functions}

\begin{mpFunctionsExtract}
\mpFunctionFour
{ellipfun? mpNum? any of the Jacobi elliptic functions.}
{kind? String? A function identifier.}
{u? mpNum? A real or complex number.}
{m? mpNum? A real or complex number.}
{Keywords? String?  q=None, k=None, tau=None.}
\end{mpFunctionsExtract}

\section{Klein j-invariant}

\begin{mpFunctionsExtract}
\mpFunctionOne
{kleinj? mpNum? the Klein j-invariant.}
{tau? mpNum? A real or complex number.}
\end{mpFunctionsExtract}

\chapter{Zeta functions, L-series and polylogarithms}

\section{Riemann and Hurwitz zeta functions}

\begin{mpFunctionsExtract}
\mpFunctionTwo
{zeta? mpNum? the Riemann zeta function}
{s? mpNum? A real or complex number.}
{Keywords? String? derivative=0.}
\end{mpFunctionsExtract}

\begin{mpFunctionsExtract}
\mpFunctionThree
{hurwitz? mpNum? the  Hurwitz  zeta function}
{s? mpNum? A real or complex number.}
{a? mpNum? A real or complex number.}
{Keywords? String? derivative=0.}
\end{mpFunctionsExtract}

\section{Dirichlet L-series}

\begin{mpFunctionsExtract}
\mpFunctionOne
{altzeta? mpNum? the Dirichlet eta function, $\eta(s)$}
{s? mpNum? A real or complex number.}
\end{mpFunctionsExtract}

\begin{mpFunctionsExtract}
\mpFunctionOneNotImplemented
{DirichletEtam1MpMath? mpNum? the Dirichlet function $\eta(s) - 1$.}
{x? mpNum? A real number.}
\end{mpFunctionsExtract}

\begin{mpFunctionsExtract}
\mpFunctionOneNotImplemented
{DirichletBetaMpMath? mpNum? the Dirichlet function $\beta(s)$.}
{s? mpNum? A real number.}
\end{mpFunctionsExtract}

\begin{mpFunctionsExtract}
\mpFunctionOneNotImplemented
{DirichletLambdaMpMath? mpNum? the Dirichlet function $\beta(s)$.}
{s? mpNum? A real number.}
\end{mpFunctionsExtract}

\begin{mpFunctionsExtract}
\mpFunctionThree
{dirichlet? mpNum? the Dirichlet L-function}
{s? mpNum? A real or complex number.}
{chi? mpNum? A periodic sequence.}
{Keywords? mpNum? derivative=0.}
\end{mpFunctionsExtract}

\section{Stieltjes constants}

\begin{mpFunctionsExtract}
\mpFunctionTwo
{stieltjes? mpNum? the $n$-th Stieltjes constant}
{n? mpNum? A real or complex number.}
{a? mpNum? A real or complex number.}
\end{mpFunctionsExtract}

\section{Zeta function zeros}

\begin{mpFunctionsExtract}
\mpFunctionTwo
{zetazero? mpNum? the $n$-th nontrivial zero of $\zeta(s)$ on the critical line}
{n? mpNum? An integer.}
{Keywords? String? verbose=False.}
\end{mpFunctionsExtract}

\begin{mpFunctionsExtract}
\mpFunctionOne
{nzeros? mpNum? the number of zeros of the Riemann zeta function in $(0,1) \times (0,t)$, usually denoted by $N(t)$.}
{t? mpNum? An integer.}
\end{mpFunctionsExtract}

\section{Riemann-Siegel Z function and related functions}

\begin{mpFunctionsExtract}
\mpFunctionOne
{siegelz? mpNum? the Riemann-Siegel Z function}
{t? mpNum? A real or complex number.}
\end{mpFunctionsExtract}

\begin{mpFunctionsExtract}
\mpFunctionOne
{siegeltheta? mpNum? the Riemann-Siegel theta function}
{t? mpNum? A real or complex number.}
\end{mpFunctionsExtract}

\begin{mpFunctionsExtract}
\mpFunctionOne
{grampoint? mpNum? the $n$-th Gram point $g_n$, defined as the solution to the equation $\theta(g_n)=\pi n$ where $\theta(t)$ is the Riemann-Siegel theta function}
{n? mpNum? A real or complex number.}
\end{mpFunctionsExtract}

\begin{mpFunctionsExtract}
\mpFunctionOne
{backlunds? mpNum? the function $S(t) = \text{arg}\zeta\left(\tfrac{1}{2}+it\right)/\pi$.}
{t? mpNum? A real or complex number.}
\end{mpFunctionsExtract}

\section{Lerch transcendent and related functions}

\begin{mpFunctionsExtract}
\mpFunctionThree
{lerchphi? mpNum? the Lerch transcendent}
{z? mpNum? A real or complex number.}
{s? mpNum? A real or complex number.}
{a? mpNum? A real or complex number.}
\end{mpFunctionsExtract}

\begin{mpFunctionsExtract}
\mpFunctionTwoNotImplemented
{FermiDiracIntMpMath? mpNum? the complete Fermi-Dirac integrals $F_n(x)$ of integer order.}
{x? mpNum? A real number.}
{n? mpNum? An Integer.}
\end{mpFunctionsExtract}

\begin{mpFunctionsExtract}
\mpFunctionOneNotImplemented
{FermiDiracPHalfMpMath? mpNum? the complete Fermi-Dirac integral $F_{-1/2}(x)$.}
{s? mpNum? A real number.}
\end{mpFunctionsExtract}

\begin{mpFunctionsExtract}
\mpFunctionOneNotImplemented
{FermiDiracHalfMpMath? mpNum? the complete Fermi-Dirac integral $F_{1/2}(x)$.}
{s? mpNum? A real number.}
\end{mpFunctionsExtract}

\begin{mpFunctionsExtract}
\mpFunctionOneNotImplemented
{FermiDirac3HalfMpMath? mpNum? the complete Fermi-Dirac integral $F_{3/2}(x)$.}
{s? mpNum? A real number.}
\end{mpFunctionsExtract}

\begin{mpFunctionsExtract}
\mpFunctionTwoNotImplemented
{LegendreChiMpMath? mpNum? the Legendre Chi-Function function $\chi_s(x)$.}
{s? mpNum? A real number.}
{x? mpNum? A real number.}
\end{mpFunctionsExtract}

\begin{mpFunctionsExtract}
\mpFunctionOneNotImplemented
{InverseTangentMpMath? mpNum? the inverse-tangent integral.}
{x? mpNum? A real number.}
\end{mpFunctionsExtract}

\section{Polylogarithms and Clausen functions}

\begin{mpFunctionsExtract}
\mpFunctionTwo
{polylog? mpNum? the polylogarithm}
{s? mpNum? A real or complex number.}
{z? mpNum? A real or complex number.}
\end{mpFunctionsExtract}

\begin{mpFunctionsExtract}
\mpFunctionOne
{dilog? mpNum? the dilogarithm function $\text{Li}_2(x)$.}
{x? mpNum? A real number.}
\end{mpFunctionsExtract}

\begin{mpFunctionsExtract}
\mpFunctionTwoNotImplemented
{DebyeMpMath? mpNum? the Debye function of order $n$.}
{n? mpNum? An Integer.}
{x? mpNum? A real number.}
\end{mpFunctionsExtract}

\begin{mpFunctionsExtract}
\mpFunctionTwo
{clsinlog? mpNum? the Clausen sine function}
{s? mpNum? A real or complex number.}
{z? mpNum? A real or complex number.}
\end{mpFunctionsExtract}

\begin{mpFunctionsExtract}
\mpFunctionTwo
{clcos? mpNum? the Clausen cosine function}
{s? mpNum? A real or complex number.}
{z? mpNum? A real or complex number.}
\end{mpFunctionsExtract}

\begin{mpFunctionsExtract}
\mpFunctionTwo
{polyexp? mpNum? the polyexponential function}
{s? mpNum? A real or complex number.}
{z? mpNum? A real or complex number.}
\end{mpFunctionsExtract}

\section{Zeta function variants}

\begin{mpFunctionsExtract}
\mpFunctionOne
{primezeta? mpNum? the prime zeta function.}
{s? mpNum? A real or complex number.}
\end{mpFunctionsExtract}

\begin{mpFunctionsExtract}
\mpFunctionTwo
{secondzeta? mpNum? the secondary zeta function}
{s? mpNum? A real or complex number.}
{Keywords? String? a=0.015, error=False.}
\end{mpFunctionsExtract}

\chapter{Number-theoretical, combinatorial and integer functions}

\section{Fibonacci numbers}

\begin{mpFunctionsExtract}
\mpFunctionTwo
{fibonacci? mpNum? the $n$-th Fibonacci number, $F(n)$}
{n? mpNum? A real or complex number.}
{Keywords? String? derivative=0.}
\end{mpFunctionsExtract}

\begin{mpFunctionsExtract}
\mpFunctionTwo
{fib? mpNum? the $n$-th Fibonacci number, $F(n)$}
{n? mpNum? A real or complex number.}
{Keywords? String? derivative=0.}
\end{mpFunctionsExtract}

\section{Bernoulli numbers and polynomials}

\begin{mpFunctionsExtract}
\mpFunctionOne
{bernoulli? mpNum? the $n$th Bernoulli number, $B_n$, for any integer $n>0$}
{n? mpNum? An integer}
\end{mpFunctionsExtract}

\begin{mpFunctionsExtract}
\mpFunctionOne
{bernfrac? mpNum? a tuple of integers $(p,q)$ such that $p/q=B_n$ exactly, where $B_n$denotes the $n$-th Bernoulli number.}
{n? mpNum? An integer}
\end{mpFunctionsExtract}

\begin{mpFunctionsExtract}
\mpFunctionTwo
{bernpoly? mpNum? the Bernoulli polynomial $B_n(z)$}
{n? mpNum? A real or complex number.}
{z? mpNum? A real or complex number.}
\end{mpFunctionsExtract}

\section{Euler numbers and polynomials}

\begin{mpFunctionsExtract}
\mpFunctionOne
{eulernum? mpNum? the $n$-th Euler number}
{n? mpNum? An integer}
\end{mpFunctionsExtract}

\begin{mpFunctionsExtract}
\mpFunctionTwo
{eulerpoly? mpNum? the Euler polynomial $E_n(z)$}
{n? mpNum? A real or complex number.}
{z? mpNum? A real or complex number.}
\end{mpFunctionsExtract}

\section{Bell numbers and polynomials}

\begin{mpFunctionsExtract}
\mpFunctionTwo
{bell? mpNum? the Bell polynomial $B_n(x)$}
{n? mpNum? A non-negative integer.}
{x? mpNum? A real or complex number.}
\end{mpFunctionsExtract}

\section{Stirling numbers}

\begin{mpFunctionsExtract}
\mpFunctionThree
{stirling1? mpNum? the Stirling number of the first kind $s(n,k)$}
{n? mpNum? A real or complex number.}
{k? mpNum? A real or complex number.}
{Keywords? String? exact=False.}
\end{mpFunctionsExtract}

\begin{mpFunctionsExtract}
\mpFunctionThree
{stirling2? mpNum? the Stirling number of the second kind $s(n,k)$}
{n? mpNum? A real or complex number.}
{k? mpNum? A real or complex number.}
{Keywords? String? exact=False.}
\end{mpFunctionsExtract}

\section{Prime counting functions}

\begin{mpFunctionsExtract}
\mpFunctionOne
{primepi? mpNum? the prime counting function}
{x? mpNum? A real number}
\end{mpFunctionsExtract}

\begin{mpFunctionsExtract}
\mpFunctionOne
{primepi2? mpNum? an interval (as an mpi instance) providing bounds for the value of the prime counting function $\pi(x)$}
{x? mpNum? A real number}
\end{mpFunctionsExtract}

\begin{mpFunctionsExtract}
\mpFunctionOne
{riemannr? mpNum? the Riemann R function, a smooth approximation of the prime counting function $\pi(x)$}
{x? mpNum? A real number}
\end{mpFunctionsExtract}

\section{Miscellaneous functions}

\begin{mpFunctionsExtract}
\mpFunctionTwo
{cyclotomic? mpNum? the cyclotomic polynomial $\Phi_n(x)$}
{n? mpNum? A real or complex number.}
{x? mpNum? A real or complex number.}
\end{mpFunctionsExtract}

\begin{mpFunctionsExtract}
\mpFunctionOne
{mangoldt? mpNum? the von Mangoldt function}
{n? mpNum? An integer}
\end{mpFunctionsExtract}

\chapter{q-functions}

\section{q-Pochhammer symbol}

\begin{mpFunctionsExtract}
\mpFunctionThree
{qp? mpNum? the q-Pochhammer symbol (or q-rising factorial)}
{a? mpNum? A real or complex number.}
{q? mpNum? A real or complex number.}
{n? mpNum? An integer.}
\end{mpFunctionsExtract}

\section{q-gamma and factorial}

\begin{mpFunctionsExtract}
\mpFunctionTwo
{qgamma? mpNum? the q-gamma function}
{z? mpNum? A real or complex number.}
{q? mpNum? A real or complex number.}
\end{mpFunctionsExtract}

\begin{mpFunctionsExtract}
\mpFunctionTwo
{qfac? mpNum? the  q-factorial}
{z? mpNum? A real or complex number.}
{q? mpNum? A real or complex number.}
\end{mpFunctionsExtract}

\section{Hypergeometric q-series}

\begin{mpFunctionsExtract}
\mpFunctionFour
{qhyper? mpNum? the hypergeometric q-series}
{as? mpNum? A real or complex number.}
{bs? mpNum? A real or complex number.}
{q? mpNum? A real or complex number.}
{z? mpNum? A real or complex number.}
\end{mpFunctionsExtract}

\chapter{Matrix functions}

\section{Matrix exponential}

\begin{mpFunctionsExtract}
\mpFunctionTwo
{expm? mpNum? the matrix exponential of a square matrix $A$}
{A? mpNum? A real or complex matrix.}
{Keywords? String? method='taylor'.}
\end{mpFunctionsExtract}

\section{Matrix cosine}

\begin{mpFunctionsExtract}
\mpFunctionOne
{cosm? mpNum? the matrix cosine of a square matrix $A$}
{A? mpNum? A real or complex matrix.}
\end{mpFunctionsExtract}

\section{Matrix sine}

\begin{mpFunctionsExtract}
\mpFunctionOne
{sinm? mpNum? the matrix sine of a square matrix $A$}
{A? mpNum? A real or complex matrix.}
\end{mpFunctionsExtract}

\section{Matrix square root}

\begin{mpFunctionsExtract}
\mpFunctionTwo
{sqrtm? mpNum? a square root of a square matrix $A$}
{A? mpNum? A real or complex matrix.}
{Keywords? String?  mayrotate=2.}
\end{mpFunctionsExtract}

\section{Matrix logarithm}

\begin{mpFunctionsExtract}
\mpFunctionOne
{logm? mpNum? the matrix logarithm of a square matrix $A$}
{A? mpNum? A real or complex matrix.}
\end{mpFunctionsExtract}

\section{Matrix power}

\begin{mpFunctionsExtract}
\mpFunctionTwo
{powm? mpNum? $A^r=\exp(A \log r)$ for a matrix $A$ and complex number $r$}
{A? mpNum? A real or complex matrix.}
{r? mpNum? A real or complex number.}
\end{mpFunctionsExtract}

\chapter{Eigensystems and related Decompositions}

\section{Singular value decomposition}

\begin{mpFunctionsExtract}
\mpFunctionTwo
{svd? mpNum? the singular value decomposition of matrix A}
{A? mpNum? A real or complex number.}
{Keywords? String? compute\_uv = True.}
\end{mpFunctionsExtract}

\section{The Schur decomposition}

\begin{mpFunctionsExtract}
\mpFunctionOne
{schur? mpNum? the Schur decomposition of a square matrix $A$}
{A? mpNum? A real or complex matrix.}
\end{mpFunctionsExtract}

\section{The eigenvalue problem}

\begin{mpFunctionsExtract}
\mpFunctionTwo
{eig? mpNum? the solution of the (ordinary) eigenvalue problem for a real or complex square matrix $A$}
{A? mpNum? A real or complex number.}
{Keywords? String? left = False, right = False.}
\end{mpFunctionsExtract}

\section{The symmetric eigenvalue problem}

\begin{mpFunctionsExtract}
\mpFunctionTwo
{eigh? mpNum? the solution of the (ordinary) eigenvalue problem for a real symmetric or complex hermitian square matrix $A$}
{A? mpNum? A real or complex number.}
{Keywords? String? eigvals\_only = False.}
\end{mpFunctionsExtract}

\chapter{Polynomials}

\section{Polynomial evaluation}

\begin{mpFunctionsExtract}
\mpFunctionThree
{polyval? mpNum? a polynomial }
{coeffs? mpNum? A list of coefficients (real or complex numbers).}
{x? mpNum? A real or complex number.}
{Keywords? String? derivative=False.}
\end{mpFunctionsExtract}

\section{Polynomial roots}

\begin{mpFunctionsExtract}
\mpFunctionTwo
{polyroots? mpNum? all roots (real or complex) of a given polynomial.}
{coeffs? mpNum? A real or complex number.}
{Keywords? String? maxsteps=50, cleanup=True, extraprec=10, error=False.}
\end{mpFunctionsExtract}

\chapter{Root-finding and optimization}

\section{Root-finding}

\begin{mpFunctionsExtract}
\mpFunctionThree
{findroot? mpNum?  a solution to $f(x)=0$, using \textit{x0} as starting point or interval for \textit{x}.}
{f? mpNum? A one dimensional function}
{x0? mpNum? A real or complex number.}
{Keywords? String? solver=Secant, tol=None, verbose=False, verify=True. Many more, see below.}
\end{mpFunctionsExtract}

\section{Solvers}

\chapter{Sums, products, limits and extrapolation}

\section{Summation}

\begin{mpFunctionsExtract}
\mpFunctionThree
{nsum? mpNum? a one dimensional (possibly infinite) sum.}
{f? mpNum? A one dimensional function}
{interval? mpNum? A real interval.}
{Keywords? String? method=r+s, tol=eps, verbose=False, maxterms=10*dps. Many more, see below.}
\end{mpFunctionsExtract}

\begin{mpFunctionsExtract}
\mpFunctionFour
{nsum2d? mpNum? a two dimensional (possibly infinite) sum.}
{f? mpNum? A one dimensional function}
{interval1? mpNum? A real interval.}
{interval2? mpNum? A real interval.}
{Keywords? String? method=r+s, tol=eps, verbose=False, maxterms=10*dps. Many more, see below.}
\end{mpFunctionsExtract}

\begin{mpFunctionsExtract}
\mpFunctionFive
{nsum3d? mpNum? a three dimensional (possibly infinite) sum .}
{f? mpNum? A one dimensional function}
{interval1? mpNum? A real interval.}
{interval2? mpNum? A real interval.}
{interval3? mpNum? A real interval.}
{Keywords? String? method=r+s, tol=eps, verbose=False, maxterms=10*dps. Many more, see below.}
\end{mpFunctionsExtract}

\begin{mpFunctionsExtract}
\mpFunctionThree
{sumem? mpNum? an infinite series of an analytic summand f using the Euler-Maclaurin formula}
{f? mpNum? A one dimensional function}
{interval? mpNum? A real interval.}
{Keywords? String?  tol=None, reject=10, integral=None, adiffs=None, bdiffs=None, verbose=False, error=False, fastabort=False}
\end{mpFunctionsExtract}

\begin{mpFunctionsExtract}
\mpFunctionThree
{sumap? mpNum? an infinite series of an analytic summand f using the Abel-Plana formula.}
{f? mpNum? A one dimensional function}
{interval? mpNum? A real interval.}
{Keywords? String?  integral=None, error=False}
\end{mpFunctionsExtract}

\section{Products}

\begin{mpFunctionsExtract}
\mpFunctionThree
{nprod? mpNum? a one dimensional (possibly infinite) product.}
{f? mpNum? A one dimensional function}
{interval? mpNum? A real interval.}
{Keywords? String? nsum=False.}
\end{mpFunctionsExtract}

\section{Limits}

\begin{mpFunctionsExtract}
\mpFunctionThree
{limit? mpNum? an estimate of the limit.}
{f? mpNum? A one dimensional function}
{interval? mpNum? A real interval.}
{Keywords? String? direction=1, exp=False.}
\end{mpFunctionsExtract}

\section{Extrapolation}

\begin{mpFunctionsExtract}
\mpFunctionOne
{richardson? mpNum? the $N$-term Richardson extrapolate for the limit.}
{seq? mpNum? a list of the first $N$ elements of a slowly convergent infinite sequence.}
\end{mpFunctionsExtract}

\begin{mpFunctionsExtract}
\mpFunctionTwo
{shanks? mpNum? the $N$-term Shanks extrapolate for the limit.}
{seq? mpNum? a list of the first $N$ elements of a slowly convergent infinite sequence.}
{Keywords? String? table=None, randomized=False.}
\end{mpFunctionsExtract}

\begin{mpFunctionsExtract}
\mpFunctionOne
{levin? Object? an object with sn update method. Only for use within Python}
{Keywords? String? method='levin', variant='u'.}
\end{mpFunctionsExtract}

\begin{mpFunctionsExtract}
\mpFunctionZero
{cohen\_alt? Object? an object with sn update method. Only for use within Python}
\end{mpFunctionsExtract}

\chapter{Differentiation}

\section{Numerical derivatives}

\begin{mpFunctionsExtract}
\mpFunctionFour
{diff? mpNum? the $n$-th derivative $f^{(n)}(x)$.}
{f? mpNum? A one dimensional function}
{x? mpNum? A real number.}
{n? mpNum? An integer, indicating the nth derivative.}
{Keywords? String? method=step, tol=eps, direction=0. Many more, see below.}
\end{mpFunctionsExtract}

\begin{mpFunctionsExtract}
\mpFunctionFour
{diffs? mpNum? the $n$-th derivative $f^{(n)}(x)$.}
{f? mpNum? A one dimensional function}
{x? mpNum? A real number.}
{n? mpNum? An integer, indicating the nth derivative.}
{Keywords? String? method=step, tol=eps, direction=0. Many more, see below.}
\end{mpFunctionsExtract}

\section{Composition of derivatives}

\begin{mpFunctionsExtract}
\mpFunctionOne
{diffsprod? mpNum? the result of the differentiation of products of functions.}
{factors? Object? a list of $N$ iterables or generators.}
\end{mpFunctionsExtract}

\begin{mpFunctionsExtract}
\mpFunctionOne
{diffsexp? mpNum? the result of the differentiation of the exponential of functions.}
{fdiffs? Object? a list of $N$ iterables or generators.}
\end{mpFunctionsExtract}

\section{Fractional derivatives}

\begin{mpFunctionsExtract}
\mpFunctionFour
{differint? mpNum? the Riemann-Liouville differintegral.}
{f? mpNum? A one dimensional function}
{x? mpNum? A real interval.}
{n? mpNum? A real interval.}
{x0? mpNum? A real interval.}
\end{mpFunctionsExtract}

\chapter{Numerical integration (quadrature)}

\section{Standard quadrature}

\begin{mpFunctionsExtract}
\mpFunctionThree
{quad? mpNum? a one dimensional integral.}
{f? mpNum? A one dimensional function}
{interval? mpNum? A real interval.}
{Keywords? String? method=TanhSinh, error=False, verbose=False, maxdegree. Many more, see below.}
\end{mpFunctionsExtract}

\begin{mpFunctionsExtract}
\mpFunctionFour
{quad2d? mpNum? a two dimensional integral.}
{f? mpNum? A one dimensional function}
{interval1? mpNum? A real interval.}
{interval2? mpNum? A real interval.}
{Keywords? String? method=TanhSinh, error=False, verbose=False, maxdegree. Many more, see below.}
\end{mpFunctionsExtract}

\begin{mpFunctionsExtract}
\mpFunctionFive
{quad3d? mpNum? a three dimensional integral.}
{f? mpNum? A one dimensional function}
{interval1? mpNum? A real interval.}
{interval2? mpNum? A real interval.}
{interval3? mpNum? A real interval.}
{Keywords? String? method=TanhSinh, error=False, verbose=False, maxdegree. Many more, see below.}
\end{mpFunctionsExtract}

\begin{mpFunctionsExtract}
\mpFunctionThree
{quadosc? mpNum? a one dimensional oscillatory integral.}
{f? mpNum? A one dimensional function}
{interval? mpNum? A real interval.}
{Keywords? String? omega=None, period=None, zeros=None}
\end{mpFunctionsExtract}

\section{Main Quadrature rules}

\section{Additional Quadrature rules}

\chapter{Ordinary differential equations}

\section{Solving the ODE initial value problem}

\begin{mpFunctionsExtract}
\mpFunctionFour
{odefun? mpNum? a function $y(x) = [y_0(x),y_1(x),\ldots,y_n(x)]$ that is a numerical solution of a $n+1$-dimensional first-order ordinary differential equation (ODE) system}
{f? mpNum? A one dimensional function}
{x0? mpNum? A real number.}
{y0? mpNum? A real number.}
{Keywords? String?  tol=None, degree=None, method='taylor', verbose=False}
\end{mpFunctionsExtract}

\chapter{Function approximation}

\section{Taylor series}

\begin{mpFunctionsExtract}
\mpFunctionFour
{taylor? mpNum? a list of coefficients of a degree-$n$ Taylor polynomial around the point $x$ of the given function $f$.}
{f? mpNum? A one dimensional function}
{x? mpNum? A real number.}
{n? mpNum? A real number.}
{Keywords? String?  method=step, tol=eps, direction=0. Many more, see diff()}
\end{mpFunctionsExtract}

\section{Pade approximation}

\begin{mpFunctionsExtract}
\mpFunctionThree
{pade? mpNum? coefficients of a Pade approximation of degree $(L,M)$ to a function}
{a? mpNum? A list of at least $L+M+1$ Taylor coefficients approximating a function $A(x)$ }
{L? mpNum? An integer, specifying the degree of polynomials $P$.}
{M? mpNum? An integer, specifying the degree of polynomials $Q$.}
\end{mpFunctionsExtract}

\section{Chebyshev approximation}

\begin{mpFunctionsExtract}
\mpFunctionFour
{chebyfit? mpNum? coefficients of a polynomial of degree $N-1$ that approximates the given function $f$ on the interval $[a,b]$}
{f? mpNum? A one dimensional function}
{interval? mpNum? A real interval.}
{N? mpNum? An integer.}
{Keywords? String? error=False}
\end{mpFunctionsExtract}

\section{Fourier series}

\begin{mpFunctionsExtract}
\mpFunctionThree
{fourier? mpNum? two lists of coefficients of the Fourier series of degree $N$ of the given function on the interval $[a,b]$..}
{f? mpNum? A one dimensional function}
{interval? mpNum? A real interval.}
{N? mpNum? An integer.}
\end{mpFunctionsExtract}

\begin{mpFunctionsExtract}
\mpFunctionThree
{fouriereval? mpNum? the result of the evaluation of a Fourier series (in the format computed by by fourier() for the given interval) at the point $x$.}
{series? mpNum? a pair $(c,s)$ where $c$ is the cosine series and $s$ is the sine series}
{interval? mpNum? A real interval.}
{x? mpNum? A real number.}
\end{mpFunctionsExtract}

\chapter{Number identification}

\section{Constant recognition}

\begin{mpFunctionsExtract}
\mpFunctionThree
{identify? String? the result of an attempt to find an exact formula for a given real number $x$}
{x? mpNum? A real number}
{constants? String? A list of known constants.}
{Keywords? String?  tol=None, maxcoeff=1000, full=False, verbose=False}
\end{mpFunctionsExtract}

\section{Algebraic identification}

\begin{mpFunctionsExtract}
\mpFunctionThree
{findpoly? String? the coefficients of an integer polynomial $P$ of degree at most $n$ such that $P(x) \sim 0$}
{x? mpNum? A real number}
{n? Integer? max degree of polynomial.}
{Keywords? String?  tol=None, maxcoeff=1000, maxsteps=100, verbose=False}
\end{mpFunctionsExtract}

\section{Integer relations (PSLQ)}

\begin{mpFunctionsExtract}
\mpFunctionTwo
{pslq? mpNum[]? list of integers to approximate a function.}
{x? mpNum? A vector of real numbers $x=[x_0,x_1,\ldots,x_n]$}
{Keywords? String?  tol=None, maxcoeff=1000, full=False, verbose=False}
\end{mpFunctionsExtract}

\chapter{Date, Time and Financial Functions}

\section{Date and Time: Conversions from Serial Number}

\begin{mpFunctionsExtract}
\mpWorksheetFunctionOneNotImplemented
{SECOND? mpReal? the seconds of a time value. The second is given as an integer in the range 0 (zero) to 59.}
{Timevalue? Variant? The time that contains the seconds you want to find.}
\end{mpFunctionsExtract}

\begin{mpFunctionsExtract}
\mpWorksheetFunctionOneNotImplemented
{MINUTE? mpReal? the minutes of a time value. The minute is given as an integer, ranging from 0 to 59.}
{Timevalue? Variant? The time that contains the minute you want to find.}
\end{mpFunctionsExtract}

\begin{mpFunctionsExtract}
\mpWorksheetFunctionOneNotImplemented
{HOUR? mpReal? the hour of a time value. The hour is given as an integer, ranging from 0 (12:00 A.M.) to 23 (11:00 P.M.).}
{Timevalue? Variant? The time that contains the hour you want to find.}
\end{mpFunctionsExtract}

\begin{mpFunctionsExtract}
\mpWorksheetFunctionOneNotImplemented
{DAY? mpReal? the day of a date, represented by a serial number. The day is given as an integer ranging from 1 to 31.}
{Datevalue? Date? The date of the day you are trying to find.}
\end{mpFunctionsExtract}

\begin{mpFunctionsExtract}
\mpWorksheetFunctionTwoNotImplemented
{DAYS? mpReal? the number of days between two dates. EndDateValue and  and StartDateValue  are the two dates between which you want to know the number of days.}
{EndDatevalue? Date? The end of the time interval.}
{StartDatevalue? Date? The start of the time interval.}
\end{mpFunctionsExtract}

\begin{mpFunctionsExtract}
\mpWorksheetFunctionOneNotImplemented
{MONTH? mpReal? the month of a date represented by a serial number. The month is given as an integer, ranging from 1 (January) to 12 (December).}
{Datevalue? Date? The date of the month you are trying to find.}
\end{mpFunctionsExtract}

\begin{mpFunctionsExtract}
\mpWorksheetFunctionOneNotImplemented
{YEAR? mpReal? the year corresponding to a date. The year is returned as an integer in the range 1900-9999.}
{Datevalue? Date? The date of the year you want to find.}
\end{mpFunctionsExtract}

\begin{mpFunctionsExtract}
\mpWorksheetFunctionTwoNotImplemented
{WEEKDAY? mpReal? the day of the week corresponding to a date. The day is given as an integer, ranging from 1 (Sunday) to 7 (Saturday), by default.}
{Datevalue? Date? A sequential number that represents the date of the day you are trying to find.}
{ReturnType? Integer? A number that determines the type of return value.}
\end{mpFunctionsExtract}

\begin{mpFunctionsExtract}
\mpWorksheetFunctionTwoNotImplemented
{WEEKNUM? mpReal? the week number of a specific date.}
{Datevalue? Date? A date within the week. }
{ReturnType? Integer? A number that determines on which day the week begins. The default is 1.}
\end{mpFunctionsExtract}

\begin{mpFunctionsExtract}
\mpWorksheetFunctionTwoNotImplemented
{WEEKNUM-ADD? mpReal? the week number of a specific date.}
{Datevalue? Date? A date within the week. }
{ReturnType? Integer? A number that determines on which day the week begins. The default is 1.}
\end{mpFunctionsExtract}

\begin{mpFunctionsExtract}
\mpWorksheetFunctionTwoNotImplemented
{ISOWEEKNUM? mpReal? the week number of a specific date.}
{Datevalue? Date? A date within the week. }
{ReturnType? Integer? A number that determines on which day the week begins. The default is 1.}
\end{mpFunctionsExtract}

\section{Date and Time: Conversions to Serial Number}

\begin{mpFunctionsExtract}
\mpWorksheetFunctionThreeNotImplemented
{DATE? Date? the sequential serial number that represents a particular date.}
{Year? mpReal? A number that determines the Year (1900-9999).}
{Month? mpReal? A number that determines the Month (1-12).}
{Day? mpReal? A number that determines the day of the month (1-31).}
\end{mpFunctionsExtract}

\begin{mpFunctionsExtract}
\mpWorksheetFunctionOneNotImplemented
{EASTERSUNDAY? Date? the date of Easter Sunday in a given year.}
{Year? mpReal? an integer between 1583 and 9956 or between 0 and 99, specifying the year. }
\end{mpFunctionsExtract}

\begin{mpFunctionsExtract}
\mpWorksheetFunctionOneNotImplemented
{DATEVALUE? Date? a serial number that Excel recognizes as a date}
{DateText? String? Text that represents a date in an Excel date format}
\end{mpFunctionsExtract}

\begin{mpFunctionsExtract}
\mpWorksheetFunctionTwoNotImplemented
{EDATE? Date? a serial number that Excel recognizes as a date}
{StartDate? Date? A date that represents the start date.}
{Months? Integer? The number of months before or after StartDate. A positive value for months yields a future date; a negative value yields a past date.}
\end{mpFunctionsExtract}

\begin{mpFunctionsExtract}
\mpWorksheetFunctionTwoNotImplemented
{EOMONTH? Date? a serial number that Excel recognizes as a date}
{StartDate? Date? A date that represents the start date.}
{Months? Integer? The number of months before or after StartDate. A positive value for months yields a future date; a negative value yields a past date.}
\end{mpFunctionsExtract}

\begin{mpFunctionsExtract}
\mpWorksheetFunctionZeroNotImplemented
{NOW? Date? the serial number of the current date and time.}
\end{mpFunctionsExtract}

\begin{mpFunctionsExtract}
\mpWorksheetFunctionThreeNotImplemented
{TIME? Date? the decimal number for a particular time. If the cell format was General before the function was entered, the result is formatted as a date.}
{Hour? mpReal? A number from 0 (zero) to 32767 representing the hour. Any value greater than 23 will be divided by 24 and the remainder will be treated as the hour value. For example, TIME(27,0,0) = TIME(3,0,0) = .125 or 3:00 AM.}
{Minute? mpReal? A number from 0 to 32767 representing the minute. Any value greater than 59 will be converted to hours and minutes. For example, TIME(0,750,0) = TIME(12,30,0) = .520833 or 12:30 PM.}
{Second? mpReal? A number from 0 to 32767 representing the second. Any value greater than 59 will be converted to hours, minutes, and seconds. For example, TIME(0,0,2000) = TIME(0,33,22) = .023148 or 12:33:20 AM.}
\end{mpFunctionsExtract}

\begin{mpFunctionsExtract}
\mpWorksheetFunctionOneNotImplemented
{TIMEVALUE? mpReal? the decimal number of the time represented by a text string. The decimal number is a value ranging from 0 (zero) to 0.99999999, representing the times from 0:00:00 (12:00:00 AM) to 23:59:59 (11:59:59 P.M.).}
{TimeText? String? A text string that represents a time in any one of the Microsoft Excel time formats.}
\end{mpFunctionsExtract}

\begin{mpFunctionsExtract}
\mpWorksheetFunctionZeroNotImplemented
{TODAY? mpReal? the serial number of the current date.}
\end{mpFunctionsExtract}

\begin{mpFunctionsExtract}
\mpWorksheetFunctionThreeNotImplemented
{WORKDAY? Date? a number that represents a date that is the indicated number of working days before or after a date (the starting date).}
{StartDate? Date? A date that represents the start date.}
{Days? Integer? The number of nonweekend and nonholiday days before or after StartDate. A positive value for days yields a future date; a negative value yields a past date.}
{Holidays? DateList?  An optional list of one or more dates to exclude from the working calendar}
\end{mpFunctionsExtract}

\begin{mpFunctionsExtract}
\mpWorksheetFunctionFourNotImplemented
{WORKDAY.INTL? Date? a number that represents a date that is the indicated number of working days before or after a date (the starting date).}
{StartDate? Date? The start date, truncated to integer.}
{Days? Integer? The number of nonweekend and nonholiday days before or after StartDate. A positive value for days yields a future date; a negative value yields a past date.  Day-offset is truncated to an integer.}
{Weekend? Integer? Indicates the days of the week that are weekend days and are not considered working days.}
{Holidays? DateList?  An optional list of one or more dates to exclude from the working calendar}
\end{mpFunctionsExtract}

\section{Date and Time: Calculations}

\begin{mpFunctionsExtract}
\mpWorksheetFunctionThreeNotImplemented
{DAYS360? Date? the number of days between two dates based on a 360-day year (twelve 30-day months), which is used in some accounting calculations}
{StartDate? Date? A date that represents the start date.}
{EndDate? Date? A date that represents the end date}
{Method? Boolean?  A logical value that specifies whether to use the U.S. or European method in the calculation}
\end{mpFunctionsExtract}

\begin{mpFunctionsExtract}
\mpWorksheetFunctionThreeNotImplemented
{NETWORKDAYS? mpReal? the number of whole working days between StartDate and EndDate.}
{StartDate? Date? A date that represents the start date.}
{EndDate? Date? A date that represents the end date}
{Holidays? DateList?  An optional range of one or more dates to exclude from the working calendar, such as state and federal holidays and floating holidays}
\end{mpFunctionsExtract}

\begin{mpFunctionsExtract}
\mpWorksheetFunctionThreeNotImplemented
{NETWORKDAYS.INTL? mpReal? the number of whole working days between StartDate and EndDate.}
{StartDate? Date? A date that represents the start date.}
{EndDate? Date? A date that represents the end date}
{Weekend? String? Indicates the days of the week that are weekend days and are not included in the number of whole working days between StartDate and EndDate.}
{Holidays? DateList?  An optional range of one or more dates to exclude from the working calendar, such as state and federal holidays and floating holidays}
\end{mpFunctionsExtract}

\begin{mpFunctionsExtract}
\mpWorksheetFunctionThreeNotImplemented
{YEARFRAC? mpReal? the fraction of the year represented by the number of whole days between two dates.}
{StartDate? Date? A date that represents the start date.}
{EndDate? Date? A date that represents the end date}
{Basis? DateList?  The type of day count basis to use. Basis Day count basis }
\end{mpFunctionsExtract}

\section{Coupons}

\begin{mpFunctionsExtract}
\mpWorksheetFunctionFourNotImplemented
{COUPDAYBS? mpReal? the number of days from the beginning of the coupon period to the settlement date.}
{SettlementDate? Date? The security's settlement date.}
{MaturityDate? Date? The security's maturity date.}
{Frequency? mpReal? The number of coupon payments per year.}
{Basis? Integer?  The type of day count basis to use}
\end{mpFunctionsExtract}

\begin{mpFunctionsExtract}
\mpWorksheetFunctionFourNotImplemented
{COUPDAYS? mpReal? the number of days in the coupon period that contains the settlement date.}
{SettlementDate? Date? The security's settlement date.}
{MaturityDate? Date? The security's maturity date.}
{Frequency? mpReal? The number of coupon payments per year.}
{Basis? Integer?  The type of day count basis to use}
\end{mpFunctionsExtract}

\begin{mpFunctionsExtract}
\mpWorksheetFunctionFourNotImplemented
{COUPDAYSNC? mpReal? the number of days from the settlement date to the next coupon date.}
{SettlementDate? Date? The security's settlement date.}
{MaturityDate? Date? The security's maturity date.}
{Frequency? mpReal? The number of coupon payments per year.}
{Basis? Integer?  The type of day count basis to use}
\end{mpFunctionsExtract}

\begin{mpFunctionsExtract}
\mpWorksheetFunctionFourNotImplemented
{COUPNCD? Date? a number that represents the next coupon date after the settlement date.}
{SettlementDate? Date? The security's settlement date.}
{MaturityDate? Date? The security's maturity date.}
{Frequency? mpReal? The number of coupon payments per year.}
{Basis? Integer?  The type of day count basis to use}
\end{mpFunctionsExtract}

\begin{mpFunctionsExtract}
\mpWorksheetFunctionFourNotImplemented
{COUPNUM? mpReal? the number of coupons payable between the settlement date and maturity date, rounded up to the nearest whole coupon.}
{SettlementDate? Date? The security's settlement date.}
{MaturityDate? Date? The security's maturity date.}
{Frequency? mpReal? The number of coupon payments per year.}
{Basis? Integer?  The type of day count basis to use}
\end{mpFunctionsExtract}

\begin{mpFunctionsExtract}
\mpWorksheetFunctionFourNotImplemented
{COUPPCD? Date? a number that represents the previous coupon date before the settlement date.}
{SettlementDate? Date? The security's settlement date.}
{MaturityDate? Date? The security's maturity date.}
{Frequency? mpReal? The number of coupon payments per year.}
{Basis? Integer?  The type of day count basis to use}
\end{mpFunctionsExtract}

\begin{mpFunctionsExtract}
\mpWorksheetFunctionSixNotImplemented
{DURATION? mpReal? the Macauley duration for an assumed par value of \$100.}
{SettlementDate? Date? The security's settlement date.}
{MaturityDate? Date? The security's maturity date.}
{Coupon? Integer?  Coupon payments per year}
{Yield? Integer?  The security's annual yield.}
{Frequency? mpReal? The number of coupon payments per year.}
{Basis? Integer?  The type of day count basis to use}
\end{mpFunctionsExtract}

\begin{mpFunctionsExtract}
\mpWorksheetFunctionSixNotImplemented
{MDURATION? mpReal? the modified Macauley duration for a security with an assumed par value of \$100.}
{SettlementDate? Date? The security's settlement date.}
{MaturityDate? Date? The security's maturity date.}
{Coupon? Integer?  Coupon payments per year}
{Yield? Integer?  The security's annual yield.}
{Frequency? mpReal? The number of coupon payments per year.}
{Basis? Integer?  The type of day count basis to use}
\end{mpFunctionsExtract}

\section{Securities}

\begin{mpFunctionsExtract}
\mpWorksheetFunctionEightNotImplemented
{ACCRINT? mpReal? the accrued interest for a security that pays periodic interest.}
{Issue? Date? The security's issue date.}
{First\_interest? Date? The security's first interest date.}
{Settlement? Date?  The security's settlement date.}
{Rate? Integer?  The security's annual coupon rate.}
{Par? mpReal? The security's par value}
{Frequency? Integer? The number of coupon payments per year}
{Basis? Integer?  The type of day count basis to use}
{Calc\_method? Boolean?  The type calculation to use}
\end{mpFunctionsExtract}

\begin{mpFunctionsExtract}
\mpWorksheetFunctionFiveNotImplemented
{ACCRINTM? mpReal? the accrued interest for a security that pays interest at maturity.}
{Issue? Date? The security's issue date.}
{Settlement? Date?  The security's settlement date.}
{Rate? Integer?  The security's annual coupon rate.}
{Par? mpReal? The security's par value}
{Basis? Integer? The type of day count basis to use.}
\end{mpFunctionsExtract}

\begin{mpFunctionsExtract}
\mpWorksheetFunctionFiveNotImplemented
{DISC? mpReal? the discount rate for a security.}
{Settlement? Date?  The security's settlement date.}
{Maturity? Date? The security's maturity date.}
{Pr? Integer?  The security's price per \$100 face value.}
{Redemption? mpReal? The security's redemption value per \$100 face value.}
{Basis? Integer? The type of day count basis to use.}
\end{mpFunctionsExtract}

\begin{mpFunctionsExtract}
\mpWorksheetFunctionFiveNotImplemented
{INTRATE? mpReal? the interest rate for a fully invested security.}
{Settlement? Date?  The security's settlement date.}
{Maturity? Date? The security's maturity date.}
{Investment? mpReal?  The amount invested in the security.}
{Redemption? mpReal? The amount to be received at maturity.}
{Basis? Integer? The type of day count basis to use.}
\end{mpFunctionsExtract}

\begin{mpFunctionsExtract}
\mpWorksheetFunctionEightNotImplemented
{ODDFPRICE? mpReal? the price per \$100 face value of a security having an odd (short or long) first period.}
{Settlement? Date?  The security's settlement date.}
{Maturity? Date? The security's maturity date.}
{Issue? Date?  The security's maturity date.}
{First\_Coupon? Date? The security's first coupon date.}
{Rate? mpReal? The security's interest rate.}
{Yld? mpReal? The security's annual yield.}
{Redemption? mpReal? The security's redemption value per \$100 face value.}
{Frequency? Integer? The number of coupon payments per year}
%{Basis? Integer? The type of day count basis to use.}
\end{mpFunctionsExtract}

\begin{mpFunctionsExtract}
\mpWorksheetFunctionEightNotImplemented
{ODDFYIELD? mpReal? the yield of a security that has an odd (short or long) first period.}
{Settlement? Date?  The security's settlement date.}
{Maturity? Date? The security's maturity date.}
{Issue? Date?  The security's maturity date.}
{First\_Coupon? Date? The security's first coupon date.}
{Rate? mpReal? The security's interest rate.}
{Pr? mpReal? The security's price.}
{Redemption? mpReal? The security's redemption value per \$100 face value.}
{Frequency? Integer? The number of coupon payments per year}
%{Basis? Integer? The type of day count basis to use.}
\end{mpFunctionsExtract}

\begin{mpFunctionsExtract}
\mpWorksheetFunctionEightNotImplemented
{ODDLPRICE? mpReal? the price per \$100 face value of a security having an odd (short or long) last coupon period.}
{Settlement? Date?  The security's settlement date.}
{Maturity? Date? The security's maturity date.}
{Issue? Date?  The security's maturity date.}
{Last\_interest? Date? The security's last coupon date.}
{Rate? mpReal? The security's interest rate.}
{Yld? mpReal? The security's price.}
{Redemption? mpReal? The security's redemption value per \$100 face value.}
{Frequency? Integer? The number of coupon payments per year}
%{Basis? Integer? The type of day count basis to use.}
\end{mpFunctionsExtract}

\begin{mpFunctionsExtract}
\mpWorksheetFunctionEightNotImplemented
{ODDLYIELD? mpReal? the yield of a security that has an odd (short or long) last coupon period.}
{Settlement? Date?  The security's settlement date.}
{Maturity? Date? The security's maturity date.}
{Last\_interest? Date? The security's last coupon date.}
{Rate? mpReal? The security's interest rate.}
{Pr? mpReal? The security's price.}
{Redemption? mpReal? The security's redemption value per \$100 face value.}
{Frequency? Integer? The number of coupon payments per year}
{Basis? Integer? The type of day count basis to use.}
\end{mpFunctionsExtract}

\begin{mpFunctionsExtract}
\mpWorksheetFunctionSevenNotImplemented
{PRICE? mpReal? the price per \$100 face value of a security that pays periodic interest.}
{Settlement? Date?  The security's settlement date.}
{Maturity? Date? The security's maturity date.}
{Rate? mpReal? The security's interest rate.}
{Yld? mpReal? The security's annual yield.}
{Redemption? mpReal? The security's redemption value per \$100 face value.}
{Frequency? Integer? The number of coupon payments per year}
{Basis? Integer? The type of day count basis to use.}
\end{mpFunctionsExtract}

\begin{mpFunctionsExtract}
\mpWorksheetFunctionFiveNotImplemented
{PRICEDISC? mpReal? Returns the price per \$100 face value of a discounted security.}
{Settlement? Date?  The security's settlement date.}
{Maturity? Date? The security's maturity date.}
{Discount? mpReal? The security's interest rate.}
{Redemption? mpReal? The security's redemption value per \$100 face value.}
{Basis? Integer? The type of day count basis to use.}
\end{mpFunctionsExtract}

\begin{mpFunctionsExtract}
\mpWorksheetFunctionSixNotImplemented
{PRICEMAT? mpReal? the price per \$100 face value  of a security that pays interest at maturity.}
{Settlement? Date?  The security's settlement date.}
{Maturity? Date? The security's maturity date.}
{Issue? Date? The security's issue date.}
{Rate? mpReal? The security's interest rate.}
{Yld? mpReal? The security's annual yield.}
{Basis? Integer? The type of day count basis to use.}
\end{mpFunctionsExtract}

\begin{mpFunctionsExtract}
\mpWorksheetFunctionFiveNotImplemented
{RECEIVED? mpReal? the amount received at maturity for a fully invested security.}
{Settlement? Date?  The security's settlement date.}
{Maturity? Date? The security's maturity date.}
{Investment? mpReal? The amount invested in the security.}
{Discount? mpReal? The security's discount rate.}
{Basis? Integer? The type of day count basis to use.}
\end{mpFunctionsExtract}

\begin{mpFunctionsExtract}
\mpWorksheetFunctionSevenNotImplemented
{YIELD? mpReal? the yield on a security that pays periodic interest.}
{Settlement? Date?  The security's settlement date.}
{Maturity? Date? The security's maturity date.}
{Rate? mpReal? The security's interest rate.}
{Pr? mpReal? The security's price per \$100 face value.}
{Redemption? mpReal? The security's redemption value per \$100 face value.}
{Frequency? Integer? The number of coupon payments per year}
{Basis? Integer? The type of day count basis to use.}
\end{mpFunctionsExtract}

\begin{mpFunctionsExtract}
\mpWorksheetFunctionFiveNotImplemented
{YIELDDISC? mpReal? the annual yield for a discounted security.}
{Settlement? Date?  The security's settlement date.}
{Maturity? Date? The security's maturity date.}
{Pr? mpReal? The security's price per \$100 face value.}
{Redemption? mpReal? The security's redemption value per \$100 face value.}
{Basis? Integer? The type of day count basis to use.}
\end{mpFunctionsExtract}

\begin{mpFunctionsExtract}
\mpWorksheetFunctionSixNotImplemented
{YIELDMAT? mpReal? the price per \$100 face value  of a security that pays interest at maturity.}
{Settlement? Date?  The security's settlement date.}
{Maturity? Date? The security's maturity date.}
{Issue? Date? The security's issue date.}
{Rate? mpReal? The security's interest rate.}
{Pr? mpReal? The security's price per \$100 face value.}
{Basis? Integer? The type of day count basis to use.}
\end{mpFunctionsExtract}

\section{Treasury Bills}

\begin{mpFunctionsExtract}
\mpWorksheetFunctionThreeNotImplemented
{TBILLEQ? mpReal? the bond-equivalent yield for a Treasury bill.}
{Settlement? Date?  The Treasury bill's settlement date.}
{Maturity? Date? The Treasury bill's maturity date.}
{Discount? mpReal? The Treasury bill's discount rate.}
\end{mpFunctionsExtract}

\begin{mpFunctionsExtract}
\mpWorksheetFunctionThreeNotImplemented
{TBILLPRICE? mpReal? the price per \$100 face value for a Treasury bill.}
{Settlement? Date?  The Treasury bill's settlement date.}
{Maturity? Date? The Treasury bill's maturity date.}
{Discount? mpReal? The Treasury bill's discount rate.}
\end{mpFunctionsExtract}

\begin{mpFunctionsExtract}
\mpWorksheetFunctionThreeNotImplemented
{TBILLYIELD? mpReal? the yield for a Treasury bill.}
{Settlement? Date?  The Treasury bill's settlement date.}
{Maturity? Date? The Treasury bill's maturity date.}
{Pr? mpReal? The Treasury bill's price per \$face value.}
\end{mpFunctionsExtract}

\section{Depreciation Functions}

\begin{mpFunctionsExtract}
\mpWorksheetFunctionFiveNotImplemented
{DDB? mpReal? the depreciation of an asset for a specified period using the double-declining balance method or some other method you specify.}
{Cost? mpReal?  The initial cost of the asset.}
{Salvage? mpReal? The salvage value at the end of the life of the asset.}
{Life? mpReal? The number of periods over which the asset is being depreciated.}
{Period? mpReal? The period for which you want to calculate the depreciation.}
{Factor? mpReal? The rate at which the balance declines.}
\end{mpFunctionsExtract}

\begin{mpFunctionsExtract}
\mpWorksheetFunctionThreeNotImplemented
{SLN? mpReal? the straight-line depreciation of an asset for a single period}
{Cost? mpReal?  The initial cost of the asset.}
{Salvage? mpReal? The salvage value at the end of the life of the asset.}
{Life? mpReal? The number of periods over which the asset is being depreciated.}
\end{mpFunctionsExtract}

\begin{mpFunctionsExtract}
\mpWorksheetFunctionFourNotImplemented
{SYD? mpReal? the sum-of-years' digits depreciation of an asset for a specified period.}
{Cost? mpReal?  The initial cost of the asset.}
{Salvage? mpReal? The salvage value at the end of the life of the asset.}
{Life? mpReal? The number of periods over which the asset is being depreciated.}
{Period? mpReal? The period for which you want to calculate the depreciation.}
\end{mpFunctionsExtract}

\begin{mpFunctionsExtract}
\mpWorksheetFunctionFiveNotImplemented
{DB? mpReal? the depreciation of an asset for a specified period using the fixed-declining balance method.}
{Cost? mpReal?  The initial cost of the asset.}
{Salvage? mpReal? The salvage value at the end of the life of the asset.}
{Life? mpReal? The number of periods over which the asset is being depreciated.}
{Period? mpReal? The period for which you want to calculate the depreciation.}
{Month? mpReal? The number of months in the first year.}
\end{mpFunctionsExtract}

\begin{mpFunctionsExtract}
\mpWorksheetFunctionSevenNotImplemented
{VDB? mpReal? the depreciation of an asset for any period you specify, including partial periods, using the double-declining balance method or some other method you specify. VDB stands for variable declining balance.}
{Cost? mpReal?  The initial cost of the asset.}
{Salvage? mpReal? The salvage value at the end of the life of the asset.}
{Life? mpReal? The number of periods over which the asset is being depreciated.}
{Start\_Period? mpReal? The period for which you want to calculate the depreciation.}
{End\_Period? mpReal? The ending period for which you want to calculate the depreciation. EndPeriod must use the same units as life.}
{Factor? mpReal? The rate at which the balance declines.}
{No\_switch? mpReal? A logical value specifying whether to switch to straight-line depreciation when depreciation is greater than the declining balance calculation.}
\end{mpFunctionsExtract}

\begin{mpFunctionsExtract}
\mpWorksheetFunctionSevenNotImplemented
{AMORLINC? mpReal? the depreciation for each accounting period.}
{Cost? mpReal?  The initial cost of the asset.}
{Date\_Purchased? Date? The date the asset is purchased.}
{First\_Period? mpReal? The date of the end of the first period.}
{Salvage? mpReal? The salvage value at the end of the life of the asset.}
{Period? mpReal? The period.}
{Rate? mpReal? The rate of depreciation.}
{Basis? mpReal? Year Basis: 0 for 360 days, 1 for actual, 3 for 365 days.}
\end{mpFunctionsExtract}

\begin{mpFunctionsExtract}
\mpWorksheetFunctionSevenNotImplemented
{AMORDEGRC? mpReal? the prorated linear depreciation of an asset for each accounting period.}
{Cost? mpReal?  The initial cost of the asset.}
{Date\_Purchased? Date? The date the asset is purchased.}
{First\_Period? mpReal? The date of the end of the first period.}
{Salvage? mpReal? The salvage value at the end of the life of the asset.}
{Period? mpReal? The period.}
{Rate? mpReal? The rate of depreciation.}
{Basis? mpReal? Year Basis: 0 for 360 days, 1 for actual, 3 for 365 days.}
\end{mpFunctionsExtract}

\section{Annuity Functions}

\begin{mpFunctionsExtract}
\mpWorksheetFunctionFiveNotImplemented
{FV? mpReal? the future value of an annuity based on periodic fixed payments and a fixed interest rate.}
{Rate? mpReal? The the interest rate per period.}
{Nper? mpReal? The total number of payment periods in the investment.}
{Pmt? mpReal? The payment made each period.}
{PV? mpReal? The present value.}
{Type? mpReal? a value representing the timing of payment.}
\end{mpFunctionsExtract}

\begin{mpFunctionsExtract}
\mpWorksheetFunctionFiveNotImplemented
{PV? mpReal? the present value of an annuity based on periodic fixed payments to be paid in the future at a fixed interest rate.}
{Rate? mpReal? The interest rate per period.}
{Nper? mpReal? The total number of payment periods in the investment.}
{Pmt? mpReal? The payment made each period.}
{FV? mpReal? The future value.}
{Type? mpReal? a value representing the timing of payment.}
\end{mpFunctionsExtract}

\begin{mpFunctionsExtract}
\mpWorksheetFunctionFiveNotImplemented
{PMT? mpReal? the payment for a loan based on constant payments and a constant interest rate.}
{Rate? mpReal? The interest rate per period.}
{Nper? mpReal? The total number of payment periods in the investment.}
{PV? mpReal? The present value.}
{FV? mpReal? The future value.}
{Type? mpReal? a value representing the timing of payment.}
\end{mpFunctionsExtract}

\begin{mpFunctionsExtract}
\mpWorksheetFunctionFiveNotImplemented
{NPER? mpReal? the number of periods for an investment based on periodic, constant payments and a constant interest rate.}
{Rate? mpReal? The interest rate per period.}
{Pmt? mpReal? The made each period.}
{PV? mpReal? The present value.}
{FV? mpReal? The future value.}
{Type? mpReal? a value representing the timing of payment.}
\end{mpFunctionsExtract}

\begin{mpFunctionsExtract}
\mpWorksheetFunctionThreeNotImplemented
{PDURATION? mpReal?  the number of periods required by an investment to reach a specified value.}
{Rate? mpReal? The interest rate per period.}
{PV? mpReal? The present value.}
{FV? mpReal? The future value.}
\end{mpFunctionsExtract}

\begin{mpFunctionsExtract}
\mpWorksheetFunctionFiveNotImplemented
{RATE? mpReal? the interest rate per period of an annuity}
{Nper? mpReal? The the interest rate per period.}
{Pmt? mpReal? The made each period.}
{PV? mpReal? The present value.}
{FV? mpReal? The future value.}
{Type? mpReal? a value representing the timing of payment.}
\end{mpFunctionsExtract}

\begin{mpFunctionsExtract}
\mpWorksheetFunctionSixNotImplemented
{IPMT? mpReal? the interest payment for a given period for an investment based on periodic, constant payments and a constant interest rate.}
{Rate? mpReal? The interest rate per period.}
{Per? mpReal? The period for which you want to find the interest and must be in the range 1 to NPer}
{Nper? mpReal? The total number of payment periods in the investment.}
{Pmt? mpReal? The payment made each period.}
{FV? mpReal? The future value.}
{Type? mpReal? a value representing the timing of payment.}
\end{mpFunctionsExtract}

\begin{mpFunctionsExtract}
\mpWorksheetFunctionSixNotImplemented
{PPMT? mpReal? the payment on the principal for a given period for an investment based on periodic, constant payments and a constant interest rate.}
{Rate? mpReal? The the interest rate per period.}
{Per? mpReal? The period for which you want to find the interest and must be in the range 1 to NPer}
{Nper? mpReal? The total number of payment periods in the investment.}
{PV? mpReal? The payment made each period.}
{FV? mpReal? The future value.}
{Type? mpReal? a value representing the timing of payment.}
\end{mpFunctionsExtract}

\begin{mpFunctionsExtract}
\mpWorksheetFunctionSixNotImplemented
{CUMIPMT? mpReal? the cumulative interest paid on a loan between StartPeriod and EndPeriod.}
{Rate? mpReal? The interest rate per period.}
{Nper? mpReal? The total number of payment periods in the investment.}
{PV? mpReal? The payment made each period.}
{StartPeriod? mpReal? The first period in the calculation.}
{EndPeriod? mpReal? the last period in the calculation}
{Type? mpReal? a value representing the timing of payment.}
\end{mpFunctionsExtract}

\begin{mpFunctionsExtract}
\mpWorksheetFunctionSixNotImplemented
{CUMPRINC? mpReal? the effective annual interest rate, given the nominal annual interest rate and the number of compounding periods per year.}
{Rate? mpReal? The interest rate per period.}
{Nper? mpReal? The total number of payment periods in the investment.}
{PV? mpReal? The payment made each period.}
{StartPeriod? mpReal? The first period in the calculation.}
{EndPeriod? mpReal? the last period in the calculation}
{Type? mpReal? a value representing the timing of payment.}
\end{mpFunctionsExtract}

\begin{mpFunctionsExtract}
\mpWorksheetFunctionTwoNotImplemented
{EFFECT? mpReal? the effective annual interest rate.}
{NominalRate? mpReal? The nominal interest rate per period.}
{Npery? mpReal? The number of compounding periods per year}
\end{mpFunctionsExtract}

\begin{mpFunctionsExtract}
\mpWorksheetFunctionTwoNotImplemented
{NOMINAL? mpReal? the nominal annual interest rate, given the effective rate and the number of compounding periods per year.}
{EffectiveRate? mpReal? The nominal interest rate per period.}
{Npery? mpReal? The number of compounding periods per year}
\end{mpFunctionsExtract}

\begin{mpFunctionsExtract}
\mpWorksheetFunctionTwoNotImplemented
{FVSCHEDULE? mpReal? the future value of an initial principal after applying a series of compound interest rates. Use FVSCHEDULE to calculate the future value of an investment with a variable or adjustable rate.}
{Principal? mpReal? The present value.}
{Schedule? mpNum? An array of interest rates to apply}
\end{mpFunctionsExtract}

\begin{mpFunctionsExtract}
\mpWorksheetFunctionFourNotImplemented
{ISPMT? mpReal?  the interest paid during a specific period of an investment.}
{Rate? mpReal? The interest rate per period.}
{Per? mpReal? The period for which you want to find the interest and must be in the range 1 to NPer}
{Nper? mpReal? The total number of payment periods in the investment.}
{PV? mpReal? The present value.}
\end{mpFunctionsExtract}

\section{Cash-Flow Functions}

\begin{mpFunctionsExtract}
\mpWorksheetFunctionTwoNotImplemented
{IRR? mpReal? the effective annual interest rate.}
{Values? mpNum? An array which contains numbers for which which the internal rate of return is calculated.}
{Guess? mpReal? An initial guess for the IRR, 0.1 if omitted}
\end{mpFunctionsExtract}

\begin{mpFunctionsExtract}
\mpWorksheetFunctionThreeNotImplemented
{RRI? mpReal? an equivalent interest rate for the growth of an investment}
{Nper? mpReal? The total number of periods for the investment.}
{PV? mpReal? The present value for the investment.}
{PV? mpReal? The future value for the investment.}
\end{mpFunctionsExtract}

\begin{mpFunctionsExtract}
\mpWorksheetFunctionThreeNotImplemented
{MIRR? mpReal? the modified internal rate of return for a series of periodic cash flows. MIRR considers both the cost of the investment and the interest received on reinvestment of cash.}
{Values? mpNum[]? An array that contains numbers that represent a series of payments (negative) and income (positive) at regular periods.}
{FinanceRate? mpReal? The interest rate paid on the money used in the cash flows.}
{ReinvestRate? mpReal? The interest rate received on the money used in the cash flows.}
\end{mpFunctionsExtract}

\begin{mpFunctionsExtract}
\mpWorksheetFunctionTwoNotImplemented
{NPV? mpReal? the net present value of an investment based on a series of periodic cash flows and a discount rate.}
{Rate? mpReal? The total number of periods for the investment.}
{Values? mpNum[]? An array that contains numbers that represent a series of payments (negative) and income (positive) at regular periods.}
\end{mpFunctionsExtract}

\begin{mpFunctionsExtract}
\mpWorksheetFunctionThreeNotImplemented
{XIRR? mpReal? the modified internal rate of return for a series of periodic cash flows. MIRR considers both the cost of the investment and the interest received on reinvestment of cash.}
{Values? mpNum[]? An array that contains cash flows that correspond to a schedule of payments in Dates.}
{Dates? mpReal? A schedule of payment dates that correspond to the cash flow payments}
{Guess? mpReal? An initial guess for XIRR.}
\end{mpFunctionsExtract}

\begin{mpFunctionsExtract}
\mpWorksheetFunctionThreeNotImplemented
{XNPV? mpReal? the modified internal rate of return for a series of periodic cash flows. MIRR considers both the cost of the investment and the interest received on reinvestment of cash.}
{Rate? mpReal? The discount rate to aplly to the cash flows.}
{Values? mpNum[]? An array that contains cash flows that correspond to a schedule of payments in Dates.}
{Dates? mpReal? A schedule of payment dates that correspond to the cash flow payments}
\end{mpFunctionsExtract}

\section{Conversion}

\begin{mpFunctionsExtract}
\mpWorksheetFunctionTwoNotImplemented
{DOLLARDE? mpReal? a dollar price expressed as a decimal number, converted from a dollar price expressed as an integer part and a fraction part.}
{FractionalDollar? mpReal? A number expressed as a fraction.}
{Fraction? mpReal? The integer to use in the denominator of the fraction}
\end{mpFunctionsExtract}

\begin{mpFunctionsExtract}
\mpWorksheetFunctionTwoNotImplemented
{DOLLARFR? mpReal?  a dollar price expressed as a fraction, converted from a dollar price expressed as a decimal number.}
{DecimalDollar? mpReal? A decimal number.}
{Fraction? mpReal? The integer to use in the denominator of the fraction}
\end{mpFunctionsExtract}

\chapter{Python}

\section{Overview}

\section{CPython}

\section{PyPy: a fast alternative Python interpreter}

\section{Jython: Python for the Java Virtual Machine}

\section{IronPython: Python for .NET Framework and Mono}

\section{Pythonista: Python for iOS}

\section{QPython: Python for Android}

\section{Brython: a Python to JavaScript compiler}

\chapter{LibreOffice Calc and Microsoft Excel}

\section{LibreOffice Calc}

\section{Microsoft Excel}

\section{Spreadsheet functions implemented in mpFormulaPy}

\chapter{Languages with CLR Support}

\section{Visual Basic .NET}

\section{\texorpdfstring {$\text {C\# 4.0 } $}{CSharp}}

\section{JScript 10.0}

\section{C++ 10.0, Visual Studio}

\section{\texorpdfstring {$\text {F\# 3.0 } $}{FSharp}}

\section{MatLab (.NET interface)}

\chapter{Building the library and toolbox}

\section{Building and installing the numerical library}

\section{Building the documentation and standard interfaces}

\section{Building the specific interfaces}

\section{Building the Toolbox GUI}

\section{Other Software}

\section{To Do}

\chapter{Acknowledgements}

\section{Contributors to libraries used in the numerical routines}

\section{Contributors to libraries used in the GUI}

\chapter{Licenses}

\section{GNU Licenses}

\section{Other Licenses}

\chapter{Mathematical Notation}

\chapter{\tocbibname }

\end{document}
